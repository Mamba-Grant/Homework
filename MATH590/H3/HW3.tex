
\documentclass[12pt]{article}

% Packages for mathematical symbols, equations, and formatting
\usepackage{amsmath, amssymb, amsthm}
\usepackage{geometry} % Adjust page margins
\usepackage{graphicx} % Include images
\usepackage{enumitem} % Customize lists
\usepackage{titling}  % Customize title placement

% Page setup
\geometry{letterpaper, margin=1in}
\pagestyle{plain}

% Adjust title placement
\setlength{\droptitle}{-3cm} % Move title up
\posttitle{\par\end{center}} % Remove additional space after title

% Title information
\title{HW2}
\author{Grant Saggars}
\date{\today}

\begin{document}

\maketitle

\section*{Problem 1}
Let $S=\left\{\begin{bmatrix} 1 \\ 2 \\ 3 \end{bmatrix},\begin{bmatrix} 4 \\ 5 \\ 6 \end{bmatrix},\begin{bmatrix} 7 \\ 8 \\ 9 \end{bmatrix},\begin{bmatrix} 1 \\ 1 \\ 1 \end{bmatrix}\right\}$. Show that $\operatorname{Span}(S)$ has dimension two.

\[\begin{split} 
    \begin{bmatrix} 1 & 4 & 7 & 1 \\ 2 & 5 & 8 & 1 \\ 3 & 6 & 9 & 1 \end{bmatrix} \to \begin{bmatrix} 1 & 4 & 7 & 1 \\ 0 & -3 & -6 & -1 \\ 0 & -6 & -12 & -2 \end{bmatrix} \to \begin{bmatrix} 1 & 4 & 7 & 1 \\ 0 & 1 & 2 & 1/3 \\ 0 & 6 & 12 & 2 \end{bmatrix} \to \begin{bmatrix} 1 & 0 & -1 & -1/3 \\ 0 & 1 & 2 & 1/3 \\ 0 & 0 & 0 & 0 \end{bmatrix}
\end{split}\]

Simplifying these vectors by row reduction shows that they are linearly dependent since the third vector can be expressed as a linear combination of the other two, therefore the dimension of this set cannot be $\mathbb{R}^3$ An example of a basis with span S is the first two vectors:

$$\left\{\begin{bmatrix} 1 \\ 2 \\ 3 \end{bmatrix},\begin{bmatrix} 4 \\ 5 \\ 6 \end{bmatrix}\right\}$$

\section*{Problem 2}
Let $H \subset \mathbb{R}^{4}$ be the set of solutions to the equation $x_{1}+2 x_{2}+3 x_{3}+5 x_{4}=0$. Show that $H$ is a subspace of $\mathbb{R}^{4}$ and find a basis for $H$.

\vspace{10pt}
To prove that $H$ is a subspace of $\mathbb{R}^4$, 3 things need to be shown:

\begin{enumerate}
     \item The set is closed under addition:
        \[\begin{split}
            \vec{u}, \vec{v} \in H \implies \vec{u} + \vec{v} \in H \\
            (u_1+v_1) + 2(u_2+v_2) + 3(u_3+v_3) + 5(u_4+v_4) = 0 \\
            \implies (u_1+2u_2+3u_3+5u_4) + (v_1+2v_2+3v_3+5v_4) = 0 + 0 = 0
        \end{split}\]
 
    \item The set is closed under scalar multiplication: 
        \[\begin{split}
            \vec{v} \in V \implies \alpha \vec{v} \in V\, \forall \, \alpha \\
            \alpha(v_1 + 2v_2 + 3v_3 + 5v_4) = c \cdot 0 = 0 
        \end{split}\]
 
    \item The set includes zero:
        \[ 0 + 2(0) + 3(0) + 5(0) = 0 \]
\end{enumerate}


\section*{Problem 3}
Let $U, V$ be subspaces of $\mathbb{R}^{n}$. Let $W$ be the collection of all sums $\vec{u}+\vec{v}$ with $\vec{u} \in U, \vec{v} \in V$. Prove that $W$ is also a subspace.

\begin{enumerate}
    \item Since $u_1$ is in U and $v_1$ is in V, it follows that $u_1 + v_1$ is in W. In class, we discussed that the same would be true for $u_2$ and $v_2$. This means that by induction $\vec{u}$ and $\vec{v}$ are in W.

    \item Let $\vec{w}=\vec{u}+\vec{v}$, where $\vec{w}$ is in W. To show that the set is closed under scalar multiplication, the following must be true:
        \[\begin{split}
            \alpha\vec{w} = \alpha(\vec{u} + \vec{v})
        \end{split}\]

    Because $\vec{u}$ is in U and $\vec{v}$ is in V, by the definition of a subspace $\vec{u}$ and $\vec{v}$ are independently closed under scalar multiplication. This does not change under addition, so the set must be closed under scalar multiplcation.

    \item Finally, the set must include the zero vector, which is trivial since U and V are subspaces, so we know they must contain the zero vector. 
\end{enumerate}

\end{document}

