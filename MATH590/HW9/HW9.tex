\documentclass{article}

\title{Homework}
\author{Grant Saggars}
\date{\today}

\usepackage{amsmath}
\usepackage{import}
\usepackage{pdfpages}
\usepackage{transparent}
\usepackage{xcolor}
\usepackage{framed}
\usepackage{enumerate}
\usepackage{cancel}
\usepackage{silence}
\WarningsOff*
% \ErrorsOff*

\definecolor{shadecolor}{RGB}{248,248,248}

\newenvironment{callout}[1] {\begin{shaded*} \textbf{#1}} {\end{shaded*}}

\begin{document}

\maketitle

\section{Math 590 HW9}

\subsection{Problem 1.} Let T be the following transformation $\mathbf{R}^2 \to \mathbf{R}^2$: projecting onto the line $y = x$, then projecting onto the line $y = 2x$.

\begin{enumerate}[(a)]
\item What is the matrix A for T, using standard basis?

\begin{callout}{Solution:}

    Recall that $proj_v(u) = \displaystyle\frac{u \cdot v}{|v|^2}v$. Second, because we are projecting onto the line $y=x$ first and the line $y=2x$ second, the direction vectors are $[\begin{smallmatrix} 1 \\ 1 \end{smallmatrix}]$ and $[\begin{smallmatrix} 1 \\ 2 \end{smallmatrix}]$ respectively. Therefore:

    \begin{enumerate}
        \item $y=x$:
            \begin{align*}
                \vec{e}_1 &= \frac{[\begin{smallmatrix} 1 \\ 0 \end{smallmatrix}] \cdot [\begin{smallmatrix} 1 \\ 1 \end{smallmatrix}]}{\sqrt{2}}\begin{bmatrix}1 \\ 1\end{bmatrix} = \begin{bmatrix} \sqrt{2} \\ \sqrt{2} \end{bmatrix} \\
                \vec{e}_2 &= \frac{[\begin{smallmatrix} 0 \\ 1 \end{smallmatrix}] \cdot [\begin{smallmatrix} 1 \\ 1 \end{smallmatrix}]}{\sqrt{2}}\begin{bmatrix}1 \\ 1\end{bmatrix} = \begin{bmatrix} \sqrt{2} \\ \sqrt{2} \end{bmatrix} \\
            \end{align*}
        \item $y=2x$:
            \begin{align*}
                \vec{e}_1 &= \frac{[\begin{smallmatrix} 1 \\ 0 \end{smallmatrix}] \cdot [\begin{smallmatrix} 1 \\ 2 \end{smallmatrix}]}{\sqrt{2}}\begin{bmatrix}1 \\ 1\end{bmatrix} = \frac{\sqrt{5}}{2}  \begin{bmatrix} 1 \\ 2 \end{bmatrix} \\
                \vec{e}_2 &= \frac{[\begin{smallmatrix} 0 \\ 1 \end{smallmatrix}] \cdot [\begin{smallmatrix} 1 \\ 2 \end{smallmatrix}]}{\sqrt{2}}\begin{bmatrix}1 \\ 1\end{bmatrix} = \frac{2\sqrt{5}}{5}   \begin{bmatrix} 1 \\ 2 \end{bmatrix} \\
            \end{align*}
    \end{enumerate}

    Finally, the composition of these transformations is the same as applying them sequentially:
    \begin{align*}
        \begin{bmatrix} \frac{\sqrt{5}}{2} & \frac{2\sqrt{5}}{2}  \\ \frac{2\sqrt{5}}{2} & \frac{4\sqrt{5}}{2} \end{bmatrix}
        \begin{bmatrix} \sqrt{2} & \sqrt{2} \\ \sqrt{2} & \sqrt{2} \end{bmatrix} = 
        \begin{bmatrix} \frac{3\sqrt{10}}{2} & \frac{3\sqrt{10}}{2} \\ 3\sqrt{10} & 3\sqrt{10} \end{bmatrix}
    \end{align*}
\end{callout}

\item What are the eigenvalues and eigenspaces of T, using algebraic method and the matrix A?

\begin{callout}{Solution:}
    \begin{enumerate}
        \item Eigenvalues:
            \begin{align*}
                \det\left(\lambda \begin{bmatrix} 1 & 0 \\ 0 & 1 \end{bmatrix} - \begin{bmatrix} \frac{3\sqrt{10}}{2} & \frac{3\sqrt{10}}{2} \\ 3\sqrt{10} & 3\sqrt{10} \end{bmatrix}\right) = 0 \\
                \to \left( \lambda - \frac{3\sqrt{10}}{2} \right)(\lambda - 3\sqrt{10}) - \left(\frac{3\sqrt{10}}{2} \right)(3\sqrt{10}) = 0 \\
                \to \lambda^2 - \lambda \frac{9\sqrt{5}}{\sqrt{2}} = 0 \to \lambda\left(\lambda - \frac{9\sqrt{5}}{\sqrt{2}}\right) = 0 \\
            \end{align*}
                \vspace{-1cm}
            \begin{align*}
                \implies \lambda = \left\{0,\, \frac{9\sqrt{5}}{\sqrt{2}}\right\} \\
            \end{align*}

        \item Eigenvectors:
            \begin{align*}
            \left(\lambda \begin{bmatrix} 1 & 0 \\ 0 & 1 \end{bmatrix} - \begin{bmatrix} \frac{3\sqrt{10}}{2} & \frac{3\sqrt{10}}{2} \\ 3\sqrt{10} & 3\sqrt{10} \end{bmatrix}\right)x = 0 \\
            \end{align*}
                \vspace{-1cm}
            \begin{align*}
                \left[\begin{array}{cc|c} 1 & 1 & 0 \\ 0 & 0 & 0 \end{array}\right] 
                &\implies \begin{bmatrix} -x_2 \\ x_2  \end{bmatrix} \to \begin{bmatrix} -1 \\ 1  \end{bmatrix} \tag{1} \\
                \left[\begin{array}{cc|c} 1 & 1/2 & 0 \\ 0 & 0 & 0 \end{array}\right]
                &\implies \begin{bmatrix} \frac{x_2}{2} \\ x_2  \end{bmatrix} \to \begin{bmatrix} \frac{1}{2} \\ 1  \end{bmatrix} \tag{2}
            \end{align*}
    \end{enumerate}
\end{callout}

\item Explain part (b) using geometry.
    \begin{callout}{Solution:}
        Geometrically eigenvectors represent vectors which do not change direction under the map. We can see that $\begin{bmatrix} \frac{1}{2} \\ 1  \end{bmatrix}$ is only scaled (because we project onto the line $y=x$ first), since it already lies on the line $y=2x$. The other eigenvector, $\begin{bmatrix} -1 \\ 1  \end{bmatrix}$ is mapped to zero, because it is perpendicular to the first map to the line $y=x$.
    \end{callout}
\end{enumerate}

\subsection{Problem 2.} Let A be the matrix in Problem 1. Find $A^{2023}$ by:

\begin{enumerate}[(a)]
\item Using geometry, using the description of T, and results from Problem 1.

\begin{callout}{Solution:}
    
    The final transformation represents a projection onto the line $y=2x$, so the vectors will not change direction after applying the transformation repeatedly. However, because it is projected onto the line $y=x$ before projecting onto $y=2x$, the vector will be scaled after each iteration. 
\end{callout}

\item Using algebra, via diagonalization.

\begin{callout}{Solution:}

    The diagonal matrix is $\left[\begin{smallmatrix} 0 & 0 \\ 0 & \frac{9\sqrt{5}}{\sqrt{2}}  \end{smallmatrix}\right]$. Therefore, $A^{2023}$ is equal to:
    \begin{align*} 
        \begin{bmatrix} 0 & 0 \\ 0 & \left(\frac{9\sqrt{5}}{\sqrt{2}}\right)^{2023}  \end{bmatrix} 
        % \begin{bmatrix} 0 & 0 \\ 0 & \left(\frac{9\sqrt{5}}{\sqrt{2}}\right)^{2023}  \end{bmatrix}
    \end{align*}
\end{callout}
\end{enumerate}

\end{document}
