\documentclass{article}


   \usepackage{amsmath}
   \usepackage{import}
   \usepackage{pdfpages}
   \usepackage{transparent}
   \usepackage{xcolor}
   \usepackage{framed}
   \usepackage{enumerate}


\newcommand{\incfig}[2][1]{%
        \def\svgwidth{#1\columnwidth}
        \import{./figures/}{#2.pdf_tex}
}

\definecolor{shadecolor}{RGB}{248,248,248}

\newenvironment{callout}[1]
{\begin{shaded*}
\textbf{#1}
}
{\end{shaded*}}

\pdfsuppresswarningpagegroup=1

\begin{document}

\section{Math 590 HW8}

\subsection{Problem 1.} Let 
$A = \begin{bmatrix}
a_{11} & a_{12} & a_{13}\\  
a_{21} & a_{22} & a_{23}\\
a_{31} & a_{32} & a_{33}
\end{bmatrix}$.
Let $C_{ij}$ be the matrix obtained from A by deleting row $i$, column $j$ and $D_{ij} = \det(C_{ij})$. Show by direct computations that:

\begin{enumerate}[(a)]
\item $a_{11}D_{11} - a_{12}D_{12} + a_{13}D_{13} = \det(A)$.

\begin{callout}{Solution:}

For a 3x3 matrix, we can show that the cofactor method of determining the determinant is equivalent to the triple product method of finding the determinant:

    \begin{enumerate}[1.]
        \item By Cofactor Expansion:
            \begin{align*}
                a_{11}(a_{22}a_{33}-a_{23}a_{32})-a_{12}(a_{21}a_{33}-a_{31}a_{23}) + a_{13}(a_{21}a_{32}-a_{22}a_{31}) \\
            \end{align*}
        \item By Triple Product:
            \begin{align*}
                \begin{bmatrix} a_{12} \\ a_{22} \\ a_{32} \end{bmatrix} \times \begin{bmatrix} a_{13} \\ a_{23} \\ a_{33} \end{bmatrix} = 
                (a_{22}a_{33}-a_{23}a_{32})-(a_{21}a_{33}-a_{31}a_{23}) + (a_{21}a_{32}-a_{22}a_{31}) \\
                \begin{bmatrix} a_{11} \\ a_{21} \\ a_{31} \end{bmatrix} \cdot \left(\begin{bmatrix} a_{12} \\ a_{22} \\ a_{32} \end{bmatrix} \times \begin{bmatrix} a_{13} \\ a_{23} \\ a_{33} \end{bmatrix}\right) = \\
                a_{11}(a_{22}a_{33}-a_{23}a_{32})-a_{12}(a_{21}a_{33}-a_{31}a_{23}) + a_{13}(a_{21}a_{32}-a_{22}a_{31}) \\
            \end{align*}
    \end{enumerate}

    Given that these two are equivalent, we've symbolically computed that this is a valid method to find the determinant.

\end{callout}

\item $a_{21}D_{11} - a_{22}D_{12} + a_{23}D_{13} = 0$.

\begin{callout}{Solution:}

    We can see that the following would cancel to zero below:
    \begin{align*}
        a_{21}(a_{22}a_{33}-a_{23}a_{32})-a_{22}(a_{21}a_{33}-a_{31}a_{23}) + a_{23}(a_{21}a_{32}-a_{22}a_{31})
    \end{align*}
\end{callout}

\end{enumerate}

\subsection{Problem 2.} Let 
$A = \begin{bmatrix} 
1 & 1 & 0\\
1 & 0 & 1\\
0 & 1 & 1
\end{bmatrix}$.

\begin{enumerate}[(a)]
\item Write A and $A^{-1}$ as product of elementary matrices.  

\begin{callout}{Solution:}
    \begin{enumerate}[1.]
        \item $A$:
        \begin{align*}
            \begin{bmatrix} 1 & 0 & 0 \\ 0 & 1 & 0 \\ 0 & 0 & 1 \end{bmatrix}
            \begin{bmatrix} 1 & 1 & 0 \\ 1 & 0 & 1 \\ 0 & 1 & 1 \end{bmatrix}
        \end{align*}
        \item $A^{-1}$:
        \begin{align*}
            \begin{bmatrix} 1 & 0 & 0 \\ 0 & 0 & 1 \\ 0 & 1 & 0 \end{bmatrix}
            \begin{bmatrix} 1 & 0 & 0 \\ 0 & 1 & 0 \\ 0 & -1 & 1 \end{bmatrix}
            \begin{bmatrix} 1 & 1 & 0 \\ 0 & 1 & 0 \\ 0 & 0 & 1 \end{bmatrix}
            \begin{bmatrix} 1 & 0 & 0 \\ 0 & 1/2 & 0 \\ 0 & 0 & 1 \end{bmatrix}
            \begin{bmatrix} 1 & 0 & -3 \\ 0 & 1 & 0 \\ 0 & 0 & 1 \end{bmatrix}
            \begin{bmatrix} 1 & 0 & 0 \\ 0 & 1 & 1 \\ 0 & 0 & 1 \end{bmatrix}
            \begin{bmatrix} 1 & 0 & 0 \\ -1 & 1 & 0 \\ 0 & 0 & 1 \end{bmatrix}
            \begin{bmatrix} 1 & 1 & 0 \\ 1 & 0 & 1 \\ 0 & 1 & 1 \end{bmatrix}
        \end{align*}
    \end{enumerate}
\end{callout}

\item Find $A^{-1}$ using row reduction.

\begin{callout}{Solution:}

    \includegraphics[width=\textwidth]{InverseRREF.jpg}
\end{callout}

\newpage
\item Find $A^{-1}$ using cofactor matrices.

\begin{callout}{Solution:}
    \begin{enumerate}[1.]
        \item Cofactor Matrix:
            \begin{align*}
                \begin{bmatrix} -1 & -1 & 1 \\ -1 & 1 & -1 \\ 1 & -1 & -1 \end{bmatrix}
            \end{align*}

        \item Adjoint Matrix:

            The adjoint matrix is the transpose of the cofactor matrix:
            \begin{align*}
                \left(\begin{bmatrix} -1 & -1 & 1 \\ -1 & 1 & -1 \\ 1 & -1 & -1 \end{bmatrix}\right)^{T} =
                \begin{bmatrix} -1 & -1 & 1 \\ -1 & 1 & -1 \\ 1 & -1 & -1 \end{bmatrix}
            \end{align*}

        \item Multiply by Inverse Determinant:

            Given that the determinant is -2:
            \begin{align*}
                -\frac{1}{2} \begin{bmatrix} -1 & -1 & 1 \\ -1 & 1 & -1 \\ 1 & -1 & -1 \end{bmatrix} = 
                \begin{bmatrix} 1/2 & 1/2 & -1/2 \\ 1/2 & -1/2 & 1/2 \\ -1/2 & 1/2 & 1/2 \end{bmatrix}
            \end{align*}
    \end{enumerate}
\end{callout}

\end{enumerate}

\end{document}
