\begin{homeworkProblem}
A signal at the RF port of a mixer has a frequency of 100 MHz, and we wish to down convert this signal to an IF frequency of 30 MHz.

Determine the two local oscillator frequencies that can be used to accomplish this.
\begin{callout}{Solution:}

    We know that a downconversion produces a carrier frequency at the difference between RF and LO frequencies. We may simply solve for our LO frequency:
\begin{align*}
    \omega_{IF} &= |\omega_{RF}-\omega_{LO}| &&\text{(downconversion constraint on frequency)}\\
    \mathrm{30~MHz} &= |\mathrm{100~MHz} - \omega_{LO}| &&\text{(substitute values)} \\ 
    \omega_{LO} &= \{\mathrm{70~MHz},\,\mathrm{130~MHz}\}
\end{align*}

\end{callout}
\end{homeworkProblem}

\newpage
\begin{homeworkProblem}
Incident on the RF port of a mixer is one time-harmonic signal, with unknown frequency $f_{R F}$.

Incident on the LO port of this mixer is also one time-harmonic signal, with frequency $f_{\text {LO }}=100 \mathrm{MHz}$.

Exiting the IF port is a number of time-harmonic signals!
One of these signals is a spurious third-order product whose frequency is 30 MHz .

Determine all possible positive values (i.e., $f_{R F}>0$ ) of frequency $f_{R F}$.
\begin{callout}{Solution:}

    Our spurrious $3^{\mathrm{rd}}$ order signals, which I derived in homework \#22 (and in class) are given as:
        \begin{align}
            |2\omega_{RF}-\omega_{LO}| &= \mathrm{30~MHz} \\
            |2\omega_{LO}-\omega_{RF}| &= \mathrm{30~MHz} \\
            3\omega_{RF} &= \mathrm{30~MHz} \\
            3\omega_{LO} &= \mathrm{30~MHz} \\
            (2\omega_{RF}+\omega_{LO}) &= \mathrm{30~MHz} \\
            (\omega_{RF}+2\omega_{LO}) &= \mathrm{30~MHz}
        \end{align}

        So there are only 5 possible values of $\omega_{RF}$, since the absolute values give an extra solution for (1-2), but (4) is independent of $\omega_{RF}$, and (5-6) will give negative frequencies.
        \begin{align*}
            \{ \text{65 MHz, 35 MHz, 170 MHz, 230 MHz, 10 MHz} \}
        \end{align*}

\end{callout}
\end{homeworkProblem}
