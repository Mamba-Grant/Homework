\begin{homeworkProblem}
The output of a lossless microwave filter is terminated in a matched load.

The input impedance of the filter can be expressed as a reflection coefficient :
$$
\Gamma_{\text{in}}(\omega)=\frac{2+j 2 \omega}{j \omega^2-3}
$$

Say a matched source with and available power of 1 mW is connected to the filter input. This source generates a linear eigen function with frequency $\omega=1$.

Determine the power of the wave absorbed by the matched load at the filter output.

\begin{callout}{Solution:}

For this, the goal is to model the transmitted power through a filter. We should expect that for any filter power is nearly delivered in maximum or reflected from the load.
$$\Gamma_{\text{in}}(\omega)\bigg|_{\omega=1} = -0.4-0.8j$$

We know from the lecture "microwave filter design" that a lossless filter can be described in terms of its power reflection coefficient:
\begin{align*}
    |\Gamma_{\text{in}}|^2 &= 1 - T(\omega) \\ 
    (-0.4-0.8j)(-0.4+0.8j) &= 1 - \frac{P_{\text{L}}^{\text{abs}}}{P_{\text{g}}^{\text{avl}}} \\ 
    \frac{P_{\text{L}}^{\text{abs}}}{P_{\text{g}}^{\text{avl}}} &= 0.2 \approx 7 \mathrm{~dB}
\end{align*}

Since the whole system is matched and made of lossless components, it can be taken that all power produced by the source reaches the filter, so this problem is solved:
$$\boxed{P_{\text{L}}^{\text{abs}} = 0.2 \cdot P_{\text{g}}^{\text{avl}} = 0.2 \mathrm{~mW}\approx7\mathrm{~dBm}}$$

\end{callout}
\end{homeworkProblem}

\begin{homeworkProblem}
The complex transfer function for a certain microwave filter has the form:
$$
H(\omega)=\frac{10^7}{10^7+\omega^2} e^{-j[\omega(0.002+A \omega)+B]}
$$
where $A$ and $B$ are some unknown constants.

But, it is know that the phase delay of this filter at frequency $\omega=100$ is 0.004 seconds.

Determine precisely (i.e., without any unknowns!) the phase delay of the filter at $\omega=200$.
\begin{callout}{Solution:}

It seems we have two boundary conditions, $A$ and $B$ but one constraint.
    It seems most productive to me to start by expressing the delay $\tau$: 
    \begin{align*}
        \tau(\omega) = -\frac{\partial~\mathrm{arg}[H(\omega)]}{\partial \omega} = \frac{\partial}{\partial \omega} \big[ \omega (0.002 + A \omega) + B \big] = 0.002+ 2A \omega
    \end{align*}

    It is fortunate that this is independent of $B$, we can determine $A$ from the boundary condition given:
    \begin{align*}
        0.004 = \left[0.002 + 2A (100)\right] \implies A = 10 \times 10^{-6}
    \end{align*}

    This makes computing $\tau(\omega=200)$ trivial:
    \begin{align*}
        \tau(\omega=200) = 0.002~\mathrm{s} + 20\times10^{-6} \mathrm{\frac{s}{rad}} ~(200\mathrm{~rad}) = \boxed{6\mathrm{~ms}}
    \end{align*}

\end{callout}
\end{homeworkProblem}
