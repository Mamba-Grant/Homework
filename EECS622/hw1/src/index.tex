\begin{homeworkProblem}

Consider the circuit:
\input{h1p1f1.tex}

Given the values of $V_g$ and $Z_g$, determine:

\begin{enumerate}[(a)]
    \item The value of load impedance $Z_L$ that will maximize the power source delivered by this source.
        \begin{callout}{Solution:}

        A load impedance with value:
        $$Z_L = Z_g^*$$
        will maximize the power delivered by the source. Therefore, we should choose $Z_L$ equal:
        $$\boxed{Z_L = 40 + j40~\Omega}$$

        \end{callout}
    \newpage
    \item The value of this maximum power.
        \begin{callout}{Solution:}

        Numerically, we will have 
        \begin{align*}
            P_g^{del}\bigg|_{Z_L = Z_g^{*}} &= \frac{1}{2} I_{RMS}I_{RMS}^*R &&\text{(Definition of Power for a Sinusoidal Source)} \\ 
                                            &= \frac{R_g}{2} \left( \frac{V_g}{Z_L + Z_g} \right) \left( \frac{V_g^{*}}{Z_L^{*} + Z_g^{*}} \right)\bigg|_{Z_L = Z_g^{*}} &&\text{(Substitute $I$ and $R$)} \\
                                            &= \frac{R_g}{2} \left( \frac{VV^{*}}{Z_LZ_L^* + Z_L Z_g^* + Z_L^*Z_g + Z_gZ_g^*} \right) \bigg|_{Z_L = Z_g^{*}}  &&\text{(Expand)} \\ 
                                            &= \frac{R_g}{2} \left( \frac{VV^{*}}{Z_gZ_g^* + Z_g Z_g^* + Z_g^*Z_g + Z_gZ_g^*} \right) &&\text{(Apply Boundary)} \\ 
                                            &= \frac{VV^{*}}{8} \left( \frac{R_g}{Z_gZ_g^*} \right) &&\text{(Algebra)} \\ 
                                            &= \frac{VV^{*}}{8R_g} &&\text{($Z_gZ_g^*=|R_g|^2$)} \\ 
        \end{align*}

        This just shows that we have the same circuit as in the lecture. $R_g$ is simply $\mathrm{Re}\{Z_g\}=40~\Omega$. Maximum power is then:
        $$\boxed{P_g^{del} = \frac{(4+j8)(4-j8)}{8 \cdot 40} = 0.25 \mathrm{~\frac{J}{s}}}$$

        \end{callout}
\end{enumerate}

\end{homeworkProblem}

\begin{homeworkProblem}

Now consider the circuit:

\begin{figure}[!ht]
\centering
\resizebox{0.5\textwidth}{!}{%
\begin{circuitikz}
\tikzstyle{every node}=[font=\LARGE]
\draw (5.5,15.25) to[american voltage source] (5.5,14);
\draw (6.25,16) to[R,l={ \LARGE $Z_g = ??$}] (7.75,16);
\draw (5.5,15.25) to[short] (5.5,16);
\draw (5.5,16) to[short] (6.25,16);
\draw (9,15.75) to[R,l={ \LARGE $Z_L = 40 + j40\Omega$}] (9,14.25);
\draw (5.5,14) to[short] (9,14);
\draw (9,14) to[short] (9,14.5);
\draw (9,16) to[short] (7.5,16);
\draw (9,16) to[short] (9,15.5);
\node at (8,16) [circ] {};
\node at (8,14) [circ] {};
\node [font=\LARGE] at (2.75,14.5) {$V_g = 4 + j8V$};


\end{circuitikz}
}%

\label{fig:my_label}
\end{figure}


Given the values of $V_g$ and $Z_g$, determine:

\begin{enumerate}[(a)]
    \item The value of source impedance $Z_g$ that will maximize the power absorbed by load $Z_L$.
        \begin{callout}{Solution:}

It can be shown that the value of source impedance $Z_g$ that maximizes the power absorbed by the load $Z_L$ is-in fact-purely reactive, with value:
$$
\boxed{Z_g = -j X_L}
$$

Therefore, $Z_g = -j40~\Omega$ to maximize the power absorbed by the load.

        \end{callout}
    \item The value of this maximum power.
        \begin{callout}{Solution:}

            With a purely reactive $Z_g$, we should expect total resistance to just be $R_L$:
        \begin{align*}
            Z_{total} = Z_g + Z_L &= (-j X_L) + (R_L + j X_L) \\ 
            &= R_L + (j X_L - jX_L) \\ 
            &= R_L
        \end{align*}
        Great, I can derive the power for this in the same way:
        \begin{align*}
            P_L^{abs} &= \frac{1}{2} I_{RMS}I_{RMS}^{*} R &&\text{(Definition of Power for a Sinusoidal Source)} \\ 
                      &= \frac{V_gV_g^{*}}{2} \left( \frac{R_L}{R_{total}^{2}} \right) &&\text{(Substitute I and R)} \\
                      &= \frac{V_gV_g^{*}}{2} \left( \frac{R_L}{|R_L|^{2}} \right) &&(Z_{total}=R_L) \\
                      &= \frac{V_gV_g^{*}}{2 R_L}
        \end{align*}
        Again, what we got in class. Power delivered is then just:
        $$\boxed{P_L^{abs} = \frac{(4+8i)(4-8i)}{2 \cdot 40} = 1 \mathrm{~ \frac{J}{s}}}$$

        \end{callout}
\end{enumerate}

\end{homeworkProblem}

\begin{homeworkProblem}
$$
\mathcal{Z}=\left[\begin{array}{lll}
1 & j & 1 \\
j & 2 & 2 \\
1 & 2 & 3
\end{array}\right] ~\mathrm{k \Omega}
$$

I will define $I$ pointed into $Z$. Determine both the complex current vector I, and the complex voltage vector V.

\begin{figure}[!ht]
\centering
\resizebox{0.5\textwidth}{!}{%
\begin{circuitikz}
\tikzstyle{every node}=[font=\huge]
\draw (6,12.25) to[american current source,l={ \LARGE $e^{j0} mA $}] (6,14);
\draw (6,14) to[short] (9.75,14);
\draw (6,12.25) to[short] (9.75,12.25);
\draw [ color={rgb,255:red,26; green,95; blue,180} , line width=1.7pt ] (9.75,11.5) rectangle (13.25,15);
\draw [short] (13.25,14) -- (15.25,14);
\draw [short] (13.25,12.25) -- (15.25,12.25);
\draw [short] (15.25,12.25) -- (15.25,14);
\draw [short] (10.25,11.5) -- (10.25,8.75);
\draw [short] (12.75,11.5) -- (12.75,8.75);
\node [font=\large] at (7.25,13.75) {+};
\node [font=\large] at (7.25,12.5) {-};
\node [font=\large] at (7.25,13) {$V_1$};
\node [font=\large] at (9,13) {port 1};
\node [font=\large] at (14,13) {port 2};
\node [font=\large] at (11.5,11) {port 3};
\draw (12.75,8.75) to[R,l={ \LARGE $Z_{L3} = 2.0 k\Omega$}] (10.25,8.75);
\node [font=\large] at (15,13.75) {+};
\node [font=\large] at (15,13) {$V_2$};
\node [font=\large] at (15,12.5) {-};
\node [font=\large] at (10.5,9.5) {+};
\node [font=\large] at (11.5,9.5) {$V_3$};
\node [font=\large] at (12.5,9.5) {-};
\node [font=\huge] at (11.5,13.25) {$Z$};
\node at (7.25,12.25) [circ] {};
\node at (7.25,14) [circ] {};
\node at (15,14) [circ] {};
\node at (15,12.25) [circ] {};
\node at (12.75,9.5) [circ] {};
\node at (10.25,9.5) [circ] {};
\end{circuitikz}
}%

\label{fig:my_label}
\end{figure}

\begin{callout}{Solution:}

We will have a system of 6 equations:
\begin{align}
    V_{1} &= I_{1}+jI_{2}+I_{3} \\ 
    V_{2} &= jI_{1}+2I_{2}+2I_{3} \\ 
    V_{3} &= I_{1}+2I_{2}+3I_{3} \\ 
    I_{1} &= -e^{j0} \\ 
    V_{2} &= 0 \\ 
    V_{3} &= -2I_{3}
\end{align}

Equations 1-3 are given by the impedance matrix. Equations 4-6 are boundary conditions describing the circuit attached to the terminals. The algebra is as follows:
\begin{align*}
    &\begin{cases}
        V_{1} = -e^{j0} + jI_{2} + I_{3} \\
        0 = -je^{j0} + 2I_{2}+ 2I_{3} \\
        -2 I_{3} = -e^{j0} + 2I_{2} + 3I_{3}
    \end{cases} &&\text{(Substite 4-6 into 1-3)} \\ 
    &\begin{cases}
        V_{1} = -1 + jI_{2} + I_{3} \\
        j = 2I_{2}+ 2I_{3} \\
        1-5 I_{3} = 2I_{2}
    \end{cases} &&\text{(Simplify)} \\ 
    &\begin{cases}
        V_{1} = -1 + jI_{2} + I_{3} \\
        \frac{1}{3} - \frac{j}{3} = I_{3} \\
        1-5 I_{3} = 2I_{2}
    \end{cases} &&\text{(Simplify eqn. 2)} \\ 
    &\begin{cases}
        V_{1} = -1 + jI_{2} + I_{3} \\
        \frac{1}{3} - \frac{j}{3} = I_{3} \\
        -\frac{1}{3}+\frac{5}{6}j = I_{2}
    \end{cases} &&\text{(Substite eqn. 2 into 3)} \\ 
    &\begin{cases}
        -\frac{3}{2}-\frac{2}{3}j = V_{1} \\
        \frac{1}{3} - \frac{j}{3} = I_{3} \\
        -\frac{1}{3}+\frac{5}{6}j = I_{2}
    \end{cases} &&\text{(Substite all into eqn. 1)} \\ 
\end{align*}

To summarize:
\begin{gather*}
    V_{1} = -\frac{3}{2}-\frac{2}{3}j \qquad V_{2} = 0 \qquad V_{3} = -\frac{2}{3} + \frac{2}{3}j \\ 
    I_{1} = -1 \qquad I_{2} = -\frac{1}{3}+\frac{5}{6}j \qquad I_{3} = \frac{1}{3} - \frac{j}{3} 
\end{gather*}

\end{callout}
\end{homeworkProblem}
