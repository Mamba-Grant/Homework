\begin{homeworkProblem}
Consider this phase modulated signal:
$$
v_o(t)=\cos \left[\omega_0\left(10^{-6} t^2+t\right)\right]
$$

Determine the relative phase and total frequency of this signal.
\begin{callout}{Solution:}

    \begin{enumerate}[(a)]
        \item If there were no modulation, we would have a signal $v'(t) = \cos[\omega_{0}t]$. The total phase of the modulated signal and unmodulated signal are, respectively:
        \begin{align*}
            \begin{cases}
                    \omega_{0}(10^{-6}t^{2}+t) \\ 
                    \omega_{0}t
            \end{cases}
        \end{align*}

        Which have a difference that is the relative phase (phase noise):
        \begin{align*}
            \varphi_n(t) = \omega_{0}(10^{-6}t^{2} + t) - \omega_{0}t = \omega_{0}10^{-6}t^{2}
        \end{align*}

        Which also checks out, since for most values of $t$, $\varphi_n(t) \ll 1$.

    \item Now, the total frequency is the time derivative of the total phase.
        \begin{align*}
            \omega &= \frac{d |\omega_{0}(10^{-6}t^{2}+t)| }{dt} = \omega_{0}(2\times10^{-6}t + 1)
        \end{align*}
    \end{enumerate}

\end{callout}
\end{homeworkProblem}

\newpage
\begin{homeworkProblem}
Consider two different rates of energy flow-power $P_1$ and power $P_2$.

It is known that the sum of the two values $\left[P_2+P_1\right]$, when expressed with the $d B m$ operator, is :
$$
d B m\left[P_2+P_1\right]=0
$$

And, the difference between the two values $\left[P_2-P_1\right]$ is, when expressed with the $d B m$ operator, is :
$$
d B m\left[P_2-P_1\right]=-7
$$

Determine the value $d B m\left[P_1\right]$
\begin{callout}{Solution:}

This time I will choose to express these algebraically. We have the system of equations:
\begin{align*}
    \begin{cases}
        -10 \log_{10} \left[ \frac{\left( P_{2} + P_{1} \right) \cdot \cancel{\mathrm{1~mW}}}{\cancel{\mathrm{1~mW}}} \right] &= 0 \\
        -10 \log_{10} \left[ \frac{\left( P_{2} - P_{1} \right) \cdot \cancel{\mathrm{1~mW}}}{\cancel{\mathrm{1~mW}}} \right] &= -7
    \end{cases} \\
   \begin{cases}
       P_{2} + P_{1} &= 10^0 \\
       P_{2} - P_{1} &= 10^{0.7} 
   \end{cases} \\
   \left[ \begin{array}{cc|c} 1 & 1 & 10^0 \\ -1 & 1 & 10^{0.7} \end{array} \right] \\
   \left[ \begin{array}{cc|c} 1 & 1 & 10^0 \\ 0 & 1 & \frac{1}{2} (10^{0.7} + 1) \end{array} \right] \\
   \left[ \begin{array}{cc|c} 1 & 0 & 10^0 - \frac{1}{2} (10^{0.7} + 1) \\ 0 & 1 & \frac{1}{2} (10^{0.7} + 1) \end{array} \right]
\end{align*}

We get $P_{1}\approx-2, P_{2}\approx3$. Applying the dBm operator to these gives the final result:
\begin{align*}
    \mathrm{dBm}[P_{1}] = -10 \log_{10} [|-2|] = 3
\end{align*}

\end{callout}

\end{homeworkProblem}
