\documentclass{article}


\newcommand{\hmwkTitle}{Homework 2}
\newcommand{\hmwkDueDate}{\today}
\newcommand{\hmwkClass}{PHSX 611}
\newcommand{\hmwkAuthorName}{\textbf{Grant Saggars}}



\usepackage{fancyhdr}
\usepackage{extramarks}
\usepackage{amsmath}
\usepackage{amsthm}
\usepackage{amsfonts}
\usepackage{tikz}

\usepackage{float}
\usepackage{caption}
\usepackage{bbold}
\usepackage{xcolor}
\usepackage{framed}
\usepackage{enumerate}
\usepackage{cancel}
\usepackage{multicol}
\usepackage{XCharter}

\usetikzlibrary{automata,positioning}

\usepackage{geometry}
\geometry{top=1in, bottom=1in, left=1in, right=1in} % Adjust margins as needed

\pagestyle{fancy}
\lhead{\hmwkAuthorName}
\chead{\hmwkClass\: \hmwkTitle}
\rhead{\firstxmark}
\lfoot{\lastxmark}
\cfoot{\thepage}

%
% Basic Document Settings
%

\topmargin=-0.75in
\evensidemargin=0in
\oddsidemargin=0in
\textwidth=6.5in
\textheight=9.0in
\headsep=0.25in

\linespread{1.1}

\renewcommand\headrulewidth{0.4pt}
\renewcommand\footrulewidth{0.4pt}

\setlength\parindent{0pt}

%
% Create Problem Sections
%

\newcommand{\enterProblemHeader}[1]{
    \nobreak\extramarks{}{Problem \arabic{#1} continued on next page\ldots}\nobreak{}
    \nobreak\extramarks{Problem \arabic{#1} (continued)}{Problem \arabic{#1} continued on next page\ldots}\nobreak{}
}

\newcommand{\exitProblemHeader}[1]{
    \nobreak\extramarks{Problem \arabic{#1} (continued)}{Problem \arabic{#1} continued on next page\ldots}\nobreak{}
    \stepcounter{#1}
    \nobreak\extramarks{Problem \arabic{#1}}{}\nobreak{}
}

\setcounter{secnumdepth}{0}
\newcounter{partCounter}
\newcounter{homeworkProblemCounter}
\setcounter{homeworkProblemCounter}{1}
\nobreak\extramarks{Problem \arabic{homeworkProblemCounter}}{}\nobreak{}

%
% Homework Problem Environment
%
% This environment takes an optional argument. When given, it will adjust the
% problem counter. This is useful for when the problems given for your
% assignment aren't sequential. See the last 3 problems of this template for an
% example.
%
\newenvironment{homeworkProblem}[1][-1]{
    \ifnum#1>0
        \setcounter{homeworkProblemCounter}{#1}
    \fi
    \section{Problem \arabic{homeworkProblemCounter}}
    \setcounter{partCounter}{1}
    \enterProblemHeader{homeworkProblemCounter}
}{
    \exitProblemHeader{homeworkProblemCounter}
}

%
% Callout Box
%

\definecolor{shadecolor}{RGB}{235,235,235}
\newenvironment{callout}[1] {\begin{shaded*} \textbf{#1}} {\end{shaded*}}

%
% Title Page
%

\title{
    \textmd{\textbf{\hmwkClass:\ \hmwkTitle}}\\
    \normalsize\vspace{0.1in}\small{\hmwkDueDate}\\
}

\author{\hmwkAuthorName}
\date{}

\renewcommand{\part}[1]{\textbf{\large Part \Alph{partCounter}}\stepcounter{partCounter}\\}





\begin{document}

\maketitle

\begin{homeworkProblem}
	(Problem 2.1a) Prove the following theorem: for normalizable solutions of S.E., the separation constant $E$ must be real. ( 0.5 pt.)

	\begin{callout}{Solution:}

		%A good place to start is to try proof by contradiction. Suppose a wavefunction exists with complex energy $E_0 + i \Gamma$.
		%
		%\begin{align*}
		%	\Psi(x,t) & = A\psi e^{-i(E_0+i \Gamma )t / \hbar}          \\
		%	          & = A\psi e ^{-iE_0t/\hbar} e ^{\Gamma t / \hbar}
		%\end{align*}
		%
		%Now to try and find the normalization constant $A$:
		%
		%\begin{align*}
		%	1 & = \int_{-\infty}^{\infty} \Psi \Psi^* ~dx                             \\
		%	  & = \int_{-\infty}^{\infty}
		%	( A \psi \cancel{e ^{-iE_0t/\hbar}} e^{\Gamma t / \hbar} )
		%	( A^* \psi^* \cancel{e ^{iE_0t/\hbar}} e^{\Gamma t / \hbar} )
		%	~dx                                                                       \\
		%	  & = \int_{-\infty}^{\infty} |A|^{2}|\psi|^{2}e^{2 \Gamma t / \hbar} ~dx
		%\end{align*}
		%
		%This integral sends $\psi(x)$ to one due to finiteness, and in order to make the equation true the exponential must equal one. Therefore the only possible value $\Gamma $ can take is zero. Therefore there cannot be a complex component in a wavefunction's energy.

	\end{callout}

\end{homeworkProblem}

\begin{homeworkProblem}
	(Problem 2.1b) Prove the following theorem: the time-dependent wave function $\psi(x)$ can always be taken to be real (unlike $\Psi(x, t)$, which is complex in the general case). Note, that this doesn't mean that every solution of the time-independent Schrodinger equation is real; what this means is that if you have one that is not, it can always be expressed as a linear combination of solutions (with the same energy) that are. ( $0.5 \mathrm{pt}$.)

	\newpage \begin{callout}{Solution:}

		If a given $\psi (x)$ is a solution to the wave function, so too is its complex conjugate, since solutions exist for both:
		\begin{gather}
			-\frac{\hbar^{2}}{2m} \frac{d^{2}\psi}{dx^{2}} + E\psi(x) = 0 \\
			-\frac{\hbar^{2}}{2m} \frac{d^{2}\psi^*}{dx^{2}} + E\psi(x) = 0
		\end{gather}

		Now, because a linear combination of any two solutions of the Shr{\"o}dinger equation are in itself a new solution, the linear combination of a solution and its complex conjugate will always eliminate the complex part.

		\begin{align*}
			a+bi + a-bi = 2a
		\end{align*}

	\end{callout}

\end{homeworkProblem}

\begin{homeworkProblem}
	(Problem 2.2) Show that $E$ must exceed the minimum value of $V(x)$, for every normalizable solution to the time-independent Schrodinger equation. ( $0.5 \mathrm{pt}$.)

	\begin{callout}{Solution:}

		For the Shr{\"o}dinger equation $\frac{\partial ^{2} \psi}{\partial x ^{2}} = \frac{2m}{\hbar ^{2}} (V-E)\psi$, it is clear that if $E$ is less than $V$, the right side of the equation will be of the same sign as the left. Griffiths points out the consequence of this is that the second derivative of $\psi$ will always have the same sign as $\psi$. When this is the case, for positive $\psi$ the function will either be concave up or concave down for negative $\psi$ (for all x). This means that the function will either fly to infinity or negative infinity and never converge.
	\end{callout}

\end{homeworkProblem}

\begin{homeworkProblem}
	(Problem 1.8) Consider the following situation - you add a constant $V_0$ to potential energy ( $V_0$ is independent of $x$ and $t$ ). In classical mechanics, this won't change anything but in quantum mechanics, it may. Show that the wavefunction picks up a time-dependent phase factor: $e^{-i V_0 t / \hbar}$. What effect does it have on the expectation value of a dynamic variable? (0.5 pt.)

	\begin{callout}{Solution:}

		$$ \frac{\partial \Psi}{\partial t} = \frac{i \hbar}{2m} \frac{\partial ^{2} \Psi}{\partial x ^{2}} - \frac{i}{h} V(x,t) \Psi(x,t) $$

		I'll substitute a constant potential into the equation, separate, and see what happens:

		\begin{align*}
			\frac{\partial \Psi}{\partial t}                              & = \frac{i \hbar}{2m} \frac{\partial ^{2} \Psi}{\partial x ^{2}} - \frac{i}{\hbar} [V(x,t) + V_0 ] \Psi(x,t)                    \\
			i \hbar \psi \frac{d\phi}{dt}                                 & = - \frac{\hbar}{2m}\frac{\partial ^{2} \psi}{\partial x ^{2}}\phi + \frac{i}{\hbar} V \psi\phi + \frac{i}{\hbar} V_0 \psi\phi \\
			i \hbar \frac{1}{\phi} \frac{d\phi}{dt} - \frac{i V_0}{\hbar} & = - \frac{\hbar}{2m} \frac{\partial ^{2} }{\partial x ^{2}} \frac{1}{\psi} + V
		\end{align*}
		\begin{gather*}
			\frac{i \hbar}{\phi} \frac{d \phi}{dt} - \frac{i V_0}{\hbar} = E \tag{1} \\
			\frac{-\hbar ^{2}}{2m} \frac{d ^{2} \psi}{dx ^{2}} + V\psi = E\psi \tag{2}
		\end{gather*}

		We're focused on the time-dependent part here, (as hinted by Griffiths), so I will integrate to solve the differential equation:

		\begin{align*}
			\int \frac{1}{\phi} ~d\phi & = \frac{-iEt}{\hbar} + \frac{-iV_0t}{\hbar}                                                                                                         \\
			\ln(\phi)                  & =                                                                                                                                                   \\
			\phi                       & = \exp{\left( \frac{-iEt}{\hbar} + \frac{-iV_0t}{\hbar} \right)} =  \exp \left( \frac{-iEt}{\hbar} \right) \exp \left( \frac{-iV_0t}{\hbar} \right)
		\end{align*}

		Now trying an operator $Q$ with the new energy:

		\begin{gather*}
			\int_{-\infty}^{\infty} \left[ \Psi^(x,t) e ^{-iV_0t/\hbar} \right]^* Q \left[\Psi(x,t)e ^{-iV_0t/\hbar} \right] ~dx \\
			= \int_{-\infty}^{\infty} \left[ \Psi^*(x,t) \right] Q \left[\Psi(x,t) \right] \cancel{e^{iV_0t/\hbar}} \cancel{e^{-iV_0t/\hbar}} ~dx \\
		\end{gather*}

		In this case due to the conjugate, there is no difference made by the new potential.


	\end{callout}

\end{homeworkProblem}

\end{document}
