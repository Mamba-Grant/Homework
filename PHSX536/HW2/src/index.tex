\begin{homeworkProblem}
    If a complex voltage and current are related by the expression 
    \[ v(t) = (-1 + j\sqrt{3})i(t) \]
    what is the phase angle in degrees between the voltage and the current? Is the current leading or lagging the voltage?

    \begin{callout}{Solution:}

        $$\theta = \tan^{-1} \left( \frac{\sqrt{3}}{-1} \right) = -60^\circ = 120^\circ$$
        The current is lagging by 120 degrees.

    \end{callout}

\end{homeworkProblem}

\newpage
\begin{homeworkProblem}
    You are given the following data for a potentiometer calibration. (Data file \texttt{potDat.txt} is in the same directory as this \TeX\, file.)

    Perform both linear and quadratic least-squares fits to the data and plot the results. Which function best describes the data? Based on the least-squares analyses, for each fit explain why the functional form is or is not consistent with the data assuming the uncertainties are purely statistical.

    \begin{callout}{Solution:}

        This potentiometer is probably best modeled by the quadratic function. 
        The linear model has a very high chisq value, indicating that it probably does not have enough parameters to fit the data well considering the uncertainties are purely statistical.
        If the linear model had fit this data better, the quadratic fit would not be correct since it would be overfitting. 
        Considering the issues with the linear fit, and that we're seeing a chisq value near 1 with the quadratic model, my intuition tells me that the data is probably quadratic.
        \vspace{0.5cm}

        \centering
        \includegraphics[width=0.85\textwidth]{../myfit.png}

    \end{callout}

\end{homeworkProblem}

\newpage
\begin{homeworkProblem}
    Use LTSpice to explore the transient response of the RC circuit to be studied in Experiment \#3 (see the Week 3 video on BB titled ``LTSpice Time Dependent Sources'').

    \begin{figure}[h!]
        \centering
        \begin{circuitikz}[american]
            \draw (0,0) 
            to[R] (0,1.5)
            to[sV, v=$v_s(t)$] (0,3)
            to[R, l=$R$] (4,3)
            to (5,3)
            node[right] {CH2};

            \draw (4,3)
            node[above] {$V_c$}
            to[C, l=$C$] (4,1.5)
            to (4,0) node[ground]{}
            to (0,0);

            \draw (4,0) -- (5,0);

            \draw (0,3) -- (1, 4) node[above] {CH1};

        \end{circuitikz}
        \caption{RC Circuit Diagram}
    \end{figure}

    Assume $v_s(t)$ varies as a step function that changes from -5V to 5V with a period of 10ms. Take $C = 10 \, \mu$F and choose an $R$ value that will result in a reasonable RC time constant for the laboratory measurement. You will want the capacitor to almost fully charge and discharge during one period. 

    Use \textbf{PROBE} to plot the voltage across the capacitor for $0 \leq t \leq 20$ ms. That is, plot the CH2 voltage as a function of time.

    \begin{callout}{Solution:}

        If we are using the oscilloscope, then 500 ohms or less should be okay.
        \vspace{0.5cm}

        \centering
        \includegraphics[width=0.85\textwidth]{../RC-Circuit.png}
    \end{callout}

\end{homeworkProblem}
