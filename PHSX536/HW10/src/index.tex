\begin{homeworkProblem}
    Q1. Prob. 7.16 (1e: 7.11) Determine the truth table for the circuit below
    \begin{figure}[h]
        \centering
        \includegraphics[width=0.6\textwidth]{../assets/H10P1F1.png}
    \end{figure}
    \begin{callout}{Solution:}

        I will label "X" the line out of the OR gate.
        \center
        \begin{tabular}{|c|c|c|c|}
            \hline
            A & B & X & O \\
            \hline
            0 & 0 & 0 & 0 \\
            0 & 1 & 1 & 1 \\
            1 & 0 & 1 & 0 \\
            1 & 1 & 1 & 1 \\
            \hline
            1 & 0 & 1 & 0 \\
        \end{tabular}


    \end{callout}
\end{homeworkProblem}

\newpage
\begin{homeworkProblem}
    Q2 Prob. 7.17 (1e: 7.12) Determine the truth table for the circuit below
    \begin{figure}[h]
        \centering
        \includegraphics[width=0.6\textwidth]{../assets/H10P2F1.png}
    \end{figure}

    \begin{callout}{Solution:}

        I will label "X" the line out of the NAND gate.
        \center
    \begin{tabular}{|c|c|c|c|}
        \hline
        A & B & X & O \\
        \hline
        0 & 0 & 1 & 1 \\
        0 & 1 & 1 & 1 \\
        1 & 0 & 1 & 1 \\
        1 & 1 & 0 & 1 \\
        \hline
    \end{tabular}


    \end{callout}


\end{homeworkProblem}

\newpage
\begin{homeworkProblem}
    Simulate the circuit shown below and answer the following questions.
\begin{figure}[H]
  \centering
  \includegraphics[width=0.6\textwidth]{../assets/H10P3F1.png}
\end{figure}

    \begin{enumerate}[(a)]
        \item Vary the input voltage between 0 and 5 V and verify that the circuit is consistent with the standard TTL logic

            \begin{callout}{Solution:}

                \begin{figure}[H]
                    \centering
                    \includegraphics[width=0.6\textwidth]{../assets/H10P3F3.png}
                \end{figure}
                \begin{figure}[H]
                    \centering
                    \includegraphics[width=0.8\textwidth]{../assets/H10P3F2.png}
                \end{figure}
                We can see from this that the output is an inverted high/low signal, consistent with the intended logic.

            \end{callout}

        \item Determine whether current is entering or leaving the Q1 transistor
            \begin{callout}{Solution:}

                I have included the current through the base of the indicated transistor in the previous figure, where from 0 to 5 ms we can see low current (entering) and high current (leaving) from 5 to 10ms when there is a high input.

            \end{callout}
        \item Remove the connection to the emitter of the Q1 transistor, which allows the gate input to “float”. Determine whether such an input corresponds to a logical low or high? Explain.
            \begin{callout}{Solution:}

            This seems to produce a constant high output.
\begin{figure}[H]
  \centering
  \includegraphics[width=0.8\textwidth]{../assets/H10P3F4.png}
\end{figure}

            \end{callout}
    \end{enumerate}
\end{homeworkProblem}
