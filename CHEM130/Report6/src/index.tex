\section{Introduction}
An oxidation-reduction (redox) reaction, put simply, is an electron transfer reaction between reactants. In such a reaction, the oxidized element transfers electrons and reduces another element. In the example of this lab, we had:
\begin{align*}
	\textrm{Fe$^{2+}$~(aq) + MnO$_{4}^{-}$~(aq) + H$^{+}$~(aq)
		$\to$
		Fe$^{3+}$~(aq) + Mn$^{2+}$~(aq) + H$_{2}$O~(l) } \tag{1}
\end{align*}

\begin{tiny}
	\begin{center}
		Note that phosphoric acid is included in the reaction to prevent interference of chloride ions in the solution. This reaction involved the oxidation of iron and reduction of MnO$_{4}^{-}$
	\end{center}
\end{tiny}

This lab focused on investigating such reactions using titration. After developing the theory for this reaction, we found the molar ratio for given concentrations of reactants to determine what ratio resulted in a complete reaction. We then applied titration to find the concentration of FeCl$_{2}$ (titration involves slowly adding reactants to find a stopping point for a reaction, in this case).

\section{Procedure}

The first section of the lab involved each group separately testing the molar ratio for FeCl$_{2}$, H$_{3}$PO$_{4}$, and KMnO$_{4}$ by directly mixing different quantities of 0.0097 M KMnO$_{4}$ into a solution of the other two reactants. The second section involved titrating 0.0097 M KMnO$_{4}$ into a solution of 10mL unknown molarity FeCl$_{2}$ and H$_{3}$PO$_{4}$ to eventually determine their respective concentrations. The apparatus consisted of a burette mounted above a Ehrenmire flask containing the two reactants of unknown concentration. A stir rod was used in conjunction. This titration process was repeated 3 times to "hone in" on an accurate ratio.

\newpage \section{Results}
\begin{center}
	\Large Table 1: Section one Data
\end{center}
\begin{table}[ht]
	\centering
	\begin{tabular}{|c|c|c|c|c|c|}
		\hline
		                     & G1                    & G3                    & G4                    & G5                    & G6         \\
		\hline
		Moles                & 4.85$\times$10$^{-4}$ & 1.46$\times$10$^{-4}$ & 1.94$\times$10$^{-4}$ & 2.40$\times$10$^{-4}$ & 2.91       \\
		KMnO$_{4}$           & mol                   & mol                   & mol                   & mol                   & mol        \\
		\hline
		Moles                & 0.001                 & 0.001                 & 0.001                 & 0.001                 & 0.001      \\
		FeCl$_2$             & mol                   & mol                   & mol                   & mol                   & mol        \\
		\hline
		Moles                & 0.06                  & 0.06                  & 0.06                  & 0.06                  & 0.06       \\
		H$_3$PO$_4$          & mol                   & mol                   & mol                   & mol                   & mol        \\
		\hline
		Molar Ratio          & 20.6:1                & 6.9:1                 & 5.2:1                 & 4.2:1                 & 3.4:1      \\
		Fe$^{2+}$: MnO$_{4}$ &                       &                       &                       &                       &            \\
		\hline
		Observed             & faint                 & faint                 & purple                & dark                  & dark       \\
		Color                & yellow                & yellow                &                       & pink                  & red-purple \\
		\hline
	\end{tabular}
	\caption{Measured quantities for each group from part one. Group two did not report their findings. We found that the reaction completed between the data points gathered by groups three and four, so we conclude that the reaction completes with a molar ratio 5:1.}
	\label{tab:data}
\end{table}

\begin{center}
	\Large Calculations 1-3: Moles of Reactants
	\begin{figure}[ht]
		\begin{equation}
			\left(0.02~\frac{\textrm{mol}}{\textrm{L}}\right) (5 \textrm{~ml}) = 0.001 \textrm{~mol}
		\end{equation}
		\begin{equation}
			\left(6~\frac{\textrm{mol}}{\textrm{L}}\right) (10 \textrm{~ml}) = 0.06 \textrm{~mol}
		\end{equation}
		\begin{equation}
			\left(0.0097~\frac{\textrm{mol}}{\textrm{L}}\right) (20 \textrm{~ml}) = 1.94\times10^{-4} \textrm{~mol}
		\end{equation}
		\caption{Calculations from my table's (G4) quantity of reactant moles.}
	\end{figure}
\end{center}

\begin{center}
	\Large Figure 1. Balanced Principal Equation
	\begin{figure}[ht]
		\begin{equation*}
			\textrm{5Fe$^{2+}$~(aq) + MnO$_{4}^{-}$~(aq) + 8H$^{+}$~(aq)
				$\to$
				5Fe$^{3+}$~(aq) + Mn$^{2+}$~(aq) + 4H$_{2}$O~(l) }
		\end{equation*}
		\caption{A balancing of the singular reaction in this experiment.}
	\end{figure}
\end{center}

\newpage
\begin{center}
	\Large Table 2. Titration measurements
	\begin{table}[ht]
		\centering
		\begin{tabular}{|c|c|c|c|}
			\hline
			Trial & Initial Volume (mL) & Final Volume (mL) & Difference (mL) \\
			\hline
			\#1   & 10.8                & 2.5               & 8.3             \\
			\hline
			\#2   & 23.8                & 3.6               & 20              \\
			\hline
			\#3   & 33.6                & 3.5               & 30              \\
			\hline
		\end{tabular}
		\caption{During the lab, the impression was that the data points were relatively close. However, it became apparent that something must have been mixed incorrectly between trials. The data showed a whopping 11 standard deviations, which was rather disastrous. Since the experiment cannot be repeated, the working mean for this experiment was 20 mL.}
	\end{table}
\end{center}

\begin{center}
	\Large Calculation 4: Concentration of FeCl$_{2}$ from one Titration Trial.
	\begin{figure}[ht]
		\begin{gather*}
			\left( 0.0097~\frac{\textrm{mol}}{\textrm{L}} \right)(20 \textrm{~mL}) = 1.94 \times 10^{-4} \textrm{~mol~MnO$_{4}^{-}$} \\
			\implies 0.001 \textrm{~mol~FeCl$_{2}$} \\
			\frac{0.001 \textrm{~mol}}{10 \textrm{~mL}} = 0.1 ~\frac{\textrm{mol}}{\textrm{L}} \tag{4}
		\end{gather*}
		\caption{Using the second trial, we found a molarity of 0.1 for iron (II) chloride.}
	\end{figure}
\end{center}

\begin{center}
	\Large Calculation 5: Percent Error
	\begin{figure}[ht]
		\begin{equation}
			100\times\frac{0.10-0.150}{0.150} = -33.33\% \tag{5}
		\end{equation}
		\caption{There is very substantial error in our calculated value.}
	\end{figure}
\end{center}

\newpage \section{Discussion}
By slowly introducing a reactant of known concentration until the reaction is complete, we can determine concentrations of the other reactants by using their reaction ratio and their known volumes. An indicator of a complete reaction is necessary; however, in our case this was not necessary as the color of the solution is stained by potassium permanganate when the reaction completes. Our reaction involved the oxidation of iron and reduction of MnO$_{4}^{-}$ which are also the reducing and oxidizing agent, respectively (fig. 1). Put simply, electrons were transferred from iron to potassium permanganate.

Redox reactions are common and useful reactions in the real world. Two examples are batteries and corrosion of metals. In our instance of redox chemistry, part one focused on finding the point where the reaction completed using varying volumes of reactants. We found it completes at a ratio of 5 iron (II) to 1 potassium permanganate (tab. 1). Using this ratio we developed a solution with $33\%$ error (calc. 5).

Even without the percent error, the deviation of titration measurements is problematic. Could the experiment be repeated again, patience during this part of the experiment would be the largest factor in reducing error. Besides this, there would ordinarily be 1 mL error in titration measurements due to flow rate. Additionally, it is possible that there may be an error in the initial calculation of molar ratio as this was not determined using titration.

\section{Conclusion}
In this lab we studied the redox reaction of aqueous iron, potassium permanganate, and hydrogen. A key technique used was titration, wherein accurate ratios of reactants can be measured by slow introduction of one reactant into a solution of the others. We found a ratio of 5:1 Fe$^{2+}$ to KMnO$_{4}^{-}$ in part 1, where we then used equation in figure 1 to measure the concentration of unknown molarity Fe$^{2+}$ via titration. This was repeated three times in hopes of getting a precise measurement; however, there were apparent issues in preparing the correct quantities of reactants resulting in low quality measurements.

\section{References}
\begin{enumerate}[(1)]
	\item Dr. Applegate, lab procedure - Oxidation-Reduction Chemistry
\end{enumerate}
