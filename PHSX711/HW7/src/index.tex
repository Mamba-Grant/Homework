\begin{homeworkProblem}
Exercise 11.2.2 (Shankar) – 5 points
Using \( T^\dagger(\epsilon) T(\epsilon) = I \) to order \(\epsilon\), deduce that \( G^\dagger = G \).
\end{homeworkProblem}

\begin{homeworkProblem}
Exercise 11.4.1 (Shankar) – 5 points
Prove that if \( [\Pi, H] = 0 \), a system that starts out in a state of even/odd parity maintains its parity. (Note that since parity is a discrete operation, it has no associated conservation law in classical mechanics.)
\begin{callout}{Solution:}
    
\begin{align*}
    [\Pi, H] &= 0 \implies H(x,p) = H(-x,-p) \\ 
    \Pi \ket{\psi(x)} &= \ket{\psi(-x)} &&\text{(where $\ket{\psi(x)} = \pm \ket{\psi(x)}$)} \\
    \Pi \ket{\psi(x)} &= \pm \ket{\psi(x)} \\ \\
    \Pi \ket{\psi(x,~t)} &= \Pi e^{iHt/\hbar}\ket{\psi(x)} \\ 
    &= e^{iH(\pm x, \pm t)t/\hbar} \ket{\psi(\pm x)} &&(H(x,p)=H(-x,-p)) \\ 
    &= e^{iHt/\hbar} \pm \ket{\psi(x)} \\ 
    &= \pm \ket{\psi(x,t)} \\ 
\end{align*}

\end{callout}
\end{homeworkProblem}

\begin{homeworkProblem}
Exercise 11.4.2 (Shankar) – 5 points
A particle is in a potential
\[
V(x) = V_0 \sin\left(\frac{2\pi x}{a}\right)
\]
which is invariant under the translations \( x \rightarrow x + ma \), where \( m \) is an integer. Is momentum conserved? Why not?

Hint: You can start this problem with:
\[
\frac{d}{dt} \langle P \rangle = -\frac{i}{\hbar} \langle [P, H] \rangle
\]
Whether it goes to zero or not depends on the range you used for averaging \( \langle \dots \rangle \).
\end{homeworkProblem}
