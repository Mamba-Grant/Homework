\begin{homeworkProblem}
    Exercise 12.2.1 (Shankar) -- 5 points

    Provide the steps linking Eq. (12.2.8) to Eq. (12.2.9). [Hint: Recall the derivation of Eq. (11.2.8) from Eq. (11.2.6).]

    \[U[R]\ket{x, y} = \ket{x - y\varepsilon_z, x\varepsilon_z + y}\]
    \[\braket{x, y | I -\frac{i\varepsilon_z L_z}{\hbar}|\psi} = \psi(x + y\varepsilon_z, y - x\varepsilon_z)\]
    \begin{callout}{Solution:}
        In the derivation of rotations, we initially define infinitesimal rotations about $\epsilon_z \textbf{k}$:
        $$U[R(\epsilon_z \textbf{k})]=I-\frac{i \epsilon_z L_z}{\hbar}$$
        We can first examine the hermitian conjugate, where we just end up inverting the sign on the parts multiplied by $\epsilon_z$. 
        Then, we can consider the application of the rotation operator such that $\braket{x,y|U[R]|\psi}$ to rotate the coordinates which we project $\psi$ into.
        This leaves us with a modified projection of $\psi$ onto $x'$ and $y'$.
        \begin{align*}
            \left( U[R]\ket{x,y} \right)^{\dagger} &= \bra{x,y} \left( I+\frac{i \epsilon_z L_z}{\hbar} \right) = \bra{x+y \epsilon_z,~-x \epsilon_z +y} \\
            \braket{x, y | I -\frac{i\varepsilon_z L_z}{\hbar}|\psi} &= \braket{x+y \epsilon_z,~-x \epsilon_z +y|\psi} \\ 
            &= \psi(x+y \epsilon_z, ~y-x \epsilon_z)
        \end{align*}
    \end{callout}
\end{homeworkProblem}

\newpage
\begin{homeworkProblem}
    Exercise 12.2.3 (Shankar) -- 5 points

    Derive Eq. (12.2.19) by doing a coordinate transformation on Eq. (12.2.10), and also by the direct method mentioned above.

    \[L_z\underset{\text{coordinate basis}}{=} -i\hbar\frac{\partial}{\partial\phi}\]

    *You can choose either one of the methods.
    \begin{callout}{Solution:}
        
        Equation 12.2.10 says
        $$L_z = x\left( -i\hbar \frac{\partial}{\partial y} \right) - y\left( -i\hbar \frac{\partial}{\partial x} \right)$$

        Directly doing the coordinate transformation to this is probably more straightforward.
        The only tricky part is working out the transformation of the partial derivatives to polar coordinates for $r=1$ (since we don't want to modify length of our vectors).
        \begin{align*}
            &= \cos \phi \left( -i\hbar \frac{\partial}{\partial \phi} \frac{\partial \phi}{\partial y} \right) - \sin\phi \left( -i\hbar \frac{\partial}{\partial \phi}\frac{\partial \phi}{\partial x} \right) \\
            &= \cos \phi \left( -i\hbar \cos\phi \frac{\partial}{\partial \phi} \right) - \sin\phi \left( -i\hbar \sin\phi \frac{\partial}{\partial \phi} \right) \\
            &= -i\hbar \left( \cos^2\phi + \sin^2\phi \right) \frac{\partial}{\partial \phi} \\ 
            &= -i\hbar \frac{\partial}{\partial \phi}
        \end{align*}
    \end{callout}
\end{homeworkProblem}

\begin{homeworkProblem}
    Exercise 12.5.3 (Shankar) -- 10 points

    (1) Show that $\braket{J_x} = \braket{J_y} = 0$ in a state $\ket{jm}$.
    \begin{callout}{Solution:}
        We have 
        $$J_i = L_i + S_i$$
        Which in the basis of ladder operators is:
        $$J_x = \frac{J_+ + J_-}{2}, \qquad J_y = \frac{J_+-J_-}{2i}$$
        Which we do know how to do calculations with: 
        \begin{align*}
            J_{\pm} \ket{jm} &= \hbar[(j \mp m)(j\pm m+1)]^{1/2}\ket{j,m\pm1}
        \end{align*}
        It is relatively simple to express expectation values:
        \begin{align*}
            \braket{jm|\frac{J_++J_-}{2}|jm} &= \frac{1}{2} \left[ \braket{jm|J_+|jm} + \braket{jm|J_-|jm} \right] \\
            &= \frac{\hbar}{2} \left\{ \left[(j-m)(j+m+1)\right]^{1/2} \cancelto{0}{\braket{jm|j,m+1}} + \left[ (j+m)(j-m+1) \right]^{1/2} \cancelto{0}{\braket{jm|j,m-1}} \right\} \\ 
            &= 0 \\
            \braket{jm|\frac{J_+-J_-}{2i}|jm} &= \frac{1}{2i} \left[ \braket{jm|J_+|jm} - \braket{jm|J_-|jm} \right] \\
            &= \frac{\hbar}{2i} \left\{ \left[(j-m)(j+m+1)\right]^{1/2} \cancelto{0}{\braket{jm|j,m+1}} - \left[ (j+m)(j-m+1) \right]^{1/2} \cancelto{0}{\braket{jm|j,m-1}} \right\} \\
            &= 0
        \end{align*}
        We run into the same situation we do with harmonic oscillator ladder operators, in that our off-diagonal (block diagonal) matrix form results in expectation values of zero when a value gets raised but not lowered, and vise versa due to the delta function.
    \end{callout}

    (2) Show that in these states
    \[\braket{J_x^2} = \braket{J_y^2} = \frac{1}{2}\hbar^2[j(j+1) - m^2]\]
    (use symmetry arguments to relate $\braket{J_x^2}$ to $\braket{J_y^2}$).
    \begin{callout}{Solution:}
        As with the harmonic oscillator ladder operators, squaring the sum of these operators results in two terms that have equal amounts raising and lowering.
        \begin{align*}
            \frac{1}{2}\braket{jm|\left( J_+ + J_- \right)^{2}|jm} &= \frac{1}{2}\braket{jm|\cancelto{0}{J_+^{2}} + \cancelto{0}{J_-^{2}} + J_-J_+ + J_+J_-|jm} \\ 
            -\frac{1}{2}\braket{jm|\left( J_+ - J_- \right)^{2}jm} &= \frac{1}{2}\braket{jm|\cancelto{0}{-J_+^{2}} - \cancelto{0}{J_-^{2}} + J_-J_+ + J_+J_-|jm}
        \end{align*}
        Where the sequence of the one lowering-raising operator and vise-versa is:
        \begin{align*}
            \braket{jm|J_-J_+|jm} &= \hbar [(j-m)(j+m+1)]^{1/2}\braket{jm|J_-| jm+1} \\ 
            &= \hbar^2 [(j-m)(j+m+1)(j+(m+1))(j-(m+1)+1)]^{1/2} \cancelto{1}{\braket{jm|jm}} \\
            &= \hbar^2 \left(j+m+1\right)\left(j-m\right) \\ 
            \braket{jm|J_+J_-|jm} &= \hbar [(j+m)(j-m+1)]^{1/2}\braket{jm|J_+| jm-1} \\ 
            &= \hbar^2 [(j+m)(j-m+1)(j-(m-1))(j+(m-1)+1)]^{1/2} \cancelto{1}{\braket{jm|jm}} \\
            &= \hbar^2 \left(j+m\right)\left(j-m+1\right)
        \end{align*}
        Substituting these back in, 
        \begin{align*}
            \braket{J_x^2} = \frac{1}{2}\braket{jm|\left( J_+ + J_- \right)^{2}|jm} &= \frac{\hbar^2}{2} \left[(j+m+1)(j-m) + (j+m)(j-m+1)\right] = \frac{\hbar^2}{2}\left[ j(j+1)-m^2 \right]
        \end{align*}
        It happens that $\braket{J_y^2}$ is the same, since that extra $i$ in the demoninator when squared makes it identical to $\braket{J_x^2}$.
    \end{callout}
    \begin{callout}{Solution \#2:}        
        A bit more compactly, we have rotational invariance about the $z-axis$, so $\braket{J_x^2}=\braket{J_y^2}$.
        Now we can just work out expectation values for:
        $$J^2-J_z^2 = j_x^2+J_y^2$$
        which is the same:
        $$\braket{J_x^2} = \braket{J_y^2} = \frac{\hbar^2}{2}\left[ j(j+1)-m^2 \right]$$
    \end{callout}

    (3) Check that $\Delta J_x \cdot \Delta J_y$ from part (2) satisfies the inequality imposed by the uncertainty principle [Eq. (9.2.9)].
    \begin{callout}{Solution:}
        We have equation (9.2.9):
        $$(\Delta \Omega)^2 (\Delta \Lambda)^2 \geq |\braket{\psi|\Omega \Lambda|\psi}|^2$$
        where $\Omega \Lambda$ can also be expressed in terms of commutators
        $$\frac{[\Omega,\Lambda]_{+} +[\Omega,\Lambda]}{2}$$
        I think that it will be easier to just directly compute their product, since it is just 
        $$\braket{J_xJ_y} = \frac{1}{4i}\braket{J_-J_+ - J_+J_-} = \frac{m\hbar^2}{2}$$
        Now the left hand side, we have by rotational invariance about the $z$-axis:
        \begin{align*}
            \Delta J_x = \Delta J_y &= \sqrt{ \langle J_x^2 \rangle - \langle J_x \rangle^2 } \\ 
            &= \sqrt{ \frac{\hbar^2}{2}\left[ j(j+1)-m^2 \right] }
        \end{align*}
        So to verify, we have 
        $$\frac{\hbar^2}{2}\left[ j(j+1)-m^2 \right] \geq \frac{m\hbar^2}{2}$$
        We must realize that $j \leq |m| \implies j(j+1) \leq m(m+1)$, so then the equality is met.
    \end{callout}

    (4) Show that the uncertainty bound is saturated in the state $\ket{j, \pm j}$.
    \begin{callout}{Solution:}
        The inequality is only saturated when $j=|m| \implies m = \pm j$. This is the state $\ket{j,\pm j}$
    \end{callout}
\end{homeworkProblem}

\newpage
\begin{homeworkProblem}
    Exercise 12.5.12 (Shankar) -- 6 points

    Since $L^2$ and $L_z$ commute with $\Pi$, they should share a basis with it. Verify that $Y_l^m \underset{\Pi}{\rightarrow} Y_{l\Pi}^{m} = (-1)^l Y_l^m$. 

    \vspace{1em} Hint: (I expanded the hint from the book a little bit)
    \begin{enumerate}
        \item $Y_{l\Pi}^{m}$ in this question represents the function generated from $\Pi Y_l^m$.
        \item Recall (from Chap. 11) that $\Pi$ is an operation that changes $(x, y, z)$ to $(-x, -y, -z)$. For spherical coordinates, $\Pi Y_l^m = Y_{l\Pi}^{m}(\theta, \phi) = Y_l^m(\pi - \theta, \phi + \pi)$
        \item Show the proof for the $Y_l^l$ first.
        \item Then, show that the lower operator $L_-$ (projected in the X-basis) is not altered by the parity operation $(\theta \rightarrow \pi - \theta, \phi \rightarrow \phi + \pi)$
    \end{enumerate}
    \begin{callout}{Solution:}
        
        We have the harmonic function 
        $$Y_l^m(\theta, \phi) = \left[ \frac{(2l+1)(l-m)!}{4\pi (l+m)!} \right]^{1/2} (-1)^m e^{im\phi}P_l^m (\cos\theta)$$

        Starting with (3), we can apply the parity operator to $Y_l^l$.
        \begin{align*}
            Y_l^l(\theta,\phi) &= \left[\frac{2l+1}{4\pi(2l)!}\right]^{1/2} (-1)^{l} e^{il\phi}P_l^l(\cos\theta) \\
            \Pi Y_l^l(\theta,\phi) &= \left[\frac{2l+1}{4\pi(2l)!}\right]^{1/2} (-1)^{l} e^{il(\phi+\pi)}P_l^l(\cos(\pi-\theta)) \\
        \end{align*}
        A key thing to see here is that $P_l^l$ obeys a nice rule, where 
        $$P_\ell^\ell = \sin^\ell \theta = \sin^\ell(\pi-\theta)$$
        %And this is equal to itself when phase shifted by $\pi$. 
        We don't really care about the normalization term since it doesn't change under parity, so under parity just need to know what happens to the exponential $\phi$ term.
        $$e^{il(\phi+\pi)} = \left( e^{i\phi} \right)^le^{il\phi} = (-1)^l e^{il\phi}$$
        Hence, 
        $$\Pi Y_l^l = (-1)^lY_l^l $$
        We can now lower the harmonic function using $L_-$ to access the values of $m\neq l$.
        Recall that the operator \( L_{\pm} \) is defined as  
        \[ L_{\pm} = \pm \hbar e^{i\phi} \left( \frac{\partial}{\partial \theta} \pm i \cot\theta \frac{\partial}{\partial \phi} \right). \]  

        Under the transformations \( \theta \to \pi - \theta \) and \( \phi \to \phi + \pi \), the differentials remain unchanged except that the differential with respect to \( \theta \) picks up a negative sign.  

        To verify the behavior of \( \cot\theta \) under these transformations, we only need to check whether  
        \[ \cot\theta \stackrel{?}{=} \cot(\pi - \theta), \]  
        since it was already shown that \( e^{i\phi} \to e^{i(\phi-\pi)} = (-1)^l e^{i\phi} \).  

        Using trigonometric identities, we compute:  
        \[
            \cot(\pi - \theta) = \frac{\cos(\pi - \theta)}{\sin(\pi - \theta)} = \frac{-\cos\theta}{\sin\theta} = -\cot\theta.  
        \]  

        Thus, \( \cot(\pi - \theta) = -\cot\theta \), confirming the behavior of $L_-$ under the transformation.

        $$L_- = \Pi L_- \implies [L_-, \Pi] = 0$$
        Therefore 
        $$\Pi Y_l^m = (-1)^lY_l^m$$
    \end{callout}
\end{homeworkProblem}

\begin{homeworkProblem}
    Exercise 12.6.1 (Shankar) -- 9 points

    A particle is described by the wave function
    \[\psi_E(r, \theta, \phi) = Ae^{-r/a_0} \quad (a_0 = \text{const})\]

    (1) What is the angular momentum content of the state?
    \begin{callout}{Solution:}
        No $\phi$ or $\theta$ dependence implies it has no angular momentum. Furthermore, $\psi_E \propto Y_0^0$.
    \end{callout}

    (2) Assuming $\psi_E$ is an eigenstate in a potential that vanishes as $r \rightarrow \infty$, find $E$. (Match leading terms in Schrödinger's equation.)
    \begin{callout}{Solution:}
        \begin{align*}
            \hat{H}\psi_E = \left[-\frac{\hbar^2}{2\mu} \frac{1}{r^2} \frac{\partial}{\partial r} \left( r^2 \frac{\partial}{\partial r} \right) + V(r)\right]\psi_E &= E\psi_E \\ 
            \left[-\frac{\hbar^2}{2\mu} \frac{1}{r^2} \frac{\partial}{\partial r} \left( r^2 \frac{\partial}{\partial r} \right) + V(r)\right]Ae^{-r/a_0} &= EAe^{-r/a_0} \\
            \left[-\frac{\hbar^2}{2\mu} \frac{1}{r^2} \frac{\partial}{\partial r} \left( r^2 \frac{\partial}{\partial r} \left( Ae^{-r/a_0} \right) \right) + V(r)Ae^{-r/a_0}\right] &= EAe^{-r/a_0} \\
            -\frac{\hbar^2}{2\mu} \frac{1}{r^2} \frac{A}{a_0^2}\left(2a_0r-r^2\right)e^{-r/a_0} + V(r)Ae^{-r/a_0} &= EAe^{-r/a_0} \\
            -\frac{\hbar^2}{2\mu} \frac{1}{r^2} \frac{1}{a_0^2}\left(2a_0r-r^2\right) + V(r) &= E \\
            - \frac{\hbar^2}{\mu a_0} + \frac{\hbar^2}{2\mu a_0 r} + V(r) &= E
        \end{align*}
        If potential vanishes as $r\to\infty$, then we are left with one term 
        $$E=-\frac{\hbar^2}{\mu a_0}$$
    \end{callout}

    (3) Having found $E$, consider finite $r$ and find $V(r)$.
    \begin{callout}{Solution:}
        Substituting $E$ in, we have 
        $$-\frac{\hbar^2}{2\mu a_0 r}=V(r)$$
    \end{callout}
\end{homeworkProblem}
