\section{\label{sec:level1}DISCUSSION}\label{sec:discussion}

\begin{figure}[h]
  \centering
  \includegraphics[width=0.45\textwidth]{../../code/data.pdf}
  \caption{Fourier analysis where the derivative of our spectrum used to compute peak centers. Three clusters of peaks with diminishing intensity can be seen.}
  \label{fig:data}
\end{figure}


\begin{figure}[h]
  \centering
  \includegraphics[width=0.45\textwidth]{../../code/folded_scheme.png}
  \caption{Band structure of our acoustic analogue system plotted in the reduced zone scheme. The data points represent measured resonance frequencies ($\omega$) versus wave vectors ($k$) constructed from our experimental setup of eight 75mm tubes separated by seven 16mm apertures. Three distinct energy bands are visible, with band gaps occurring at $k = n \pi/a$ ($n$ = 1, 2), demonstrating the formation of forbidden frequency regions analogous to electronic band gaps in crystalline solids.}
  \label{fig:folded_scheme}
\end{figure}

In this paper, we have demonstrated an acoustic analogue for the quantum mechanical tight binding model, providing a tangible and accessible way to visualize band structures that are fundamental to condensed matter physics. By creating a periodic arrangement of cylindrical cavities connected by apertures, we successfully modeled the essential physics of electron waves in crystalline solids using sound waves.

Our key limitations in this experiment are equipment quality. Use of digital equipment and a computer driven control ought to allow students to accurately observe higher order states and construct a larger band structure. Future improvements should focus on implementing digital data acquisition using computer-controlled frequency sweeping and digital signal processing. This would allow for:

\begin{enumerate}
    \item Higher frequency resolution to better resolve closely spaced peaks
    \item Extended range measurements to observe higher-order bands
    \item Automated peak identification to reduce measurement uncertainty
    \item Real-time visualization of the developing band structure during measurements
\end{enumerate}
