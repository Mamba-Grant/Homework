\begin{abstract}

    %The band structure is a fundamental concept in the field of condensed matter physics, and is an essential consideration in the engineering of materials in all modern electronics. Despite this, it remains very hard to develop a strong, experimental intuition for the concept directly using wave dynamics (excluding extremely expensive pump-probe microscopy). Student-accessible experiments such as those based on mechanical pendula are accessible and well-shared but entail a layer of abstraction \cite{neder2024bloch}. For this reason we propose an acoustic analogue, requiring only basic experimental equipment to model a 1-D lattice and the band structure. 

Band structure theory forms the cornerstone of condensed matter physics and underpins the design and functionality of all modern electronic devices. However, developing robust experimental intuition for this quantum mechanical phenomenon remains challenging for students, as direct visualization typically requires sophisticated techniques like angle-resolved photoemission spectroscopy that are financially prohibitive for educational settings. While existing pedagogical approaches using coupled mechanical oscillators provide some insight, they introduce unnecessary layers of abstraction between the model and the quantum reality they represent \cite{neder2024bloch}. We address this pedagogical gap by demonstrating an acoustic wave analogue that directly models electron behavior in periodic potentials. Our approach requires only basic laboratory equipment to create a one-dimensional acoustic lattice that exhibits band formation, allowing students to directly observe, measure, and hear the quantum mechanical principles underlying band structure formation. This accessible experimental platform transforms an abstract quantum concept into a tangible, macroscopic phenomenon that students can explore firsthand, significantly enhancing conceptual understanding of this fundamental solid-state physics principle.

    %Furthermore we demonstrate the effects of broken symmetry in this chain to elucidate on the importance of doping for device design.

\end{abstract}
