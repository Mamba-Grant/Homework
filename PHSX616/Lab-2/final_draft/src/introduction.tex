\section{\label{sec:level1}INTRODUCTION}

    It has been widely observed that there exists approximately white noise as electronic fluctuations within resistive materials. Amplifier technicians in the early 20th century first noticed that a small fraction of noise persisted independently of the then-called "tube noise." Johnson first experimentally measured this quantity in 1926, where it was found that it is proportional to both resistance and frequency. This experiment also provides a delightful way to measure Boltzmann's constant, despite Johnson's original work scraping by with a 13\% error on this for the time \cite{1928PhRv...32...97J}. It is predicted that Johnson noise is proportional to total frequency bandwidth and resistance, given by:
    $$
    V_{\text{RMS}} = \sqrt{4 k_B T \Delta F}
    $$

    
    Johnson noise is typically in the regime of $1\times10^{-6} \mathrm{~V}$. To measure noise on such small scales, I repeat the method of Johnson's original work, using TeachSpin's "Noise Fundamentals" equipment \cite{TeachSpinNoiseFundamentals}. In the same light as the original work, a pre and post amplifier, coupled with a squarer on the voltage output, shown in figure \ref{fig:schematic}. Bandwidth and resistance dependence are reviewed, and results used to compute Boltzmann's constant.


\begin{figure}[h]
  \centering
  \includegraphics[width=0.4\textwidth]{../assets/Schematic.png}
  \caption{Schematic of the instrumentation. Sample resistance can be controlled in the preamplifier stage, where a fixed gain of 600x is applied. This is fed into two filters, high and low pass effectively forming a band-pass. This is fed through a primary gain, which is configured to minimize clipping in measurement. Finally, this is passed through an analog squarer and output filter, which takes a time-average.}
  \label{fig:schematic}
\end{figure}

    The nature of Johnson and amplifier noise is approximately white; in that the time average for all but terahertz frequencies is zero \cite{1928PhRv...32...97J}. Consequently, we employ an analog voltage squarer, which has advantages in 
    The output of the amplification stage introduces amplifier noise several orders larger magnitude than Johnson noise of the system, however it is possible to separate the two as a consequence of the squarer, where in a time averaged regime, uncorrelated voltage sources (noise) with gain $G$ can be superimposed as in equation \ref{eq:gain-separation}. This establishes the RMS amplifier noise, which is then subtracted off.

    \begin{equation}\label{eq:gain-separation}
        \langle V^2_{\text{out}} \rangle 
        = G^2 \left[ \langle{V_J}\rangle + \langle{V_N}\rangle \right]^2
        = G^2 \left[ \langle{V_J^2}\rangle + \langle{V_N^2}\rangle \right]
    \end{equation}
