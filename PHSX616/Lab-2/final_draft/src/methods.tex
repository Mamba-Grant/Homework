\section{\label{sec:level1}RESISTANCE AND FREQUENCY DEPENDENCE}

\subsection{Experimental Setup}

In the resistance dependency measurements, we take measurements using octave scale resistors provided by TeachSpin manufacturers, from 1 $\Omega$ to 100 $k \Omega$ (tolerance 0.1\%). This is supplemented with metal film resistors (tolerance 0.1\%) which can be directly measured using a Fluke 179 Multimeter. We label these BEXT and CEXT, and have corresponding resistances which can be found in table \ref{tab:resistor-values}.
%make this table an appendix reference instead maybe
Our configuration uses a preamplifier, filter, secondary amplifier, squarer, and time-averager. The first stage in the pre-amp has a set gain of $g=1+R_f / R_1$, where $R_f=1~k \Omega$ and $R_1 = 200~\Omega$. This is fed through a hard-wired amplifier with gain 100, set by a similar ratio of resistor values. Our filters are a separately configurable high and low pass filter, approximated well by the Butterworth response, with damping parameter $\gamma = \frac{1}{2Q}$. These considerations are significant in the determination of frequency dependence in Johnson noise. For more information about the equivalent bandwidths, see table \ref{tab:bandwidths-table}
This is then fed through the secondary amplifier, which we configure to eliminate signal clipping while providing a sufficient boost in signal, and follow the same uncertainty as with the preamplifier. Output is passed to a squarer, which is subject to zero-offsets; however, the static squaring accuracy is claimed to be about 0.2\% with a bandwidth that extends to 3 MHz. The time-average of this is fed through to a Fluke 179 multimeter. We opt for a time constant of $\tau = 0.1$ s.

\begin{table}[h]
	\centering
	\renewcommand{\arraystretch}{1.2} % Adjust row spacing
	\begin{tabular}{p{2cm} p{2cm}}
    	\toprule
    	\textbf{Resistor} & \textbf{Resistance} \\
    	\midrule
    	1 & $1.0$ $\Omega$ \\
    	2 & $10.0$ $\Omega$ \\
    	3 & $100.0$ $\Omega$ \\
    	4 & $1000.0$ $\Omega$ \\
    	5 & $10000.0$ $\Omega$ \\
    	6 & $100000.0$ $\Omega$ \\
    	BEXT & $279.9$ $\Omega$ \\
    	CEXT & $2996.0$ $\Omega$ \\
    	\bottomrule
	\end{tabular}
	\caption{Resistor Values.}
	\label{tab:resistor-values}
\end{table}

\subsection{Separation of Noise \& Resistance-Correlation of Johnson Noise}

Johnson predicts his noise is proportional to resistance, and to confirm this we systematically fix the bandwidth while varying resistance. We selected 100 Hz on the high pass filter, and 10 kHz on the low pass filter for an effective bandwidth of 10,996 Hz (see table \ref{tab:bandwidths-table} for values) by modeling our filters using the Butterworth response model. We then vary the resistor by switching between 1 $\Omega$ to 100 $k \Omega$ in octave steps.

The noise induced by our amplifier is many times larger than the Johnson noise, and must therefore be separated from the signal. This can be done because the mean-square voltages from uncorrelated sources are additive. We accomplish this by extrapolating to the zero resistance regime by fitting a linear curve, wherein only amplifier noise is measured, shown in figure \ref{fig:amp-noise}.

We collected two datasets at different gain, and determined that there is a substantial error at higher gain. We compensate for this by using data at 6000 gain only for resistances below 0.1 $k \Omega$. Gain of 1500 was used in all other ranges, which sacrifices precision at low resistances, but does not suffer from clipping at high resistance.

\begin{figure}[H]
	\centering
	\includegraphics[width=0.4\textwidth]{../../code/amp-noise.pdf}
	\caption{Output collected from the time-averager, shown in blue, is fit to a linear curve to extrapolate in the $R\to0$ regime such that amplifier noise can be extracted. The resulting data, $\langle V_J^2 \rangle$, is plotted in green, and the theoretical prediction made by Johnson's model overlaid in yellow.}
	\label{fig:amp-noise}
\end{figure}

Figure \ref{fig:amp-noise} shows that resistance-dependence in Johnson noise is fit exceptionally well by Johnson's model. We find a chi-squared value per degree of freedom of 0.11, which is to say that the observed scatter is less than that predicted by the analytical uncertainties.

\subsection{Bandwidth-Correlation of Johnson Noise}
Johnson's model also predicts a proportionality to bandwidth in the magnitude of noise. Following the separation of amplifier noise from Johnson noise, we can directly subtract off the amplifier noise from the value measured out of the time-averager (time constant of 0.1 s). In doing so, we can now vary the bandwidth while letting resistance remain at a fixed 10 $k\Omega$. We systematically adjust the high and low pass filter to the values seen in table \ref{tab:bandwidths-table}.

In our analysis of the data, we found it incredibly important to model the frequency rolloff of the gain function. Amplifier gain is not constant across frequencies, and it typically follows a transfer function, such as:
$$G(f) = G_{0} \cdot H(f)$$
In our case, the transfer function is modelled well as
$$\label{eq:transfer-function}
H(f) = \frac{1}{\sqrt{1 + \left( \frac{f}{f_c} \right)^{4}}}$$

Previously, our gain was said to be the product of preamp and primary gain. This was an acceptable approximation in the previous sections, because the difference in theoretical cutoff frequency was very close to the actual cutoff frequency, making this a very small correction term; however, this is not necessarily the case in our assessment of frequency-correlated measurements, particularly in the larger bandwidths.

Because we chose to separate amplifier noise at a particular bandwidth in the resistance-correlated measurements, we cannot reuse the amplifier noise measurement as it depends on parameters in the high and low pass filters. Instead, we fit our frequency-correlated dataset to the zero-bandwidth regime to equivalently separate amplifier noise in the same fashion.

\begin{table}[h]
	\centering
	\begin{tabular}{c|c|c|c|c|c|c|c}
    	\toprule
    	& 0.33 kHz & 1 kHz & 3.3 kHz & 10 kHz & 33 kHz & 100 kHz \\
    	\midrule
    	10 Hz & 355.0 & 1100 & 3654 & 11096 & 36643 & 111061 \\
    	30 Hz & 333.0 & 1077 & 3632 & 11074 & 36620 & 111039 \\
    	100 Hz & 258.0 & 1000 & 3554 & 10996 & 36543 & 110961 \\
    	300 Hz & 105.0 & 784 & 3332 & 10774 & 36321 & 110739 \\
    	1000 Hz & 9.0 & 278 & 2576 & 9997 & 35543 & 109961 \\
    	3000 Hz & 0.4 & 28 & 1051 & 7839 & 33324 & 107740 \\
    	\bottomrule
	\end{tabular}
	\label{tab:bandwidths-table}
	\caption{Effective bandwidths from the high pass (leftmost column) and low pass (upper row) parameters.}
\end{table}

\subsection{Noise Density}

In the previous sections, we found a linear correlation in frequency and resistance for Johnson noise. Using these relations, we define a noise density in frequency, $\langle V^2_J \rangle / \Delta f$. This allows us to parameterize noise density with a single value, $R$. Doing so provides a complete summary of our results, shown in figure \ref{fig:noise-density}.

\begin{figure}[h]
	\centering
	\includegraphics[width=0.4\textwidth]{../../code/noise_density.pdf}
	\caption{It's predicted that Johnson noise is linearly correlated with both noise and frequency, so by constructing a "density" in the squared voltage per bandwidth, it is possible to directly fit a value for Boltzmann's constant. In this case, we find a value of $1.33 \times 10^{-23} \pm 1.34\times10^{24}$ J/K. In Johnson's original work, $k_B$ was found to be significantly lower at $k_B=1.27\times10^{-23}\pm13\%$ J/K. Since 2020, the accepted value for $k_B$ has been $1.380 \times 10^{-23}$ J/K, placing our mean under 4\% of the accepted value.}
	\label{fig:noise-density}
\end{figure}



