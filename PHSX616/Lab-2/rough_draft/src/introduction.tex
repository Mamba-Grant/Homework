\section{\label{sec:level1}INTRODUCTION}

    It has been widely observed that there exists approximately white noise as electronic fluctuations within resistive materials. Amplifier technicians in the early 20th century first noticed that a small fraction of noise persisted independently of the then-called "tube noise." Johnson first experimentally measured this quantity in 1926, where it was found that it is proportional to both resistance and frequency. This experiment also provides a delightful way to measure Boltzmann's constant, despite Johnson's original work scraping by with a 13\% error on this for the time \cite{1928PhRv...32...97J}. It is predicted that Johnson noise is proportional to total frequency bandwidth and resistance, given by:
    $$
    V_{\text{RMS}} = \sqrt{4 k_B T \Delta F}
    $$

    
    Johnson noise is typically in the regime of $1\times10^{-6} \mathrm{~V}$. To measure noise on such small scales, I repeat the method of Johnson's original work, using TeachSpin's "Noise Fundamentals" equipment \cite{TeachSpinNoiseFundamentals}. In the same light as the original work, a pre and post amplifier, coupled with a squarer on the voltage output, shown in figure \ref{fig:schematic}. Bandwidth and resistance dependence are reviewed, and results used to compute Boltzmann's constant.

    The nature of Johnson and amplifier noise is approximately white; in that the time average for all but terahertz frequencies is zero \cite{1928PhRv...32...97J}. Consequently, we employ an analog voltage squarer, which has advantages in 
    The output of the amplification stage introduces amplifier noise several orders larger magnitude than Johnson noise of the system, however it is possible to separate the two as a consequence of the squarer, where in a time averaged regime, uncorrelated voltage sources (noise) with gain $G$ can be superimposed as in equation \ref{eq:gain-separation}. This establishes the RMS amplifier noise, which is then subtracted off.

    \begin{equation}\label{eq:gain-separation}
        \langle V^2_{\text{out}} \rangle 
        = G^2 \left[ \langle{V_J}\rangle + \langle{V_N}\rangle \right]^2
        = G^2 \left[ \langle{V_J^2}\rangle + \langle{V_N^2}\rangle \right]
    \end{equation}


%\subsubsection{Experimental Measurement of Charge}
%
%At the time of the original experiment by Robert Millikan, the existence of subatomic particles had not been universally accepted, however the work of many scientists of the time led to many postulating the existence of the electron\cite{millikan1917electron, enwiki:1271526381}. Millikan's experiment was the first to confirm it, operating on the very simple principle of the motion of charged particles in an electromagnetic field; motion which is easily modelled as a combination free fall, buoyant, and coulomb interactions \cite{MillikanOilDropManual}. These forces take the form, \begin{equation}\label{charge_eq1}
%    qE = mg + kv
%\end{equation}
%
%By directly observing the motion of particles during free fall and in an electric field independently, we hope to directly determine the charge on the particle. Due to the interaction of buoyant forces on our particle, it may be extremely difficult to track their motion without sophisticated computer hardware, due to the nonlinear nature free fall velocity. For this reason, we opt to observe microscopic particles using with very low mass such that they can be assumed to always be at terminal velocity. A severe consequence of this decision is that the forces on such particles will be incredibly small–on the order of $x$ Newtons. To accomplish this, we use non-volatile oil, introduced into the viewing chamber by an atomizer. With careful consideration of external conditions, it is possible to ascertain the charge on the oil droplets by their free fall velocity and coulomb interaction velocity alone. From (\ref{charge_eq1}) we can express 
%\begin{equation}\label{charge_eq2}
%    q = \frac{mg(v_f - v_i)}{Ev_f}
%\end{equation}
%
%To eliminate $m$ from equation (\ref{charge_eq2}), one uses the expression for the volume of a sphere:
%\begin{equation}\label{charge_eq3}
%    m = \frac{4}{3}\pi a^3 \rho
%\end{equation}
%
%Where $a$ is the radius of the droplet and $\rho$ is the density of oil. We assume that the oil droplets assume perfectly spherical shape due to extremely small terminal velocity. Then Stoke's law is used to relate the radius of the oil droplet to its velocity as it falls through a viscous medium.
%\begin{equation}\label{charge_eq4}
%    a = \sqrt{ \frac{9 \eta v_f}{2g\rho} }
%\end{equation}
%
%Stokes’ Law, however, becomes incorrect when the velocity of fall of the droplets is less than 0.1 cm/s. Droplets having this and smaller velocities have radii, on the order of 2 microns, comparable to the mean free path of air molecules, a condition which violates one of the assumptions made in deriving Stokes’ Law \cite{stackexchange_2017}. Since the velocities of the droplets used in this experiment will be in the range of 0.01 to 0.001 cm/s, the viscosity must be multiplied by a correction factor. The resulting effective viscosity is
%\begin{equation}\label{charge_eq4}
%    \eta_{\textrm{eff}} = \eta \left( \frac{1}{1+\frac{b}{p a}} \right)
%\end{equation}
%
%Where $b$ is a constant, $p$ is the atmospheric pressure, and $a$ is the radius of the drop as calculated by the uncorrected form of Stokes’ Law, equation (\ref{charge_eq4}). Then we arrive at a suitable radius for our oil droplets.
%\begin{equation}\label{charge_eq6}
%    a = \sqrt{ \left( \frac{b}{2p} \right)^{2} + \frac{9\eta v_f}{2g\rho} } - \left( \frac{b}{2p} \right)
%\end{equation}
%
%Which in turn yields the following equation for charge 
%\begin{equation}\label{charge_eq7}
%    q = \frac{6\pi}{E} \sqrt{ \frac{9\eta^{3}}{2g\rho \left( 1 + \frac{b}{pa} \right)^{3} } }(v_f + v_r) \sqrt{v_f}
%\end{equation}
%
%Due to the use of a simple air-plate capacitor to generate an electric field, electric field is straightforwardly
%\begin{equation}\label{charge_eq8}
%    E = \frac{V}{d}
%\end{equation}
%
%Where $V$ is the potential difference across the parallel plates separated by the distance $d$
%
%\subsubsection{Considerations in Instrument Configuration}
%
%There are a variety of ways to manipulate charged particles using external electromagnetic fields \cite{GriffithsEM}.
%
%We employ a uniform electric field via a simple plate capacitor due to two key advantages over magnetic fields. The electromagnetic fields created must be uniform such that it is possible to reliably measure the forces acting on particles–for this it is possible to use several devices, however magnetic fields are conventionally produced using windings of wire, dramatically increasing the complexity of optical observation of the particles inside the field. Second, the motion of charged particles in a uniform magnetic field is known to be circular, increasing the complexity in the determination of force. For these reasons, we employ a simple plate capacitor to produce a uniform electric field, as it permits the construction of a straightforward viewing chamber, shown in figure \ref{fig:apparatus-diagram} . A capacitor also has the advantage of producing uniform fields which drive particles in a perfectly vertical manner, simplifying the determination of their motion. 
%
%\begin{figure}[h]
%  \centering
%  \includegraphics[width=0.35\textwidth]{../assets/tmp_apparatus_diagram.png}
%  \caption{A schematic of the experimental setup. The switch shown in the top right is used to switch the plates between charged and grounded mode. Droplets are monitored using the scope, and introduced between the capacitors using the port above. The sample of Thorium-232 located in the droplet chamber can be rotated to ionize the oil droplets by use of the source lever, pictured on the left.}
%  \label{fig:apparatus-diagram}
%\end{figure}
