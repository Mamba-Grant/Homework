\begin{abstract}


    \textit{The abstract is still a work in progress.} It had been known in Johnson's time that the characteristic noise of linear electronics, attributed to statistical mechanical effects, was a function of bandwidth and resistance, and was white in nature. We attempt to replicate Johnson's original work using an array of modern amplifiers and band pass filters to assess the resistance and frequency correlation in this noise, which has come to be known as Johnson noise. We conclude with high certainty that resistance and bandwidth dependences are linearly correlated, and use these results to provide a strong estimate of Boltzmann's constant.
    %It has been known since Johnson-Nyquist noise is said to 

%We present an apparatus designed for the observation of charged oil droplets in an electric field for the purposes of determining the atomic nature of charge by measuring the force experienced by the particle in an electric field of known strength. While it is easy to produce an electric field of known strength, there is great difficulty in the observation of particles with little charge. By careful observation of charged oil droplets, we find that charge is quantized in nature, directly observing integer multiples of charge on our oil droplets

%An article usually includes an abstract, a concise summary of the work
%covered at length in the main body of the article. 
%\begin{description}
%\item[Usage]
%Secondary publications and information retrieval purposes.
%\item[Structure]
%You may use the \texttt{description} environment to structure your abstract;
%use the optional argument of the \verb+\item+ command to give the category of each item. 
%\end{description}

\end{abstract}
