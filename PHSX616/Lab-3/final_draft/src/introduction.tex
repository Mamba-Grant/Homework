\section{\label{sec:level1}INTRODUCTION}
    In early 1913, proposals were put forth that it may be possible to observe diffraction due to the light wave, spin-orbit interaction in crystalline lattices in an effect which is known well known as Bragg Diffraction. In the century since, application of such physics has been applied to pioneer crystallography and metrology, and in this work we aim characterize the x-radiation of molybdenum by imaging the bragg reflection upon a calibration substrate.

    The emission spectra of molybdenum observed in this work arises from the fine structure of the $L$ shell, and subsequently the spin-orbit interaction of the electrons with x-radiation \cite{LD_Didactic_Manual}. Furthermore, there exist three sub-shells, $L_{\text{I}}$, $L_{\text{II}}$, $L_{\text{III}}$, which are subject to emission rules:
    $$\Delta I = \pm 1, \qquad \Delta j = 0, \pm 1$$
    Where $I$ is the orbital angular momentum, and $j$ the total angular momentum, therefore, two transitions from the $L$-shell to the $K$-shell are permitted. 

    We aim to directly measure these by first directing electrons from a source to the molybdenum anode, bombarding it with electrons and ionizing electrons in the inner $K$ shell. This prompts the formation of a hole which is immediately relaxed by the electrons in the upper $L$ orbital, emitting x-radiation. Bragg's diffraction relation models these constructive interference produced by emitted x-radiation via equation \ref{eq:bragg}. For information on the derivation, see \cite{Bragg1929-BRATDO-17}.

    \begin{equation}\label{eq:bragg}
        n\lambda = 2d \sin \theta \tag{I} 
    \end{equation}

    Our machine operates using Bragg-Brentano geometry, such that the substrate is flat relative to the incident x-rays which are fired into a detector (see figure \ref{fig:apparatus}). A photon sensitive detector is equipped to orbit the sample in increments of 0.1$^\circ$. The detector sees a powder diffraction pattern, appearing as a series of rings, where brightness peaks will occur at characteristic $K_\alpha, K_{\beta}, K_{\gamma}$ lines \cite{taddei2015xrd}.

    %$$
    %V_{\text{RMS}} = \sqrt{4 k_B T \Delta F}
    %$$
    %
    %
    %Johnson noise is typically in the regime of $1\times10^{-6} \mathrm{~V}$. To measure noise on such small scales, I repeat the method of Johnson's original work, using TeachSpin's "Noise Fundamentals" equipment \cite{TeachSpinNoiseFundamentals}. In the same light as the original work, a pre and post amplifier, coupled with a squarer on the voltage output, shown in figure \ref{fig:schematic}. Bandwidth and resistance dependence are reviewed, and results used to compute Boltzmann's constant.
    %
    %The nature of Johnson and amplifier noise is approximately white; in that the time average for all but terahertz frequencies is zero \cite{1928PhRv...32...97J}. Consequently, we employ an analog voltage squarer, which has advantages in 
    %The output of the amplification stage introduces amplifier noise several orders larger magnitude than Johnson noise of the system, however it is possible to separate the two as a consequence of the squarer, where in a time averaged regime, uncorrelated voltage sources (noise) with gain $G$ can be superimposed as in equation \ref{eq:gain-separation}. This establishes the RMS amplifier noise, which is then subtracted off.
    %
    %\begin{equation}\label{eq:gain-separation}
    %    \langle V^2_{\text{out}} \rangle 
    %    = G^2 \left[ \langle{V_J}\rangle + \langle{V_N}\rangle \right]^2
    %    = G^2 \left[ \langle{V_J^2}\rangle + \langle{V_N^2}\rangle \right]
    %\end{equation}
