\section{\label{sec:level1}DETERMINATION OF CHARGE ON OIL DROPLETS}

\subsection{Observations}

With the capacitor connected to the DC power supply with 500V, we manipulated the electric field in the viewing chamber by selectively switching between ground and charged modes. With the plates grounded, we introduce oil droplets between the plates through a port above. Then, we isolate droplets which we perceive to be of low charge and record their rise and fall velocity with plates charged and grounded, respectively. By carefully controlling the conditions within the viewing chamber, we then record the time using a stopwatch for the oil droplets to move between the 0.5 mm markings in the viewing chamber.

To eliminate uncertainty about the effects of the oil droplet on the motion in the electric field, we manipulate the charge on our oil droplets by means of ionization. An alpha source, Thorium-232, is placed near the drop and can be toggled using a lever on the side of the apparatus (\ref{fig:apparatus_diagram}). Using this, we were able to modify the charge on an oil droplet to assess the effects of changes in charge on the oil droplets. 


\begin{table}[h]
    \centering
    \begin{tabular}{c|c|c|c|c}
        \hline
        Droplet & Charge (Coulomb) & \# Charges on Oil Drop \\
        \hline
        A & $1.81276 \times 10^{-19}$ & 1.0 \\
        B & $10.9941 \times 10^{-19}$ & 6.06482 \\
        C & $12.8595 \times 10^{-19}$ & 7.09385 \\
        \hline
    \end{tabular}
    \caption{Measured charges on oil drops}
    \label{tab:oil_drops}
\end{table}

The measured charges on the oil droplets exhibit distinct, quantized values, supporting the hypothesis that electric charge exists in discrete units rather than as a continuous variable. By analyzing the charge values obtained from multiple oil droplets, we observe that each charge closely corresponds to integer multiples of a fundamental unit of charge.

To determine this fundamental charge, we examine the measured charges in Table \ref{tab:oil_drops}. The smallest observed charge is approximately $1.81 \times 10^{-19}$ C, which we identify as the unit charge. The charges of other droplets, such as Droplet B ($10.99 \times 10^{-19}$ C) and Droplet C ($12.86 \times 10^{-19}$ C), align well with integer multiples of this fundamental charge.

\subsection{Uncertainty Analysis}

Comparing our experimental result with the accepted value of the elementary charge \cite{millikan1917electron}, $1.602 \times 10^{-19}$ C, we calculate the percentage error:

\[
\text{Percentage error} = \left( \frac{|1.81 - 1.602|}{1.602} \right) \times 100 \approx 13.0\%
\]

We notice a fixed error in our results such that all measured values are multiples of each other with the same 13\% deviation. This implies the deviation may result from quantities we lacked sufficient means to appropriately measure: namely local atmospheric conditions. Furthermore, rudimentary means were used to time the motion of oil droplets, and we see a significant standard deviation in this degree of freedom.

\textit{todo: proper uncertainty calculations}
