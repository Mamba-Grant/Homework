\appendix

\section{Charge Calculation Error Propagation}

To start the appendixes, use the \verb+\appendix+ command.
This signals that all following section commands refer to appendixes
instead of regular sections. Therefore, the \verb+\appendix+ command
should be used only once---to setup the section commands to act as
appendixes. Thereafter normal section commands are used. The heading
for a section can be left empty. For example,
\begin{verbatim}
\appendix
\section{}
\end{verbatim}
will produce an appendix heading that says ``APPENDIX A'' and
\begin{verbatim}
\appendix
\section{Background}
\end{verbatim}
will produce an appendix heading that says ``APPENDIX A: BACKGROUND''
(note that the colon is set automatically).

If there is only one appendix, then the letter ``A'' should not
appear. This is suppressed by using the star version of the appendix
command (\verb+\appendix*+ in the place of \verb+\appendix+).

%\section{A little more on appendixes}
%
%Observe that this appendix was started by using
%\begin{verbatim}
%\section{A little more on appendixes}
%\end{verbatim}
%
%Note the equation number in an appendix:
%\begin{equation}
%E=mc^2.
%\end{equation}
%
%\subsection{\label{app:subsec}A subsection in an appendix}
%
%You can use a subsection or subsubsection in an appendix. Note the
%numbering: we are now in Appendix~\ref{app:subsec}.
%
%Note the equation numbers in this appendix, produced with the
%subequations environment:
%\begin{subequations}
%\begin{eqnarray}
%E&=&mc, \label{appa}
%\\
%E&=&mc^2, \label{appb}
%\\
%E&\agt& mc^3. \label{appc}
%\end{eqnarray}
%\end{subequations}
%They turn out to be Eqs.~(\ref{appa}), (\ref{appb}), and (\ref{appc}).
