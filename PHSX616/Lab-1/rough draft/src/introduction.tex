\section{\label{sec:level1}INTRODUCTION}

The motion of any charged particle in atmosphere, within an electromagnetic field are easily modelled as a 

%This sample document demonstrates proper use of REV\TeX~4.2 (and
%\LaTeXe) in mansucripts prepared for submission to APS
%journals. Further information can be found in the REV\TeX~4.2
%documentation included in the distribution or available at
%\url{http://journals.aps.org/revtex/}.
%
%When commands are referred to in this example file, they are always
%shown with their required arguments, using normal \TeX{} format. In
%this format, \verb+#1+, \verb+#2+, etc. stand for required
%author-supplied arguments to commands. For example, in
%\verb+\section{#1}+ the \verb+#1+ stands for the title text of the
%author's section heading, and in \verb+\title{#1}+ the \verb+#1+
%stands for the title text of the paper.
%
%Line breaks in section headings at all levels can be introduced using
%\textbackslash\textbackslash. A blank input line tells \TeX\ that the
%paragraph has ended. Note that top-level section headings are
%automatically uppercased. If a specific letter or word should appear in
%lowercase instead, you must escape it using \verb+\lowercase{#1}+ as
%in the word ``via'' above.
%
%\subsection{\label{sec:level2}Second-level heading: Formatting}
%
%This file may be formatted in either the \texttt{preprint} or
%\texttt{reprint} style. \texttt{reprint} format mimics final journal output. 
%Either format may be used for submission purposes. \texttt{letter} sized paper should
%be used when submitting to APS journals.
%
%\subsubsection{Wide text (A level-3 head)}
%The \texttt{widetext} environment will make the text the width of the
%full page, as on page~\pageref{eq:wideeq}. (Note the use the
%\verb+\pageref{#1}+ command to refer to the page number.) 
%\paragraph{Note (Fourth-level head is run in)}
%The width-changing commands only take effect in two-column formatting. 
%There is no effect if text is in a single column.
%
%\subsection{\label{sec:citeref}Citations and References}
%A citation in text uses the command \verb+\cite{#1}+ or
%\verb+\onlinecite{#1}+ and refers to an entry in the bibliography. 
%An entry in the bibliography is a reference to another document.
