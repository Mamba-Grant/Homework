\section{\label{sec:level1}DETERMINATION OF CHARGE ON OIL DROPLETS AND RESULTS}

\subsection{Experimental Setup}

We select Squibb \#5597 Mineral Oil which has density 886 $\mathrm{kg}/\mathrm{m}^3$, provided by the manufacturer. Additionally, we use weather data in the determination of air pressure, 101 kPa, which has caveats discussed in \ref{sec:uncertainty}. Furthermore, we used an antique power supply of unknown specification provided by the physics department. A Fluke 179 Voltmeter was used to accurately measure voltage supplied to the capacitors.

With the capacitor connected to the DC power supply with 500V, we manipulated the electric field in the viewing chamber by selectively switching between ground and charged modes. With the plates grounded, we introduce oil droplets between the plates through a port above. Then, we isolate droplets which we perceive to be of low charge and record their rise and fall velocity with plates charged and grounded, respectively. By carefully controlling the conditions within the viewing chamber, we then record the time using a stopwatch for the oil droplets to move between the 0.5 mm markings in the viewing chamber.

To eliminate uncertainty about the effects of the oil droplet on the motion in the electric field, we manipulate the charge on our oil droplets by means of ionization. An alpha source, Thorium-232, is placed near the drop and can be toggled using a lever on the side of the apparatus (figure \ref{fig:apparatus-diagram}). Using this, we were able to modify the charge on an oil droplet to assess the effects of changes in charge on the oil droplets. This procedure was done for droplets B and C in table \ref{tab:oil-drops-hq}. This table also serves to demonstrate the quantized nature of charge. I let charge A be the unit charge, since it is the smallest which appears in the dataset. Then, I normalize the charge on each droplet by this charge, where it can be observed that the number of charges on each droplet are round multiples of this unit charge. 

Given the time constraint of this research, it is necessary to gather both high-precision and low-precision data. The high-precision population serves the purpose to test our theory that charge is quantized in nature, and to provide an accurate measure of electron charge, while the low-precision population serves to provide sufficient statistical grounds for our work.

\begin{table}[h]
    \centering
    \renewcommand{\arraystretch}{1.0} % Adjust row height for better readability
    \begin{tabularx}{0.5\textwidth}{lXXXXX}
        \toprule
           & Charge (C) & \# Charges & Statistical Uncertainty & Systematic Uncertainty \\
        \midrule
        A & $1.695 \times 10^{-19}$ & $1.0 \pm 0.033$ & $3.162 \times 10^{-20}$ & $5.639 \times 10^{-21}$    \\
        B & $1.146 \times 10^{-18}$ & $6.76 \pm 0.21$ & $3.536 \times 10^{-20}$ & $3.628 \times 10^{-20}$    \\
        C & $1.167 \times 10^{-18}$ & $6.89 \pm 0.22$ & $3.536 \times 10^{-20}$ & $3.767 \times 10^{-20}$    \\
        D & $4.903 \times 10^{-19}$ & $2.893 \pm 0.093$ & $4.472 \times 10^{-20}$ & $1.576 \times 10^{-20}$  \\
        E & $8.033 \times 10^{-19}$ & $4.74 \pm 0.15$ & $3.333 \times 10^{-20}$ & $2.59 \times 10^{-20}$     \\
        F & $5.955 \times 10^{-19}$ & $3.51 \pm 0.11$ & $3.536 \times 10^{-20}$ & $1.895 \times 10^{-20}$    \\
        \bottomrule
    \end{tabularx}
    \caption{High-precision data, collected by taking many samples of the same droplet. The corresponding statistical uncertainty associated with these are listed in the rightmost column. All droplets are round multiples (within statistical deviation) of the smallest charge in the dataset, A.}
    \label{tab:oil-drops-hq}
\end{table}

\begin{table}[h]
    \centering
    \renewcommand{\arraystretch}{1.0} % Adjust row height for better readability
    \begin{tabularx}{0.5\textwidth}{lXXXXX}
        \toprule
           & Charge (C) & \# Charges & Statistical Uncertainty & Systematic Uncertainty \\
        \midrule
        G & $2.82 \times 10^{-19}$ & $1.664 \pm 0.054$ & $1.0 \times 10^{-19}$ & $9.216 \times 10^{-21}$ \\
        H & $2.422 \times 10^{-19}$ & $1.429 \pm 0.047$ & $1.0 \times 10^{-19}$ & $7.942 \times 10^{-21}$\\
        I & $1.24 \times 10^{-18}$ & $7.32 \pm 0.23$ & $1.0 \times 10^{-19}$ & $3.905 \times 10^{-20}$   \\
        J & $3.389 \times 10^{-19}$ & $2.0 \pm 0.065$ & $1.0 \times 10^{-19}$ & $1.108 \times 10^{-20}$  \\
        K & $6.326 \times 10^{-19}$ & $3.73 \pm 0.12$ & $1.0 \times 10^{-19}$ & $2.025 \times 10^{-20}$  \\
        L & $8.313 \times 10^{-19}$ & $4.9 \pm 0.15$ & $1.0 \times 10^{-19}$ & $2.609 \times 10^{-20}$   \\
        M & $7.709 \times 10^{-19}$ & $4.55 \pm 0.14$ & $1.0 \times 10^{-19}$ & $2.421 \times 10^{-20}$  \\
        N & $5.7 \times 10^{-19}$ & $3.36 \pm 0.11$ & $1.0 \times 10^{-19}$ & $1.821 \times 10^{-20}$    \\
        O & $1.277 \times 10^{-18}$ & $7.54 \pm 0.24$ & $1.0 \times 10^{-19}$ & $4.06 \times 10^{-20}$   \\
        P & $2.577 \times 10^{-18}$ & $15.2 \pm 0.48$ & $1.0 \times 10^{-19}$ & $8.154 \times 10^{-20}$  \\
        Q & $3.151 \times 10^{-19}$ & $1.859 \pm 0.06$ & $1.0 \times 10^{-19}$ & $1.011 \times 10^{-20}$ \\
        R & $8.048 \times 10^{-19}$ & $4.75 \pm 0.15$ & $1.0 \times 10^{-19}$ & $2.621 \times 10^{-20}$  \\
        S & $1.184 \times 10^{-18}$ & $6.99 \pm 0.22$ & $1.0 \times 10^{-19}$ & $3.797 \times 10^{-20}$  \\
        \bottomrule
    \end{tabularx}
    \caption{Low-precision data, collected by sampling many droplets single times. The purpose of this data is to primarily provide statistical motivation to my results, however there is very large statistical uncertainty associated with these.}
    \label{tab:oil-drops-lq}
\end{table}

\begin{figure}[H]
  \centering
  \includegraphics[width=0.45\textwidth]{../../DataCalculations/output_plot.png}
    \caption{A plot of the high-precision, low-precision data, and the commonly accepted values for electron charge. Units are plotted in relative charges, (elementary charge). The high-precision data has large uncertainty, primarily due to statistical sources, however the mean values in the high-precision data set lie very close to the commonly accepted values.}
  \label{fig:output-plot}
\end{figure}

\subsection{Results}

Table \ref{tab:oil-drops-hq} effectively demonstrates the quantized nature of electron charge, which is the same conclusion also found in other publications on the matter\cite{millikan1917electron}. Crucially, we also observe a unit electron charge of $e=1.695\times10^{-19} \pm 1.035\times10^{-19}$ coulombs. This is within 5.8\% of the commonly accepted value \ref{eq:percent-diff}. In the comprehensive data, we find a chi-squared parameter $\chi^2 = 0.39$, indicating that uncertainties are indeed very large. This is also clear in \ref{fig:output-plot}, where a summary of the data can be seen with the associated uncertainty.
