\documentclass{article}

\title{Homework 10}
\author{Grant Saggars}
\date{November 11, 2023}

\usepackage{amsmath}
\usepackage{import}
\usepackage{pdfpages}
\usepackage{transparent}
\usepackage{xcolor}
\usepackage{framed}
\usepackage{enumerate}
\usepackage{geometry}
\usepackage{cancel}
\usepackage{multicol}
\usepackage{lipsum}  
\usepackage{caption}
\usepackage{float}
\usepackage{bbold}
% \usepackage{fontspec}

% \setmainfont{BespokeSerif-Regular}
\definecolor{shadecolor}{RGB}{235,235,235}

\geometry{top=1in, bottom=1in, left=1in, right=1in}
\newenvironment{callout}[1] {\begin{shaded*} \textbf{#1}} {\end{shaded*}}

%%%%%%%%%%%%%%%%%%%%%%%%
% DOCUMENT BEGINS HERE %
%%%%%%%%%%%%%%%%%%%%%%%%

\begin{document}

\maketitle

% \section{Problems}

\begin{enumerate}

\item (4 pts) A particle is confined between rigid walls it cannot penetrate separated by a distance $L = 0.189 \text{ nm}$. The particle is in the second excited state. Evaluate the probability to find the particle in an interval of width 1.00 pm at $x = 0.188$ nm.

    \begin{callout}{Solution:}
        \begin{multicols}{2}

            \begin{center} This is an infinite square well problem: \end{center}

            Because potential is infinite, the probability of the particle being in region $I$ or $III$ is zero. Therefore, the wave equation is:

            \columnbreak

            \center
            Piecewise regions \& potentials:
            \begin{align*}
                \begin{array}{rcl} 
                    I: & x < 0 & u=\infty \\
                    II: & 0 \leq x \leq L & u= 0 \\
                    III: & x > L & u=\infty
                \end{array}    
            \end{align*}
        \end{multicols}

        \vspace{-0.8 cm} \begin{align*}
            -\frac{\hbar}{2m} \frac{\partial^2 \psi_{II}(x)}{\partial x^2} = E \psi_{II}(x) \to \psi'' + k^2 \psi = 0 && k = \sqrt{ \frac{2mE}{\hbar^2} }\\
            r^2 + k^2 = 0 \implies r = i \pm k
        \end{align*}

        Therefore the general solution is:

        \vspace{-0.8 cm} \begin{align*}
            \psi_{II}(x) = A \sin(kx) + B \cos(kx)
        \end{align*}

        \begin{enumerate}
            \item Continuity condition:

            \begin{align*}
                \psi_{II}(0)=\psi_{I}(0) &\to B \cancelto{1}{\cos(0)} = 0 \\
                \implies &B=0 \\
                \psi_{II}(L)=\psi_{III}(L) &\to A \sin(kx) = 0 \\
                \implies &A=0 \textrm{ or } kL = n \pi
            \end{align*}

        \item Normalization Condition:

            Because $|\psi_n(x)|^2$ is the PDF of the position of the particle:
            
            \vspace{-0.7 cm} \begin{align*}
                1 &= \int_{-\infty}^{\infty} |\psi_n(x)|^{2} ~dx 
                \to \int_{0}^{L} \psi_n(x) ~dx \\
                &= \int_{0}^{L} |A|^{2} \sin^{2}\left(\frac{n \pi x}{L}\right) ~dx \\
                &= A^2 \frac{L}{2} \implies A = \sqrt{ \frac{2}{L} }
            \end{align*}
        \end{enumerate}

        Therefore, our final wave equation is: $\sqrt{ \frac{2}{L} } \sin\left(\frac{n \pi x}{L}\right)$. To evaluate the given probability:
        \begin{align*}
            \int_{0.188\textrm{nm} - \frac{1}{2}\textrm{pm}}^{0.188\textrm{nm}+\frac{1}{2}\textrm{pm}}  
            \sqrt{ \frac{2}{L} } \sin\left(\frac{n \pi x}{L}\right)~dx = 0.0357\%
        \end{align*}
    \end{callout}

\newpage
\item Consider a Gaussian wavefunction:
$\psi(x) = A \exp\{-a (x - \mu)^2\}$, 
where $a > 0$.
\begin{enumerate}
\item (4 pts) Find the normalization factor $A$ in terms of $\mu$ and $a$. (You can look up integration of Gaussian’s an quote the results)

    \begin{callout}{Solution: (Polar Transformation)}

        \begin{align*}
            \psi = \int_{-\infty}^{\infty} \exp(-a(x-\mu)^2 ) ~dx \to \int_{-\infty}^{\infty} \exp(-a(u)^2) ~dx \tag{1}
        \end{align*}
        Transformation (1) has a jacobian of one, which should make sense since $x-\mu$ does not scale the function. Now, we can temporarily square $\psi$ to convert to polar coordinates.
        \begin{align*}
            \psi^2 &= \int_{-\infty}^{\infty} \exp(-a(u^2 + v^2)) ~dx \to \int_{0}^{2\pi} \int_{0}^{\infty} \rho \exp(-a\rho^2) ~d\rho~d\theta
        \end{align*}
        Finally, a substitution of $a\rho^2 = w$ allows the integral to be solved to completion:
        \begin{align*}
            \psi^2 &=\frac{\cancel{2}\pi}{\cancel{2}a} \int_{0}^{\infty} e^{-w} ~dx \\
             &= \frac{\pi}{a} \left(-e^{-w}\middle|_{0}^{\infty} \right) \\
             &= \sqrt{ \frac{\pi}{a} } 
        \end{align*}

        (This is the scale factor, so the inverse of this is the normalization factor $A$). Plotting the normalized wavefunction again shows that $\mu$ has no effect on the scale of the function, so it is normalized for all $\mu$.
    \end{callout}

\item (4 pts) What is the probability that $x < \mu$? 

    \begin{callout}{Solution:}

        $\mu$ represents the location of the center of the wave function for all $\mu$. Because the wave function is a (symmetrical) gaussian, this also is where the peak of the PDF occurs. Therefore, for $x < \mu$, the particle would be to the left of the center of the PDF. Therefore the probability is 50\%
    \end{callout}

\newpage
\item (4 pts) For $a = 0.05 \text{ m}^{-2}$ and $\mu = 1.3\text{m}$, find the average position of the particle.

    \begin{callout}{Solution:}

        The average position is defined as $\int_{-\infty}^{\infty} x|\psi(x)|^{2} ~dx$.
        \begin{align*}
           \sqrt{ \frac{\pi}{a} }^{-2} \int_{-\infty}^{\infty} x \exp^2(-a(x-\mu)^{2}) ~dx 
           \to \frac{a}{\pi} \int_{-\infty}^{\infty} x \exp(-2a(x-\mu)^{2}) ~dx 
        \end{align*}

        \begin{enumerate}[(1)]
            \item make the substitution: $x-\mu \to u$, $dx = du$

                \begin{align*}
                    \frac{a}{\pi} \int_{-\infty}^{\infty} (u + \mu) \exp(-2au^{2}) ~du \\
                    \to \frac{a}{\pi} \left( \int_{-\infty}^{\infty} u\exp(-2au^2) ~du + \int_{-\infty}^{\infty} \mu\exp(-2au^2) ~du \right)
                \end{align*}

            \item The first integral can be solved with a single substitution
                \begin{align*}
                    \int_{-\infty}^{\infty} u\exp(-2au^2) ~du &= \left(-\exp(-2au^2) \cdot \frac{1}{4a}\middle) \right|_{-\infty}^\infty = 0 \\
                \end{align*}

            \item The second integral is a gaussian, and can be solved like in part (a).
                \begin{align*}
                    \int_{-\infty}^{\infty} \mu\exp(-2au^2) ~du 
                    &= \left( \mu^2 \int_{-\infty}^{\infty} \int_{-\infty}^{\infty} \exp(-2a(x^2+y^2)) ~dy ~dx \right)^\frac{1}{2} \\ 
                    &= \left(\mu^2\int_0^{2\pi} \int_{0}^{\infty} \rho \exp(-2a\rho^2) ~d\rho~d\theta\right)^{\frac{1}{2}} \\
                    &= \left(\mu^2\frac{2\pi}{2a} \int_{0}^{\infty} e^{-w} ~dw \right)^{\frac{1}{2}} \\
                    &= \mu \sqrt{ \frac{\pi}{a} }
                \end{align*}

            \item Substituting these into (1):

                \begin{align*}
                    \frac{a}{\pi} \left(0 + \mu \sqrt{ \frac{\pi}{a} }\right) = \frac{\mu a}{\pi} \sqrt{ \frac{\pi}{a} } = 0.164~\textrm{m}
                \end{align*}

        \end{enumerate}
    \end{callout}

\newpage
\item (7 pts) Standard deviation is defined as: 
$\sigma = \sqrt{[x^2]_{\text{ave}} - (x_{\text{ave}})^2}$,
where $[x^2]_{\text{ave}}$ is the average of $x^2$ and $x_{\text{ave}}$ is the average position. Find the standard deviation of the Gaussian wavefunction for arbitrary $a$, $\mu$.

\begin{callout}{Solution:}
    
    $x_{\textrm{ave}}$ was found in part (c). $x^2_{\textrm{ave}}$ can be found as:

    \vspace{-0.3 cm}\begin{align*}
        \frac{a}{\pi} \int_{-\infty}^{\infty} x^2 \exp(-2a(x-\mu)^2) ~dx 
        &= -2 \frac{d}{da} \int_{-\infty}^{\infty} \exp(-2a(x-\mu)^2) ~dx \quad \textrm{(part a.)}\\
        &= -2 \frac{d}{da} \sqrt{ \frac{\pi}{a} } \\
        &= \frac{4}{a} \sqrt{ \frac{\pi}{a} }
    \end{align*}

    Therefore, the standard deviation $\sigma$ is:
    \begin{align*}
        \sqrt{ \frac{4}{a} \sqrt{ \frac{\pi}{a} } - \left(\frac{\mu a}{\pi} \sqrt{ \frac{\pi}{a} }\right)^{2} }
    \end{align*}

\end{callout}

\end{enumerate}
\end{enumerate}
\end{document}

