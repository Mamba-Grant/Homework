\documentclass{article}

\title{Physics 313 Homework 12}
\author{Grant Saggars}
\date{\today}

\usepackage{amsmath}
\usepackage{import}
\usepackage{pdfpages}
\usepackage{transparent}
\usepackage{xcolor}
\usepackage{framed}
\usepackage{enumerate}
\usepackage{geometry}
\usepackage{cancel}
\usepackage{multicol}
\usepackage{lipsum}  
\usepackage{caption}
\usepackage{float}
\usepackage{bbold}
% \usepackage{fontspec}

% \setmainfont{BespokeSerif-Regular}
\definecolor{shadecolor}{RGB}{235,235,235}

\geometry{top=1in, bottom=1in, left=1in, right=1in}
\newenvironment{callout}[1] {\begin{shaded*} \textbf{#1}} {\end{shaded*}}

%%%%%%%%%%%%%%%%%%%%%%%%
% DOCUMENT BEGINS HERE %
%%%%%%%%%%%%%%%%%%%%%%%%

\begin{document}
\maketitle

\begin{enumerate}
\item (4 pts) Calculate the average radius of a hydrogen atom in the first excited state with $l = 1$.

\begin{callout}{Solution:}

    A hydrogen atom in the first excited state with $l=1$ has an average radius:
    $$ R_{avg}(r) = \int_{0}^{\infty} rP(r) ~dr = \int_{0}^{\infty} r^3 \left( \frac{1}{ \sqrt{ 3 } (2a_0)^{3/2}} \frac{r}{a_0} e^{-r/2a_0} \right)^{2} ~dr $$
    Numerically, this comes out to be approximately 25 nm.
\end{callout}

\item (3 pts) What is the most probable radius of a hydrogen atom in the first excited state with $l = 1$ and $l = 0$?

\begin{callout}{Solution:}

    A hydrogen atom in the first excited state with $l$ = 0 has a radial probability:
    $$ P(r) = r^2R^2(r) = r^2 \left( \frac{1}{(2a_0)^{3/2}} \left( 2 - \frac{r}{a_0} \right) e^{-r/2a_0} \right)^{2}$$
    The radial probability is at a maximum when $r$ is approximately 27.699 nm.

    A hydrogen atom in the first excited state with $l$ = 1 has a radial probability:
    $$ P(r) = r^2R^2(r) = r^2 \left( \frac{1}{ \sqrt{ 3 } (2a_0)^{3/2}} \frac{r}{a_0} e^{-r/2a_0} \right)^{2} $$
    The radial probability is at a maximum at exactly $r = 4a_0 \approx 21.16$ nm.
\end{callout}

\item (2 pts) What is the configuration of a magnesium atom (Z = 12)?

\begin{callout}{Solution:}
    
    A magnesium atom has 12 electrons, so the configuration is 1s, 2s, 2p, 3s. 2 electrons fill subshells 1s and 2s, 6 fill 2p, and 2 more occupy 3s.

    \begin{itemize}
        \item In s1: $n$=1, $l$=0, implying two electrons with opposite spin
        \item In s2: $n$=2, $l$=0, implying the same
        \item In 2p: $n$=2, $l$=1, implying $2l+1=3 \times 2 = 6$ combinations of electrons
        \item In 3s: $n$=3, $l$=1, implying the same
    \end{itemize}

\end{callout}

\item (3 pts) Consider an electron in an infinite square well with a length of L = 2.0 nm.  
The electron is in a state with a wavefunction of the form
\begin{equation}
\psi(x) = e^{i K x} \frac{1}{\sqrt{L}},  
\end{equation}
where $K = 10^{10} \ \mathrm{m}^{-1}$. What is the probability that when we measure the energy of the electron, we find it to be in the ground state?

\begin{callout}{Solution:}
    

\end{callout}

\end{enumerate}

\end{document}

\end{document}
