\documentclass{article}


\newcommand{\hmwkTitle}{Homework 2}
\newcommand{\hmwkDueDate}{\today}
\newcommand{\hmwkClass}{MATH 648}
\newcommand{\hmwkAuthorName}{\textbf{Grant Saggars}}



\usepackage{fancyhdr}
\usepackage{extramarks}
\usepackage{amsmath}
\usepackage{amsthm}
\usepackage{amsfonts}
\usepackage{tikz}

\usepackage{float}
\usepackage{caption}
\usepackage{bbold}
\usepackage{xcolor}
\usepackage{framed}
\usepackage{enumerate}
\usepackage{cancel}
\usepackage{multicol}
\usepackage{XCharter}

\usetikzlibrary{automata,positioning}

\usepackage{geometry}
\geometry{top=1in, bottom=1in, left=1in, right=1in} % Adjust margins as needed

\pagestyle{fancy}
\lhead{\hmwkAuthorName}
\chead{\hmwkClass\: \hmwkTitle}
\rhead{\firstxmark}
\lfoot{\lastxmark}
\cfoot{\thepage}

%
% Basic Document Settings
%

\topmargin=-0.75in
\evensidemargin=0in
\oddsidemargin=0in
\textwidth=6.5in
\textheight=9.0in
\headsep=0.25in

\linespread{1.1}

\renewcommand\headrulewidth{0.4pt}
\renewcommand\footrulewidth{0.4pt}

\setlength\parindent{0pt}

%
% Create Problem Sections
%

\newcommand{\enterProblemHeader}[1]{
    \nobreak\extramarks{}{Problem \arabic{#1} continued on next page\ldots}\nobreak{}
    \nobreak\extramarks{Problem \arabic{#1} (continued)}{Problem \arabic{#1} continued on next page\ldots}\nobreak{}
}

\newcommand{\exitProblemHeader}[1]{
    \nobreak\extramarks{Problem \arabic{#1} (continued)}{Problem \arabic{#1} continued on next page\ldots}\nobreak{}
    \stepcounter{#1}
    \nobreak\extramarks{Problem \arabic{#1}}{}\nobreak{}
}

\setcounter{secnumdepth}{0}
\newcounter{partCounter}
\newcounter{homeworkProblemCounter}
\setcounter{homeworkProblemCounter}{1}
\nobreak\extramarks{Problem \arabic{homeworkProblemCounter}}{}\nobreak{}

%
% Homework Problem Environment
%
% This environment takes an optional argument. When given, it will adjust the
% problem counter. This is useful for when the problems given for your
% assignment aren't sequential. See the last 3 problems of this template for an
% example.
%
\newenvironment{homeworkProblem}[1][-1]{
    \ifnum#1>0
        \setcounter{homeworkProblemCounter}{#1}
    \fi
    \section{Problem \arabic{homeworkProblemCounter}}
    \setcounter{partCounter}{1}
    \enterProblemHeader{homeworkProblemCounter}
}{
    \exitProblemHeader{homeworkProblemCounter}
}

%
% Callout Box
%

\definecolor{shadecolor}{RGB}{235,235,235}
\newenvironment{callout}[1] {\begin{shaded*} \textbf{#1}} {\end{shaded*}}

%
% Title Page
%

\title{
    \textmd{\textbf{\hmwkClass:\ \hmwkTitle}}\\
    \normalsize\vspace{0.1in}\small{\hmwkDueDate}\\
}

\author{\hmwkAuthorName}
\date{}

\renewcommand{\part}[1]{\textbf{\large Part \Alph{partCounter}}\stepcounter{partCounter}\\}





\begin{document}

\maketitle

% \vspace{0.3cm} \hrule

\begin{itemize}
	\item Section 2.5, \#2, \#3.
	\item Section 3.1, Page 59: \#1, \#2, \#3.
	\item Section 3.2, Page 64: \#1, \#2, \#3.
\end{itemize}

[Hint for §3.1 problem \#3: This problem is tricky. Note that the integrand $F = y'\sqrt{1 + (y'')^2}$ does not depend on $y$ neither on $x$. Thus,
$\dfrac{d}{dx}Fy'' - Fy' = c_1$ and $H = y''\dfrac{F}{y''} - y'\left(\dfrac{d}{dx}Fy'' - Fy'\right) - F = c_2$.
Together, they give
$y''\dfrac{F}{y''} - c_1y' - F = c_2$ or $y'\left( \dfrac{y''^2}{\sqrt{1 + y''^2}} - c_1 - \sqrt{1 + y''^2}\right) = c_2$. (1)
Normally, one would solve for $y''$ form (1) to get $y'' = g(y', c_1, c_2)$, find a general solution for
$y$, then, apply the boundary conditions at the last moment. For this problem, this general
approach would be extremely complicated. Instead, the observation is that, the boundary
condition $y'(0) = 0$ allows you to conclude, from equation (1), that $c_2 = 0$.]

[Hint for §3.2 problem \#3: For $k \neq 0$, apply formula (3.17) first and then use the E-L for
$q_1$-component. For $k=0$, make the change of variable $q_0 = \dfrac{q_2^2}{2}$ to simplify the functional.]

\vspace{0.3cm} \hrule \newpage

\section{Section 2.5 (Invariance/Substitution)}

\begin{homeworkProblem}[2]
	Let $J$ be the functional defined by
	$$ J(r)=\int_{\pi / 2}^\pi \sqrt{r^2+\dot{r}^2} d \phi . $$

	Find an extremal for $J$ satisfying the boundary conditions $r(\pi / 2)=1$ and $r(\pi)=-1$.

	\begin{callout}{Solution:}

		The integrand $F$ does not explicitly depend on $\phi$, so the E-L equation goes to:
		\begin{align*}
			\dot{r} \frac{dF}{d \dot{r}} - F                                                        & = C                            \\
			\dot{r} \frac{\dot{r}}{\sqrt{ r^2 + \dot{r}^2 }} - \sqrt{ r^2 + \dot{r}^2 }             & = C                            \\
			\frac{\cancel{\dot{r}^{2}} - (r^2 + \cancel{\dot{r}^{2}}) }{\sqrt{ r^2 + \dot{r}^{2} }} & = C                            \\
			\frac{r ^{2}}{C ^{2}} = r ^{2} + \dot{r} ^{2}                                                                            \\
			\dot{r} = \frac{dr}{d\phi}                                                              & = \frac{r}{C} \sqrt{ r^2-C^2 } \\
			\int \frac{C}{r \sqrt{ r^2 - C ^{2} }} ~dr                                              & = \int d\phi
			% -r ^{2}                                                                                 & = C \sqrt{ r^2 + \dot{r}^2 } \\
			% -r^4                                                                                    & = C^2r^2 + C^2\dot{r}^2      \\
			% \dot{r} + \sqrt{ \frac{r ^{4}}{C ^{2}} +  r ^{2} }                                      & = 0
		\end{align*}
		$$ \arctan\left( \frac{\sqrt{ r^2 - C^2 }}{C} \right) + c_1 = \phi $$

		Solving this for $r(\phi)$ gives (Applying the trig identity $\sec^2 - 1 = \tan^2$):

		\begin{align*}
			\frac{\sqrt{ r^2 - C^2 }}{C^2} & = \tan(\phi-c_1)                            \\
			r^2                            & = C^2\tan^2\phi + C^2 = C^2\sec^2(\phi-c_1) \\
			r                              & = C\sec(\phi - c_1)
		\end{align*}

		For the boundary conditions:
		\begin{align*}
			r(\pi/2) & = 1 \implies 1 = C\sec(\pi/2-c_1) \\
			r(\pi)   & = -1 \implies -1 = C\sec(\pi-c_1)
		\end{align*}
		\begin{gather*}
			A = \frac{1}{\sqrt{ 2 }}, \quad B = \left( \pi n + \frac{\pi}{8} \right)
		\end{gather*}

	\end{callout}

\end{homeworkProblem}

\begin{homeworkProblem}[3]
	Let $J$ be a functional of the form
	$$ J(y)=\int_{x_0}^{x_1} g\left(x^2+y^2\right) \sqrt{1+y^{\prime 2}} d x $$
	where $g$ is some function of $x^2+y^2$. Use the polar coördinate transformation to find the general form of the extremals in terms of $g, r$, and $\phi$.
	\begin{callout}{Solution:}

		Let
		\begin{gather}
			g(x^{2} + y^{2}) \to g(r^{2}) \\
			\sqrt{ 1+y'^{2} } ~dx \to \sqrt{ r ^{2} + \dot{r}^{2} } ~d\phi
		\end{gather}
		(these substitutions are made in the textbook example 2.5.1)
		$$ J(y) = \int_{\phi_0}^{\phi_1} g(r ^{2}) \sqrt{ r^2 + \dot{r}^{2} } ~d\phi $$
		The E-L equation for this is:
		$$ g(r^2) \frac{\dot{r}^2}{\sqrt{ r^2+\dot{r}^2 }} - g(r^2) \sqrt{ r^2 + \dot{r}^2 } = C $$
		The final solution to this differential equation is:
		$$ g(r^2) r(\phi) = C\sqrt{\tan^2(c_1C + \phi) + 1} $$

	\end{callout}
\end{homeworkProblem}

% \vspace{1.5cm} \hrule \newpage

\section{Section 3.1 (Second Variation)}
\begin{homeworkProblem}[1]
	Find the general solution for the extremals to the functional $J$ defined by
	$$ J(y)=\int_{x_0}^{x_1}\left(\left(y^{\prime \prime}\right)^2-y^2+2 y x^3\right) d x . $$

	\begin{callout}{Solution:}

		% $$ \frac{\delta J}{\delta y} = f_{y}-\frac{d}{dx}(f_{y'})-\frac{d^{2}}{dx^{2}}(f_{y''})=0. $$
		Because there is no explicit $y'$ dependence, the E-L equation goes to:
		\begin{align*}
			\frac{d ^{2}}{dx ^{2}}f_{y''} - f_y                    & = C          \\
			\frac{d ^{2}}{dx ^{2}} 2y'' - \left( 2y + 2x^3 \right) & = C          \\
			2y^{(4)} - 2y                                          & = C + 2x^{3}
		\end{align*}
		The characteristic polynomial for this is $2 \lambda^4 - 2 = 0$, which gives solutions $\lambda = \{ 1, -1, i, -i \}$.
		\begin{align*}
			y_c(x)     & = c_1e^x + c_2e^{-x} + c_3e^{ix} + c_4e^{-ix}                                     \\
			y_c'(x)    & = c_1e^x - c_2e^{-x} + c_3ie^{ix} - c_4ie^{-ix}                                   \\
			y_c''(x)   & = c_1e^x + c_2e^{-x} + c_3\cancelto{-1}{i^2}e^{ix} + c_4\cancelto{-1}{i^2}e^{-ix} \\
			y_c'''(x)  & = c_1e^x - c_2e^{-x} - c_3ie^{ix} + c_4ie^{-ix}                                   \\
			y_c''''(x) & = c_1e^x + c_2e^{-x} - c_3\cancelto{-1}{i^2}e^{ix} - c_4\cancelto{-1}{i^2}e^{-ix}
		\end{align*}
		Substituting these:
		\begin{align*}
			2\left[ \cancel{c_1e^x} + c_2e^{-x} + c_3e^{ix} + c_4e^{-ix}
			- \cancel{c_1e^x} + c_2e^{-x} + c_3e^{ix} + c_4e^{-ix} \right] & = C+2x^3                    \\
			\cancel{2}[ 2c_2e^{-x} + 2c_3e^{ix} + 2c_4e^{-ix} ]            & = \frac{C}{2}+\cancel{2}x^3 \\
			y_p = 2c_2e^{-x} + 2c_3e^{ix} + 2c_4e^{-ix} - \frac{C}{2} - x^3
		\end{align*}
		General solution:
		\begin{align*}
			y(x) & = c_1e^x+3c_2e^{-x}+3c_3e^{ix}+3c_4e^{-ix}-\frac{C}{2}-x^3
		\end{align*}

	\end{callout}

\end{homeworkProblem}

% \newpage
\begin{homeworkProblem}
	Conservation Law: Suppose the integrand $f$ defining the functional $J$ does not depend on $x$ explicitly. Prove that equation (3.4) is satisfied along any extremal.
	\begin{callout}{Solution:}

		\begin{align*}
			y''f_{y''}-y'\left( \frac{d}{dx} f_{y''}-f_{y'} \right) - f = const \tag{3.4}
		\end{align*}
		Suppose that y is an extremal for J. Now,
		\begin{gather*}
			\frac{d}{dx}\left( y''f_{y''}-y'\left( \frac{d}{dx} f_{y''}-f_{y'} \right) - f \right) \\
			= y'''\frac{\partial f}{\partial y''} + y'' \frac{d}{dx} \frac{\partial f}{\partial y''} - y'' \frac{d}{dx} \frac{\partial f}{\partial y''} - y' \frac{d^2}{dx^2} \frac{\partial f}{\partial y''} + y' \frac{\partial f}{\partial y'} + y' \frac{d}{dx} \frac{\partial f}{\partial y'} - \frac{d}{dx}(f(y,y',y'')) \\
			= \cancel{y'''\frac{\partial f}{\partial y''}} + \cancel{y'' \frac{d}{dx} \frac{\partial f}{\partial y''}} - \cancel{y'' \frac{d}{dx} \frac{\partial f}{\partial y''}} - y' \frac{d^2}{dx^2} \frac{\partial f}{\partial y''} + \cancel{y'' \frac{\partial f}{\partial y'}} + y' \frac{d}{dx} \frac{\partial f}{\partial y'} - y' \frac{\partial f}{\partial y} -  \cancel{y'' \frac{\partial f}{\partial y'}} - \cancel{y''' \frac{\partial f}{\partial y''}} \\
			= -y'\left( f_{y} - \frac{d}{dx}f_{y'} - \frac{d^2}{dx^2}f_{y''} \right)
		\end{gather*}
		Since y is an extremal, the E-L equation is satisfied.

	\end{callout}
\end{homeworkProblem}

\begin{homeworkProblem}
	For the functional $J$ defined by
	$$ J(y)=\int_0^1 y^{\prime} \sqrt{1+\left(y^{\prime \prime}\right)^2} d x $$
	find an extremal satisfying the conditions $y(0)=0, y^{\prime}(0)=0, y(1)=1$, and $y^{\prime}(1)=2$.
	\begin{callout}{Solution:}

		% I do not recall any particular conditions which would allow me to simplify this problem using the boundary conditions alone, so I will proceed using the special case of the E-L where there is no y-dependence:
		$$ f_{y'} - \frac{d}{dx}f_{y''} = 0 $$

		\begin{gather*}
			\implies \sqrt{ 1+y''^{2} } - \frac{d}{dx}\left(y'\frac{y''}{\sqrt{ 1+y''^{2} }}\right) = 0 \\
			\sqrt{ 1+y''^{2} } - \frac{y''^{4} + y''^{2} + y^{(3)}y'}{(1+y''^2)^{3/2}} = 0 \\
			\frac{\cancel{y''^4} + \cancel{2y''^2} + 1}{(1+y''^2)^{3/2}} - \frac{\cancel{y''^{4}} + \cancel{y''^{2}} + y^{(3)}y'}{(1+y''^2)^{3/2}} = 0 \\
			\frac{1}{\cancel{(1+y''^{2})^{3/2}}} = \frac{y^{(3)}y'}{\cancel{(1+y''^2)^{3/2}}} \\
			y^{(3)}y' = 1
		\end{gather*}
		Unfortunately we never worked with higher order differential equations in MATH 220, so I have no idea how to solve this directly, however, I am very curious if the special form of the functional briefly discussed in class could be applied: Let $y''' = z''$, $y'' = z'$, and $y' = z$, subject to the constraint $\int_{0}^{1} z(x) ~dx = y_1 - y_0$. This may be an inappropriate time to apply this substitution, however, I will try continuing regardless.
		$$ y''y' = 1 $$
		(can a laplace transformation be used here?)
		\begin{align*}
			z(x) = y'(x) & = c_2 + \frac{2}{3}\sqrt{ 2 } (c_1 + x)^{3/2}                                                                       \\
			y(x)         & = \int c_2 + \frac{2}{3}\sqrt{ 2 } (c_1 + x)^{3/2} ~dx                                                              \\
			             & = c_2x + \int \frac{2}{3}\sqrt{ 2 } (c_1 + x)^{3/2} ~dx                                                             \\
			             & = c_2x+\frac{2\sqrt{2}}{15}\left(2x\left(c_1+x\right)^{\frac{3}{2}}+2c_1\left(x+c_1\right)^{\frac{3}{2}}\right)+c_3 \\
			             & = \frac{4 \sqrt{ 2 }}{15} (c_1+x)^{5/2} + c_2x + c_3
		\end{align*}
		Now, applying the constraints:
		\begin{gather*}
			y_1 - y_0 = 1 = \frac{4 \sqrt{ 2 }}{15} (c_1+x)^{5/2} + c_2x + c_3 \\
			0 = c_2 + \frac{3 \sqrt{ 2 }}{2} (c_1)^{3/2} \\
			2 = c_2 + \frac{3 \sqrt{ 2 }}{2} (c_1 + 1)^{3/2}
		\end{gather*}

	\end{callout}
\end{homeworkProblem}

\section{Section 3.2}
\begin{homeworkProblem}[1]
	Let
	$$ L(t, \mathbf{q}, \dot{\mathbf{q}})=\frac{1}{2}\left(\dot{q}_1^2+\dot{q}_2^2\right)-g q_2, $$
	where $g$ is a constant.
	\begin{enumerate}[(a)]
		\item Find the extremals for the functional $J$ defined by
		      $$ J(\mathbf{q})=\int_{t_0}^{t_1} L(t, \mathbf{q}, \dot{\mathbf{q}}) d t . $$

		      \begin{callout}{Solution:}

			      $$ \frac{d}{dt} \frac{\partial L}{\partial \dot{q}_{j}} - \frac{\partial L}{\partial q_j} = 0 $$
			      The special case could be applied here, however it gives a horrible differential equation whereas the standard form is much kinder:
			      \begin{align*}
				      \frac{d}{dt}\left( \dot{q}_1 \right) = \ddot{q}_{1} = 0 \\
				      \frac{d}{dt}\left( \dot{q}_{2}  \right) - g = \ddot{q}_{2} - g = 0
			      \end{align*}
			      This gives a beautifully independent system of linear differential equations with solutions:
			      \begin{align*}
				      q_1(t) & = c_2(t)+c_1                  \\
				      q_2(t) & = \frac{1}{2}gt^2 +c_2t + c_1
			      \end{align*}

		      \end{callout}
		\item Verify that equation (3.17) is satisfied.
		      \begin{callout}{Solution:}

			      Showing that equation 3.17 is satisfied implies showing that energy is conserved along an extremal. Following the Hamiltonian definition of energy: $E = T(\dot{q})+V(q)$ and applying equation 3.17:
			      \begin{align*}
				      H & = \dot{q}\frac{\partial L}{\partial \dot{q}} - L  \\
				        & = \dot{q} \frac{\partial T}{\partial \dot{q}} - L \\
				        & = \cancelto{2T}{2m\dot{q}^{2}} - T + V            \\
				        & = T+V
			      \end{align*}
			      Therefore, energy is conserved along any extremal.

		      \end{callout}
	\end{enumerate}
\end{homeworkProblem}


\begin{homeworkProblem}[2]
	Prove equation (3.17).
	\begin{callout}{Solution:}

		Suppose that there is an etremal $q_j$ for $L$, given that $L$ is a functional: $L(q, \dot{q})$ Now,
		\begin{align*}
			\frac{d}{dt}\left( \dot{q} \frac{\partial L}{\partial \dot{q}} - L \right) & =
			\ddot{q} \frac{\partial L}{\partial \dot{q}} + \dot{q} \frac{d}{dt} \frac{\partial L}{\partial \dot{q}} - \frac{d}{dt}\left( L(q, \dot{q}) \right)                                                                                                                                           \\
			                                                                           & = \ddot{q} \frac{\partial L}{\partial \dot{q}} + \dot{q} \frac{d}{dt} \frac{\partial L}{\partial \dot{q}} - \left( \dot{q} \frac{\partial L}{\partial q} + \ddot{q} \frac{\partial L}{\partial \dot{q}} \right) \\
			                                                                           & = \dot{q} \left( \frac{d}{dt} \frac{\partial L}{\partial \dot{q}} - \frac{\partial L}{\partial q} \right)
		\end{align*}

	\end{callout}
\end{homeworkProblem}

\begin{homeworkProblem}[3]
	Let
	$$ L(t, \mathbf{q}, \dot{\mathbf{q}})=\sqrt{\dot{q}_1^2+q_2^2 \dot{q}_2^2}-k q_2, $$
	where $k$ is a constant. Find the extremals for the functional $J$ defined by
	$$ J(\mathbf{q})=\int_{t_0}^{t_1} L(t, \mathbf{q}, \dot{\mathbf{q}}) d t . $$
	\begin{callout}{Solution:}

		$$ \sum_{j=1}^{n}q_{j}L_{\dot{q}_{j}}-L=C $$

		\begin{align*}
			q_1 \frac{\dot{q_1}}{\sqrt{ \dot{q}_{1}^{2} + q_2^{2}\dot{q}_{2}^{2} }} - \sqrt{\dot{q}_1^2+q_2^2 \dot{q}_2^2}-k q_2 = C \\
			q_2 \frac{\dot{q_2}}{\sqrt{ \dot{q}_{1}^{2} + q_2^{2}\dot{q}_{2}^{2} }} - \sqrt{\dot{q}_1^2+q_2^2 \dot{q}_2^2}-k q_2 = C \\
		\end{align*}

		I unfortunately have no idea how to even begin to approach these differential equations, and attempting the typical E-L equation would likely produce an even more difficult system of equations with higher order derivatives.

	\end{callout}
\end{homeworkProblem}
\end{document}
