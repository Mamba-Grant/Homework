\begin{homeworkProblem}
    5.1: Determine an equation for the chemical potential of a Van der Waals gas.
    \begin{callout}{Solution Attempt \#2:}
        
        By equation 5.14 and 5.8:
        $$
        \frac{\mu}{T}=-\left( \frac{\partial S}{\partial N} \right)_{V,N}, \qquad 
        S_{\text{VDW}} = Nk_B\left[\ln \left(\frac{V-b N}{N}\right)+\frac{3}{2} \ln \left(\frac{2 m \pi}{\beta h^2}\right)+\frac{5}{2}\right]
        $$

            $$-\left( \frac{\partial S}{\partial N} \right)_{V,P} = -\left[\ln \left(\frac{V-b N}{N}\right)+\frac{3}{2} \ln \left(\frac{2 m \pi}{\beta h^2}\right)+\frac{5}{2}\right] + \frac{1}{N}$$
            \begin{align*}
                \frac{\mu}{T} &= -k_B\left[\ln \left(\frac{V-b N}{N}\right)+\frac{3}{2} \ln \left(\frac{2 m \pi}{h^2} T \right)+\frac{5}{2}\right] + \frac{k_B}{N} \\ 
            \end{align*}
            $$\boxed{\mu = \frac{k_BT}{N} - k_BT\ln\left( \frac{V-bN}{N} \right) + \frac{3T}{2} \ln\left( \frac{2m\pi}{h^2}T \right) + \frac{5k_B}{2}}$$


    \end{callout}
\end{homeworkProblem}

\begin{homeworkProblem}
    5.2: Determine an equation for \( C_V \) for a Van der Waals gas.
    \begin{callout}{Solution:}
        
        By equation 5.15 and 5.8:
        $$
        C_V=- \beta \left( \frac{\partial S}{\partial \beta} \right)_{V}, \qquad 
        S_{\text{VDW}} = Nk_B\left[\ln \left(\frac{V-b N}{N}\right)+\frac{3}{2} \ln \left(\frac{2 m \pi}{\beta h^2}\right)+\frac{5}{2}\right]
        $$

        $$-\left( \frac{\partial S}{\partial \beta} \right)_{V} = \frac{3Nk_B}{2 \beta} \implies \boxed{C_V = \frac{3Nk_B}{2}}$$

    \end{callout}
\end{homeworkProblem}

\newpage
\begin{homeworkProblem}
    5.3: Determine an equation for \( C_P \) for a Van der Waals gas.
    \begin{callout}{Solution Attempt \#2:}
        
        By equation 5.17 and 5.8:
        $$
        C_P=C_V+T\left( \frac{\partial P}{\partial T} \right)_{V} \left( \frac{\partial V}{\partial T} \right)_{P}, \qquad 
        \begin{cases}
            S_{\text{VDW}} = N\left[\ln \left(\frac{V-b N}{N}\right)+\frac{3}{2} \ln \left(\frac{2 m \pi}{\beta h^2}\right)+\frac{5}{2}\right] \\ 
            \left( P+a \frac{N^2}{V^2} \right)\left( V-bN \right) = \frac{N}{\beta}
        \end{cases}
        $$

        \begin{align*}
            \left(\frac{\partial P}{\partial T} \right)_{V} &= \frac{\partial}{\partial T}\left( \frac{Nk_BT}{V-bN} - a \frac{N^2}{V^2} \right) = \frac{Nk_B}{V-bN} \\ 
            %\left(\frac{\partial V}{\partial T} \right)_{P} &= \frac{\partial}{\partial T}\left( \frac{Nk_BT}{P+a \frac{N^2}{V^2}}+bN \right) = \frac{Nk_B}{P+a \frac{N^2}{V^2}} \\ 
        \end{align*}
        On my first attempt, I incorrectly solved for $V$ algebraically. Doing this the right way would lead to some cubic equation which I am not equipped to differentiate.
        A bit of research reminded me of implicit differentiation, which I have not used in a few years.

        \vspace{1em} From the VDW equation: $[P + a (N^2/V^2)][V - bN] = Nk_BT$
        Taking the partial derivative with respect for $T$ for constant $P$ to both sides:

        \begin{align*}
            \frac{\partial}{\partial T}\left( P+a \frac{N^2}{V^2} \right) (V-bN) + \left(P+a \frac{N^2}{V^2}\right) \frac{\partial}{\partial T} \left( V-bN \right) &= \frac{\partial}{\partial T} (Nk_BT) \\ 
            (V-bN)\left( -2a \frac{N^2}{V^3} \right) \left( \frac{\partial V}{\partial T} \right)_{P}  + \left(P+a \frac{N^2}{V^2}\right) \left( \frac{\partial V}{\partial T} \right)_{P} &= Nk_B \\ 
            \left[P+a \frac{N^2}{V^2} -\frac{2N^2a\left(V-bN\right)}{V^3}\right]\left( \frac{\partial V}{\partial T} \right)_{P} &= Nk_B \\
            \left( \frac{\partial V}{\partial T} \right)_{P} &= \frac{Nk_B}{P+a \frac{N^2}{V^2} -\frac{2N^2a\left(V-bN\right)}{V^3}} \\ 
            \left( \frac{\partial V}{\partial T} \right)_{P} &= \frac{Nk_BV^3}{PV^3-N^2aV+2N^3ab}
        \end{align*}

            $$\boxed{C_P = \frac{3}{2}Nk_B + T\left( \frac{Nk_B}{V-bN} \right) \left( \frac{Nk_BV^3}{PV^3-N^2aV+2N^3ab} \right)}$$
            %&= \frac{3Nk_B}{2 \beta} + \left( \frac{T\left( Nk_B \right)^{2}}{(V-bN)(P+a \frac{N^2}{V^2})} \right) 

    \end{callout}
\end{homeworkProblem}

\newpage
\begin{homeworkProblem}
    5.4: Consider a real gas that is confined within a vertical box of cross-sectional area \( A \). The molecules of this gas will have translational kinetic energy and gravitational potential energy, but no other kinetic or potential energies. You can also assume that the molecules of the gas are indistinguishable. Calculate the partition function for a horizontal slice of this gas between a vertical position of \( z \) and \( z + dz \) above the bottom of the box.
    \begin{callout}{Solution:}
       $$Z = \left( \frac{e}{Nh^{3}} \right)^{N}Z_{K}Z_{U}$$ 
        \begin{align*}
            Z_{K}&=\int e^{-\beta K(\bar{p}_{1},\bar{p}_{2},\dots)}\,d^{3}p_{1}~d^{3}p_{2} \dots d^{3}p_{N} \\
            Z_{U}&=\int e^{-\beta K(\bar{q}_{1},\bar{q}_{2},\dots)}\,d^{3}q_{1}~d^{3}q_{2} \dots q^{3}p_{N}
        \end{align*}
        For kinetic energy:
        \begin{align*}
            Z_K &= \left( \int_0^{\infty} e^{\frac{\beta}{2m} p^2 } 4\pi p^2 ~dp \right)^N \\ 
            &= \left( 2 \sqrt{ 2 } \left( \frac{m\pi}{\beta} \right)^{3/2} \right)^{N} \\ 
            &= \left( \frac{2m\pi}{\beta} \right)^{3N/2}
        \end{align*}
        For gravitational potential energy, we have $\epsilon=mgz$. the $x-y$ integrals give a since its gravitational:
        \begin{align*}
            Z_U &= \left( A\int_{z}^{z+dz} e^{-\beta mgz} ~dz \right)^{N} \\ 
            &=\left[\left(-\frac{1}{\beta mg} e^{-\beta mgz} \middle)\right|_{z}^{z+dz}\right]^{N} = \left[\frac{1}{\beta mg}\left( e^{-\beta mgz}- e^{-\beta mg(z+dz)} \right)\right]^{N}
        \end{align*}
        \begin{align*}
            Z &= Z_KZ_U \\ 
            &= A\left( \frac{e}{Nh^{3}} \right)^{N}\left( \frac{2m\pi}{\beta} \right)^{3N/2}\left( \frac{1}{\beta mg} \right)^{N}\left( e^{-\beta mgz}- e^{-\beta mg(z+dz)} \right)^{N}
        \end{align*}

        %\textit{Should we do something to simplify the difference in exponentials?}

    \end{callout}
\end{homeworkProblem}

\newpage
\begin{homeworkProblem}
    5.5: A Van der Waals gas is allowed to freely expand into a vacuum under adiabatic conditions. What is the change in the temperature of the gas following the expansion? Express your answer as a relationship between \( dT \) and \( dV \).
%    \begin{callout}{Derivation steps for heat capacity at constant pressure:}
%        \textit{I elect to put this in a separate block since it's basically just notes.} \vspace{1em}
%
%        For the sake of review, I'll walk through part of the derivation for heat capacity at constant pressure, so that we can arrive at heat in terms of some relevant variables:
%        %Walking through the derivation of heat capacity at constant pressure again, we can arrive at an expression for heat in terms of some relevant variables:
%        \begin{align*}
%            dE = T dS - P dV &= \left( \frac{\partial E}{\partial V} \right)_{T} dV + \left( \frac{\partial E}{\partial T} \right)_{V}dT \\ 
%        \end{align*}
%        We can express $dS$:
%        \begin{gather*}
%            dS = \frac{1}{T} \left( \frac{\partial E}{\partial T} \right)_{V} dT + \frac{1}{T} \left[ P + \left( \frac{\partial E}{\partial V} \right)_{T} \right] dV \\ 
%            \left( \frac{\partial S}{\partial T} \right)_{V} = \frac{1}{T} \left( \frac{\partial E}{\partial T} \right)_{V}, \qquad \left( \frac{\partial S}{\partial V} \right)_{T} = \frac{1}{T} \left[ P + \left( \frac{\partial E}{\partial V} \right)_{T} \right] \\ 
%            \frac{\partial ^{2} S}{\partial V \partial T} = \frac{\partial ^{2} S}{\partial T \partial V}\quad \to \quad \frac{\partial }{\partial V} \left( \frac{1}{T} \left( \frac{\partial E}{\partial T} \right)_{V} \right) = \frac{\partial }{\partial T} \left(\frac{1}{T}\left[  P+\left( \frac{\partial E}{\partial V} \right)_{T} \right]\right) \\ 
%            \frac{1}{T} \frac{\partial ^{2}E}{\partial V \partial T} = -\frac{1}{T^2} \left[ P+\left( \frac{\partial E}{\partial V} \right)_{T} \right] + \frac{1}{T} \left[ \left( \frac{\partial P}{\partial V} \right)_{V} + \left( \frac{\partial^2 E}{\partial T \partial V} \right)_{T} \right]
%        \end{gather*}
%        And since internal energy is also an exact differential, its cross partial derivatives must be equal
%        \begin{gather*}
%            \frac{\partial ^{2}E}{\partial V \partial T} = \frac{\partial ^2 E}{\partial T\partial V} \quad \to \quad 0 = - \frac{1}{T^2} \left[ P + \left( \frac{\partial E}{\partial V} \right)_{T} \right] + \frac{1}{T} \left( \frac{\partial P}{\partial T} \right)_{V} \\ 
%            P + \left( \frac{\partial E}{\partial V} \right)_{T} = T \left( \frac{\partial P}{\partial T} \right)_{V} \\ 
%        \end{gather*}
%    \end{callout}
%    \begin{callout}{Solution:}
%
%        Using the above derivation, we can go back to the previously derived equation for $dS$.
%        \begin{gather*}
%            dS = \frac{1}{T} \left( \frac{\partial E}{\partial T} \right)_{V} dT + \frac{1}{T} \left[ P + \left( \frac{\partial E}{\partial V} \right)_{T} \right]dV \\ 
%            TdS = \left( \frac{\partial E}{\partial T} \right)_{V} dT + \left[ P + \left( \frac{\partial E}{\partial V} \right)_{T} \right]dV \\ 
%            \bar{d}Q = C_V dT + \left[P + \left( \frac{\partial E}{\partial V} \right)_{T}\right]dV \\
%            \underbrace{\bar{d}Q}_{=0} = C_V dT + \underbrace{P + \left( \frac{\partial E}{\partial V} \right)_{T}}_{=T\left( \frac{\partial P}{\partial T} \right)_{V}=0} dV \\
%            0 = C_VdT
%        \end{gather*}
%        Adibatic conditions implies that both heat capacity and pressure are zero. As a consequence the pressure differential is also zero. This implies $dT$ must be zero because $C_V$ cannot be zero. Therefore the temperature is constant!
%
%    \end{callout}

    \begin{callout}{Solution Attempt \#2:}

        But for an adiabatic system, $Q=0$, so we can use the first law of thermodynamics to say:
        $$dE = -PdV$$
        And if I have a way to express $dE$, I could work out an expression for $dV/dT$. 
        In working with heat capacity, we found:
        $$dE = T dS - P dV = \left( \frac{\partial E}{\partial V} \right)_{T} dV + \left( \frac{\partial E}{\partial T} \right)_{V}dT $$
        So,
        \begin{align*}
            \left( \frac{\partial E}{\partial V} \right)_{T} dV + \left( \frac{\partial E}{\partial T} \right)_{V}dT &= -PdV \\ 
            \left( \frac{\partial E}{\partial T} \right)_{V} \frac{dT}{dV} &= - \left[ \left( \frac{\partial E}{\partial V} \right)_{T} + P \right] \\ 
            \frac{dV}{dT} &= - \frac{\left( \frac{\partial E}{\partial T} \right)_{V}}{\left( \frac{\partial E}{\partial V} \right)_{T} + P}
        \end{align*}
        Looking through my notes again, I realize that:
        $$\left( \frac{\partial E}{\partial T} \right)_{V} = C_V, \qquad \left( \frac{\partial E}{\partial V} \right)_{T} + P = T \left( \frac{\partial P}{\partial T} \right)_{V}$$
        So, 
        $$\frac{dV}{dT} = - \frac{C_V}{T\left( \frac{\partial P}{\partial T} \right)_{V}}$$
        Earlier in problem \#3 I used the equation of state for a VDW gas to find 
        $$\left(\frac{\partial P}{\partial T} \right)_{V} = \frac{\partial}{\partial T}\left( \frac{Nk_BT}{V-bN} - a \frac{N^2}{V^2} \right) = \frac{Nk_B}{V-bN} $$
        Which makes the expression $dV/dT$:
        $$\frac{dV}{dT} = - \frac{C_V}{T \left( \frac{Nk_B}{V-bN} \right)}$$
        Which can be simplified using $C_V= \frac{3Nk_B}{2}$ and a touch of algebra:
        $$\boxed{\frac{dV}{dT} = - \frac{(3Nk_B)(V-bN)}{TNk_B}}$$

    \end{callout}
\end{homeworkProblem}

\newpage
\begin{homeworkProblem}
    5.6: Show that for a gas in \( n \) dimensions, whose single-particle energy is described by \( \epsilon = p^\alpha \), that \( \frac{C_P}{C_V} = 1 + \frac{\alpha}{n} \).
    \begin{callout}{Solution:}
        \begin{align*}
            Z_T &= \int \dots \int \exp\left( -\beta p^a \right) ~d^np \\ 
            &= \int \dots \int \exp\left( -\beta p^a \right) ~p^{n-1}dp \int d \Omega n \\ 
            E &= - \frac{\partial }{\partial \beta } \ln(Z) \\
            &= \frac{\int_{0}^{\infty} p^{n-1+a} \exp\left( -\beta p^a \right) ~dp}{\int_{0}^{\infty} p^{n-1}\exp(-\beta p^a) ~dp}
        \end{align*}
        Transforming to n-D spherical coordinates gives $p^{n-1}dp$ times the jacobian $(d \Omega n)$. This luckily vanishes when we go from the partition function to energy thanks to it being in the numerator and denominator. If we set $u=-\beta p^\alpha$, this is a ratio of gamma functions, where $\Gamma\left( \frac{n}{a} \right) = \left( \frac{1}{\beta} \right) × \frac{\Gamma\left( \frac{n+a}{a} \right) }{\Gamma \left( \frac{n}{a} \right)}$. Then,
        $$E=k_BT \frac{n}{\alpha}, \quad E_N = Nk_BT \frac{n}{a}$$
        \begin{align*}
            C_V &= \left( \frac{\partial E}{\partial T} \right)_{V} = \frac{nNk_B}{\alpha } \\ 
            C_P &= C_V + T \left( \frac{\partial P}{\partial T} \right)_{V} \left( \frac{\partial V}{\partial T} \right)_{P}
        \end{align*}
        Since there is no specified potential, this should be an ideal gas. In this case, we have already derived the right hand side of the sum for $C_P$,
        \begin{align*}
            C_P &= C_V + \left( \frac{Nk_BT}{PV} \right)Nk_BT \\ 
            &= C_V + Nk_BT \\ 
            &= \frac{nNk_B}{\alpha} + Nk_BT 
        \end{align*}
        Then we have the ratio:
        \begin{align*}
            \frac{C_P}{C_V} &= \frac{\frac{nNk_B}{T} + Nk_BT}{\frac{nNk_B}{\alpha}} \\ 
            &= 1+ \frac{\alpha}{n}
        \end{align*}

    \end{callout}
    \begin{callout}{Equation of State / Explicit Solution:}
        
        Halfway in, I use our ideal gas solution, but the solution can be explicitly determined using the parition function to (1) find equation of state, then (2) taking partial derivatives:
        \begin{enumerate}[(1)]
            \item Equation of state:

                As with the ideal gas, we need a partition function over space to get volume. Since there is no potential it is just $Z_U=V^N$.
                \begin{align*}
                    S &= \ln (Z_TZ_U) + \beta \bar{E}  \\ 
                    &=  \ln[\Gamma(n/a)] - (n/a)\ln(\beta) +  \ln\left(\int d\Omega_n\right) + \ln(V^N) + N \beta \frac{n}{a}
                \end{align*}
                Now, 
                \begin{align*}
                    \beta P &= \left( \frac{\partial S}{\partial V} \right)_{E,N} \\
                    \beta P &= \frac{N}{V}
                \end{align*}

            \item Partial derivatives \& heat capacity: 
                \begin{align*}
                    \left( \frac{\partial P}{\partial T} \right)_{V} &= k_B \frac{N}{V} \\ 
                    \left( \frac{\partial V}{\partial T} \right)_{P} &= k_B \frac{N}{P} \\ 
                    C_P &= C_V + Tk_B^2\left( \frac{N}{V} \right) \left( \frac{N}{P} \right) \\ 
                    &= \frac{nNk_B}{a} + Nk_BT &&(\text{since }VP = N/\beta)
                \end{align*}
                Which matches the solution from before.
        \end{enumerate}

    \end{callout}
\end{homeworkProblem}
