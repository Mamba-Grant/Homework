\documentclass{article}


\newcommand{\hmwkTitle}{Homework \#4}
\newcommand{\hmwkDueDate}{\today}
\newcommand{\hmwkClass}{MATH 526}
\newcommand{\hmwkAuthorName}{\textbf{Grant Saggars}}



\usepackage{fancyhdr}
\usepackage{extramarks}
\usepackage{amsmath}
\usepackage{amsthm}
\usepackage{amsfonts}
\usepackage{tikz}

\usepackage{float}
\usepackage{caption}
\usepackage{bbold}
\usepackage{xcolor}
\usepackage{framed}
\usepackage{enumerate}
\usepackage{cancel}
\usepackage{multicol}
\usepackage{XCharter}

\usetikzlibrary{automata,positioning}

\usepackage{geometry}
\geometry{top=1in, bottom=1in, left=1in, right=1in} % Adjust margins as needed

\pagestyle{fancy}
\lhead{\hmwkAuthorName}
\chead{\hmwkClass\: \hmwkTitle}
\rhead{\firstxmark}
\lfoot{\lastxmark}
\cfoot{\thepage}

%
% Basic Document Settings
%

\topmargin=-0.75in
\evensidemargin=0in
\oddsidemargin=0in
\textwidth=6.5in
\textheight=9.0in
\headsep=0.25in

\linespread{1.1}

\renewcommand\headrulewidth{0.4pt}
\renewcommand\footrulewidth{0.4pt}

\setlength\parindent{0pt}

%
% Create Problem Sections
%

\newcommand{\enterProblemHeader}[1]{
    \nobreak\extramarks{}{Problem \arabic{#1} continued on next page\ldots}\nobreak{}
    \nobreak\extramarks{Problem \arabic{#1} (continued)}{Problem \arabic{#1} continued on next page\ldots}\nobreak{}
}

\newcommand{\exitProblemHeader}[1]{
    \nobreak\extramarks{Problem \arabic{#1} (continued)}{Problem \arabic{#1} continued on next page\ldots}\nobreak{}
    \stepcounter{#1}
    \nobreak\extramarks{Problem \arabic{#1}}{}\nobreak{}
}

\setcounter{secnumdepth}{0}
\newcounter{partCounter}
\newcounter{homeworkProblemCounter}
\setcounter{homeworkProblemCounter}{1}
\nobreak\extramarks{Problem \arabic{homeworkProblemCounter}}{}\nobreak{}

%
% Homework Problem Environment
%
% This environment takes an optional argument. When given, it will adjust the
% problem counter. This is useful for when the problems given for your
% assignment aren't sequential. See the last 3 problems of this template for an
% example.
%
\newenvironment{homeworkProblem}[1][-1]{
    \ifnum#1>0
        \setcounter{homeworkProblemCounter}{#1}
    \fi
    \section{Problem \arabic{homeworkProblemCounter}}
    \setcounter{partCounter}{1}
    \enterProblemHeader{homeworkProblemCounter}
}{
    \exitProblemHeader{homeworkProblemCounter}
}

%
% Callout Box
%

\definecolor{shadecolor}{RGB}{235,235,235}
\newenvironment{callout}[1] {\begin{shaded*} \textbf{#1}} {\end{shaded*}}

%
% Title Page
%

\title{
    \textmd{\textbf{\hmwkClass:\ \hmwkTitle}}\\
    \normalsize\vspace{0.1in}\small{\hmwkDueDate}\\
}

\author{\hmwkAuthorName}
\date{}

\renewcommand{\part}[1]{\textbf{\large Part \Alph{partCounter}}\stepcounter{partCounter}\\}





\begin{document}

\maketitle

\begin{homeworkProblem}[4]
	A coin is biased such that a head is three times as likely to occur as a tail. Find the expected number of tails when this coin is tossed twice.
	\begin{callout}{Solution:}

		Let $3P(H) = P(T)$, and $P(H)+P(T)=1$. We can solve for the probability of each event:

		$$ \left(\begin{array}{cc|c} 3 & -1 & 0 \\ 1 & 1 & 1 \end{array}\right) \implies P(H) = 3/4, ~P(T) = 1/4 $$

		The expectation asked for in this problem is given by the linear rule for expectation values: $E(X) = E_1(X) + E_2(X)$.

		\begin{align*}
			E_1(X) & = 1\times \frac{1}{4} + 0\times \frac{3}{4} = 1 \implies E_1(X) = \frac{1}{4} \\
			E_2(X) & = 1\times \frac{1}{4} + 0\times \frac{3}{4} = 1 \implies E_2(X) = \frac{1}{4} \\
			E(X)   & = \frac{1}{4} + \frac{1}{4} = \frac{1}{2}
		\end{align*}

	\end{callout}
\end{homeworkProblem}

\begin{homeworkProblem}[10]
	Two tire-quality experts examine stacks of tires and assign a quality rating to each tire on a 3-point scale. Let $X$ denote the rating given by expert A and $Y$ denote the rating given by B. The following table gives the joint distribution for $X$ and $Y$:
	\[
		\begin{array}{c|ccc}
			    & y=1  & y=2  & y=3  \\
			\hline
			x=1 & 0.10 & 0.05 & 0.02 \\
			x=2 & 0.10 & 0.35 & 0.05 \\
			x=3 & 0.03 & 0.10 & 0.20 \\
		\end{array}
	\]
	Find $\mu_X$ and $\mu_Y$.
	\newpage \begin{callout}{Solution:}

		\begin{enumerate}[(I)]
			\item Marginal probabilities (the sum of each element in each desired row/column):
			      \begin{align*}
				      P_x(1) & = (0.10)+(0.05)+(0.02) = 0.17 \\
				      P_x(2) & = (0.10)+(0.35)+(0.05) = 0.50 \\
				      P_x(3) & = (0.03)+(0.10)+(0.20) = 0.33
			      \end{align*}
			      Similarly,
			      \begin{align*}
				      P_y(1) & = (0.10)+(0.10)+(0.03) = 0.23 \\
				      P_y(2) & = (0.05)+(0.35)+(0.10) = 0.50 \\
				      P_y(3) & = (0.02)+(0.05)+(0.20) = 0.27
			      \end{align*}
			\item The expectation value for a discrete PDF is $\sum x_nP(x_n)$:
			      \begin{align*}
				      \mu_x & = 1(0.17)+2(0.50)+3(0.33) = 2.16 \\
				      \mu_y & = 1(0.23)+2(0.50)+3(0.27) = 2.04
			      \end{align*}
		\end{enumerate}

	\end{callout}
\end{homeworkProblem}

\begin{homeworkProblem}[12]
	If a dealer’s profit, in units of \$5000, on a new automobile can be looked upon as a random variable $X$ having the density function $f(x) = 2(1 - x)$, $0 < x < 1$, $0$, elsewhere, find the average profit per automobile.
	\begin{callout}{Solution:}

		The mean for a continuous PDF is $\int_{-\infty}^{\infty} xP(x) ~dx$
		\begin{align*}
			\mu_x & = \int_{0}^{1} 2x-2x^2 ~dx \\
			      & = \frac{1}{3}
		\end{align*}
		Putting this in terms of dollars gives $\$1667$.

	\end{callout}
\end{homeworkProblem}

\begin{homeworkProblem}[20]

	A continuous random variable $X$ has the density function
	\[
		f(x) =
		\begin{cases}
			e^{-x}, & \text{if } x > 0, \\
			0,      & \text{elsewhere.}
		\end{cases}
	\]
	Find the expected value of $g(X) = e^{\frac{2X}{3}}$
	\newpage \begin{callout}{Solution:}

		\begin{align*}
			E(g(X)) & = \int_{-\infty}^{\infty} g(X)f(x) ~dx = \int_{0}^{\infty} \exp\left( -\frac{x}{3} \right) ~dx = 3
		\end{align*}

	\end{callout}
\end{homeworkProblem}

\begin{homeworkProblem}[34]
	Let $X$ be a random variable with the following probability distribution:
	\[
		\begin{array}{c|ccc}
			x    & -2  & 3   & 5   \\
			\hline
			f(x) & 0.3 & 0.2 & 0.5 \\
		\end{array}
	\]
	Find the standard deviation of $X$.
	\begin{callout}{Solution:}

		\begin{align*}
			\sigma ^{2} & = \sum P(x)(x-\overline{x})^{2} = 0.3(-2-2)^{2} + 0.2(3-2)^{2} + 0.5(5-2)^{2} \\
			            & = 1.127                                                                       \\
			\sigma      & = 1.062
		\end{align*}

	\end{callout}
\end{homeworkProblem}

\begin{homeworkProblem}[36]
	Suppose that the probabilities are 0.4, 0.3, 0.2, and 0.1, respectively, that 0, 1, 2, or 3 power failures will strike a certain subdivision in any given year. Find the mean and variance of the random variable $X$ representing the number of power failures striking this subdivision.
	\begin{callout}{Solution:}

		\begin{align*}
			\overline{x} & = \frac{1}{4}(0+1+2+3) = 1.5                                                                         \\
			\sigma^{2}   & = \sum f(x) (x-\overline{x})^{2} = 0.4(0-1.5)^{2} + 0.3(1-1.5)^{2} + 0.2(2-1.5)^{2} + 0.1(3-1.5)^{2} \\
			             & = 1.25                                                                                               \\
			\sigma       & = 1.12
		\end{align*}

	\end{callout}
\end{homeworkProblem}

\begin{homeworkProblem}[39]
	The total number of hours, in units of 100 hours, that a family runs a vacuum cleaner over a period of one year is a random variable $X$ having the density function given in Exercise 4.13 on page 117. Find the variance of $X$.
	\begin{callout}{Solution:}

		\[
			f(x) = \begin{cases}
				x,     & 0 < x < 1,        \\
				2 - x, & 1 \leq x < 2,     \\
				0,     & \text{elsewhere}.
			\end{cases}
		\]

		\begin{align*}
			\langle x \rangle   & = \int_{0}^{1} x^2 ~dx + \int_{1}^{2} 2x-x^2 ~dx = 1                          \\
			\langle x^2 \rangle & = \int_{0}^{1} x^3 ~dx + \int_{1}^{2} 2x^2-x^3 ~dx = \frac{7}{6}              \\
			\sigma^2            & = \langle x^2 \rangle - \langle x \rangle^2 = \frac{7}{6} - 1^2 \approx 0.167
		\end{align*}

	\end{callout}
\end{homeworkProblem}

\begin{homeworkProblem}[50]
	For a laboratory assignment, if the equipment is working, the density function of the observed outcome $X$ is $f(x) = 2(1 - x)$, $0 < x < 1$, $0$ otherwise. Find the variance and standard deviation of $X$.
	\begin{callout}{Solution:}

		\begin{align*}
			\langle x \rangle   & = \int_{0}^{1} 2x(1-x) ~dx = \frac{1}{3}   \\
			\langle x^2 \rangle & = \int_{0}^{1} 2x^2(1-x) ~dx = \frac{1}{6} \\
			\sigma^2            & = \frac{1}{6} - \frac{1}{9} \approx 0.0556 \\
			\sigma              & = 0.236
		\end{align*}

	\end{callout}
\end{homeworkProblem}

\begin{homeworkProblem}[52]
	Random variables $X$ and $Y$ follow a joint distribution $f(x, y) = 2$, $0 < x \leq y < 1$, $0$ otherwise. Determine the correlation coefficient between $X$ and $Y$.
	\newpage \begin{callout}{Solution:}

		\begin{enumerate}[(I)]
			\item Calculating Covariance:
			      $$ \textrm{Cov}(X,Y) = \langle XY \rangle - \langle X \rangle \langle Y \rangle $$
			      \begin{align*}
				      \langle XY \rangle         & = \int_{0}^{1} \int_{x}^{1} 2xy ~dy~dx = \frac{1}{4} \\
				      \langle X \rangle          & = \int_{0}^{1} \int_{x}^{1} 2x ~dy ~dx = \frac{1}{3} \\
				      \langle Y \rangle          & = \int_{0}^{1} \int_{x}^{1} 2y ~dy ~dx = \frac{2}{3} \\
				      \implies \textrm{Cov}(X,Y) & = \frac{1}{4} - \frac{2}{9} = \frac{1}{36}
			      \end{align*}
			\item Calculating Correlation:
			      $$ \rho = \frac{\textrm{Cov}(X,Y)}{\sqrt{ \textrm{Var(X)}\textrm{Var(Y)} }} $$

			      \begin{align*}
				      \langle X^2 \rangle & = \int_{0}^{1} \int_{x}^{1}  2x^2  ~dy ~dx = \frac{1}{6} \\
				      \langle Y^2 \rangle & = \int_{0}^{1} \int_{x}^{1}  2y^2  ~dy ~dx = \frac{1}{2}
			      \end{align*}

			      $$ \rho = \frac{\frac{1}{36}}{\sqrt{ (1/6 - 1/9)(1/2-4/9) }} = \frac{1}{2} $$
		\end{enumerate}

	\end{callout}
\end{homeworkProblem}

\begin{homeworkProblem}[57]
	Let $X$ be a random variable with the following probability distribution:
	\[
		\begin{array}{c|ccc}
			x    & -3          & 6           & 9           \\
			\hline
			f(x) & \frac{1}{6} & \frac{1}{2} & \frac{1}{3} \\
		\end{array}
	\]
	Find $E(X)$ and $E(X^2)$ and then, using these values, evaluate $E[(2X + 1)^2]$.
	\begin{callout}{Solution:}

		$$ E(X) = -3 \frac{1}{6} + 6 \frac{1}{2} + 9 \frac{1}{3} = \frac{24}{3} $$
		$$ E(X^2) = 9 \frac{1}{6} + 36 \frac{1}{2} + 81 \frac{1}{3} = 127 $$
		$$ E[(2X+1)^{2}] = E(4X^2 + 4X + 1) = 4(127) + 4\left( \frac{24}{3} \right) + 1 = 541 $$

	\end{callout}
\end{homeworkProblem}

\begin{homeworkProblem}[62]
	If $X$ and $Y$ are independent random variables with variances $\sigma_X^2 = 5$ and $\sigma_Y^2 = 3$, find the variance of the random variable $Z = -2X + 4Y - 3$.
	\begin{callout}{Solution:}

		% The variance of a linear combination of variables is defined as:
		% $$ \textrm{Var}(aX+bY+c) = a^2\sigma_x + b^2\sigma_y + 2ab \textrm{Cov(X,Y)} $$
		Variance for independent variables has linear properties:
		$$ \textrm{Var}(-2X+4Y-3) = -2^2\textrm{Var}(X) + 4^2\textrm{Var}(y) + \textrm{Var(-3)} $$
		Which works out to:
		$$ 4(5) + 16(3) + 9(0) = 68 $$

	\end{callout}
\end{homeworkProblem}

\end{document}
