\documentclass[14pt]{extarticle}

\title{Homework 1}
\author{Grant Saggars}
\date{\today}

\usepackage{amsmath}
\usepackage{import}
\usepackage{pdfpages}
\usepackage{pgfplots}
\usepackage{transparent}
\usepackage{xcolor}
\usepackage{framed}
\usepackage{enumerate}
\usepackage{geometry}
\usepackage{cancel}
\usepackage{multicol}
\usepackage{lipsum}  
\usepackage{caption}
\usepackage{float}
\usepackage{bbold}
% \usepackage{fontspec}

% \setmainfont{BespokeSerif-Regular}
\definecolor{shadecolor}{RGB}{235,235,235}

\geometry{top=1in, bottom=1in, left=1in, right=1in}
\newenvironment{callout}[1] {\begin{shaded*} \textbf{#1}} {\end{shaded*}}

%%%%%%%%%%%%%%%%%%%%%%%%
% DOCUMENT BEGINS HERE %
%%%%%%%%%%%%%%%%%%%%%%%%

\begin{document}
\maketitle
\begin{enumerate}[1.]
	\item The following measurements were recorded for the drying time, in hours, of a certain brand of latex paint.
	      \begin{align*}
		      \text{3.4 2.5 4.8 2.9 3.6} \\
		      \text{2.8 3.3 5.6 3.7 2.8} \\
		      \text{4.4 4.0 5.2 3.0 4.8}
	      \end{align*}
	      Assume that the measurements are a simple random sample.

	      \begin{enumerate}[(a)]
		      \item What is the sample size for the above sample?
		            \begin{callout}{Solution:}

			            The sample size is 15, since there are 15 elements.
		            \end{callout}

		      \item Calculate the sample mean, median, sample variance and sample standard deviation.
		            \begin{callout}{Solution:}

			            (1) mean:
			            \begin{align*}
				            \begin{array}{c}
					            3.4+2.5+4.8+2.9+3.6+2.8+3.3+5.2 \\
					            +3.7+2.8+4.4+4.0+5.2+3.0+4.8
				            \end{array} / 15 = 3.787
			            \end{align*}

			            (2) median:
			            When sorted, the set has a middle value of 3.6.

			            \newpage
			            (3) variance:

			            Variance (for a discrete data set) is defined as:
			            \begin{align*}
				            \sigma ^{2} = \sum \frac{(x_i - \overline{x})^{2}}{n-1}
			            \end{align*}

			            Written out for this set, the variance is

			            \begin{align*}
				            \begin{array}{r}
					            (3.4 - 3.787)^{2}
					            + (2.5 - 3.787)^{2}
					            + (4.8 - 3.787)^{2} \\
					            + (2.9 - 3.787)^{2}
					            + (3.6 - 3.787)^{2}
					            + (3.3 - 3.787)^{2} \\
					            + (5.2 - 3.787)^{2}
					            + (3.7 - 3.787)^{2}
					            + (2.8 - 3.787)^{2} \\
					            + (4.4 - 3.787)^{2}
					            + (4.0 - 3.787)^{2}
					            + (5.2 - 3.787)^{2} \\
					            + (3.0 - 3.787)^{2}
					            + (4.8 - 3.787)^{2}
				            \end{array} = \frac{13.16 \dots}{14} = 0.94
			            \end{align*}


			            (4) standard deviation:

			            The standard deviation is the square root of the variance which equals 0.97.

		            \end{callout}

		            % \setcounter{enumi}{8}
		      \item Compute the 20\% trimmed mean for the above data set.
		            \begin{callout}{Solution:}

			            The 20\% trimmed mean is the mean of the dataset after trimming 20\% of the data points off of each end of the sorted dataset. Because the dataset is 14 elements in size, the 3 smallest and greatest values should be dropped:

			            $$3.0, 3.3, 3.4, 3.6, 3.7, 4.0, 4.4$$
		            \end{callout}

	      \end{enumerate}

	      \newpage
	      \setcounter{enumi}{13}
	\item A tire manufacturer wants to determine the inner diameter of a certain grade of tire. Ideally, the diameter would be 570 mm. The data are as follows:
	      $$\text{572, 572, 573, 568, 569, 575, 565, 570.}$$

	      \begin{enumerate}[(a)]
		      \item Find the sample mean and median. (I used my calculator)
		            \begin{callout}{Solution:}
			            \begin{enumerate}[(1)]
				            \item Mean: 570.5
				            \item Median: 571
			            \end{enumerate}
		            \end{callout}

		      \item Find the sample variance, standard deviation, and range.
		            \begin{callout}{Solution:}
			            \begin{enumerate}[(1)]
				            \item Variance: 8.75
				            \item Stdev: 2.95
				            \item Range: 10
			            \end{enumerate}
		            \end{callout}

		      \item Using the calculated statistics in parts (a) and (b), can you comment on the quality of the tires?
		            \begin{callout}{Solution:}

			            The given the mean and standard deviation, there are 6 tires within 1 standard deviation, and all tires are within 3 standard deviations, meaning there are no significant outliers from this set of tires.

		            \end{callout}

	      \end{enumerate}

	      \setcounter{enumi}{20}
	\item The lengths of power failures, in minutes, are recorded in the following table.
	      \begin{align*} \begin{array}{rrrrrrrrrr}
			      22 & 18 & 135 & 15  & 90  & 78 & 69 & 98  & 102 \\
			      83 & 55 & 28  & 121 & 120 & 13 & 22 & 124 & 112 \\
			      70 & 66 & 74  & 89  & 103 & 24 & 21 & 112 & 21  \\
			      40 & 98 & 87  & 132 & 115 & 21 & 28 & 43  & 37  \\
			      50 & 96 & 118 & 158 & 74  & 78 & 83 & 93  & 95
		      \end{array} \end{align*}

	      \begin{enumerate}[(a)]
		      \item Find the sample mean and sample standard deviation  using TI-83/84 calculator.

		            \begin{callout}{Solution:}

			            (1) Mean: 74.02

			            (2) Standard Deviation: 31.82

		            \end{callout}

		      \item Find the five number summary  using TI-83/84 calculator built-in function.
	      \end{enumerate}

	      \begin{callout}{Solution:}

		      A five-number summary includes the minimum, Q1, the median, Q3, and the maximum. For this dataset, this is: $$\{ 13, 32.5, 78, 102.5, 158 \}$$

	      \end{callout}

	\item The following data are the measures of the diameters of 36 rivet heads in 1/100 of an inch.

	      \begin{align*} \begin{array}{rrrrrrrrr}
			      6.72 & 6.77 & 6.82 & 6.70 & 6.78 & 6.70 & 6.62 & 6.75 \\
			      6.66 & 6.66 & 6.64 & 6.76 & 6.73 & 6.80 & 6.72 & 6.76 \\
			      6.76 & 6.68 & 6.66 & 6.62 & 6.72 & 6.76 & 6.70 & 6.78 \\
			      6.76 & 6.67 & 6.70 & 6.72 & 6.74 & 6.81 & 6.79 & 6.78 \\
			      6.66 & 6.76 & 6.76 & 6.72
		      \end{array} \end{align*}

	      \begin{enumerate}[(a)]
		      \item Calculate the sample mean, median, sample variance and sample standard deviation.
		            \begin{callout}{Solution:}

			            (1) Mean: 6.726

			            (2) Median: 6.725

			            (3) Sample Variance: 0.002870

			            (4) Stdev: 0.05357

		            \end{callout}
		      \item Construct a relative frequency histogram of the data.

		            \begin{callout}{Solution:}

			            \center
			            \includegraphics[width=0.7\textwidth]{histogram.png}

		            \end{callout}

		            \newpage
		            \setcounter{enumi}{23}
		      \item The following are historical data on staff salaries (dollars per pupil) for 30 schools sampled in the eastern part of the United States in the early 1970s.

		            \begin{align*}
			            \begin{array}{rrrrrrrrr}
				            3.79 & 2.99 & 2.77 & 2.91 & 3.10 & 1.84 & 2.52 & 3.22 \\
				            2.45 & 2.14 & 2.67 & 2.52 & 2.71 & 2.75 & 3.57 & 3.85 \\
				            3.36 & 2.05 & 2.89 & 2.83 & 3.13 & 2.44 & 2.10 & 3.71 \\
				            3.14 & 3.54 & 2.37 & 2.68 & 3.51 & 3.37
			            \end{array}
		            \end{align*}

		            \begin{enumerate}[(a)]
			            \item Find the sample mean and sample standard deviation using TI-83/84 calculator built-in function.
			                  \begin{callout}{Solution:}

				                  (1) Mean: 2.897

				                  (2) Stdev: 0.5415

			                  \end{callout}
			            \item Find the five number summary  using TI-83/84 calculator built-in function.

			                  \begin{callout}{Solution:}

				                  $$\{ 1.84, 2.52, 2.86, 3.325, 3.85 \}$$

			                  \end{callout}
		            \end{enumerate}
	      \end{enumerate}
\end{enumerate}
\end{document}
