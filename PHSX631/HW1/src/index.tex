\begin{homeworkProblem}
    Permanent cylindrical magnet with axial magnetization $M$.

    \begin{enumerate}[(a)]
        \item Do Griffith Problem 6.9. Sketch field lines.
        \item Go beyond Griffiths Problem 6.9 by finding the magnetic field as a function of distance, $z$, along the symmetry axis, and outside the magnet. Assume $L = 2a$. Plot/sketch $B(z)$, $H(z)$, $M(z)$ along the axis. Hint: See Example 5.6 in Griffiths.
    \end{enumerate}

    \textbf{(Problem 6.9)} A short circular cylinder of radius $a$ and length $L$ carries a "frozen-in" uniform magnetization $\mathbf{M}$ parallel to its axis. Find the bound current, and sketch the magnetic field of the cylinder. (Make three sketches: one for $L \gg a$, one for $L \ll a$, and one for $L \approx a$.) Compare this bar magnet with the bar electret of Prob. 4.11.
    \begin{callout}{Solution, part (a):}

        We have equations for bound current (volume and surface respectively):
        $$\mathbf{J_{b}}=\nabla \times \mathbf{M}=0, \qquad  \mathbf{K_{b}}=\mathbf{M}\times \mathbf{\hat{n}}=M\hat{\phi}$$
        Or equivalently, $\mathbf{K_b}=(-M \sin\phi, M \cos\phi, 0)$.
        
    \end{callout}

    \begin{callout}{Solution, part (b):}

        As we have bound current, we can use the Biot-Savart law: 
        $$\textbf{B}(\textbf{r}) = \frac{\mu_0}{4\pi} \int_{S} \frac{\mathbf{K_b}(\textbf{r})\times \hat{\boldscriptr}}{|\scriptr|^2} ~dA$$
        We have a point of interest at $(0,0,z)$. The points on the surface of the cylinder are given by $(a\cos\phi', a\sin\phi', z')$.
        This gives us $\scriptr = (-a\cos\phi', -a\sin\phi', z-z')$ and magnitude $|\scriptr| = (a^2 + (z-z')^2)^{1/2}$.
        Finally, the cross product is 
        $$\mathbf{K_b} \times \frac{\scriptr}{|\scriptr|} = \left|\begin{array}{ccc} \hat{\textbf{i}} & \hat{\textbf{j}} & \hat{\textbf{k}} \\ -M\sin\phi & M\cos\phi & 0 \\ \frac{-a\cos\phi}{(a^2 + (z-z')^2)^{1/2}} & \frac{-a\sin\phi}{(a^2 + (z-z')^2)^{1/2}} & \frac{z-z'}{(a^2 + (z-z')^2)^{1/2}} \end{array}\right|$$
            $$=M\cos\phi \left( \frac{z-z'}{(a^2 + (z-z')^2)^{1/2}} \right) \hat{\textbf{i}} + M\sin\phi \left( \frac{z-z'}{(a^2 + (z-z')^2)^{1/2}} \right) \hat{\textbf{j}} + \frac{Ma}{(a^2 + (z-z')^2)^{1/2}}\hat{\textbf{z}}$$
        The $x,y$ components of this term leave a sin and cos term in the surface integral. This will integrate to zero, as we'd expect. This means that we only need to worry about the $z$-component for the final result:
        \begin{align*}
            \textbf{B}(\textbf{r}) &= \frac{\mu_0}{4\pi} \int_{-a}^{a} \frac{Ma \hat{\textbf{z}}}{\left( a^2 + (z-z')^2 \right)^{3/2}} ~dz'
        \end{align*}
        If the rotation occurs at a constant angular velocity $\omega$, then $\theta = \omega t$

    \end{callout}
\end{homeworkProblem}

\newpage
\begin{homeworkProblem}
    Consider a square loop of wire with resistance $R$ and size $a$ by $a$. The surface normal is initially oriented parallel to a uniform magnetic field with magnitude $B_0$ . The loop is then rotated by 90 deg such that the normal vector is perpendicular to the magnetic field. How much charge passes through the circuit during this procedure?
   \begin{callout}{Solution:}

        By lenz's law, the sudden magnetic flux flowing through the loop will induce a clockwise current in the loop of wire. The current will reduce as it is rotated until at 90 degrees there is no more current, since there is no flux.

        We obviously need some measure of magnetic flux to work with, so I will compute that:
        $$\Phi = \int \textbf{B} \cdot d \textbf{a} = B_0 a^2 \cos\theta$$

        From here I can jump to the time derivative of magnetic flux (EMF), then to current using the resistance.
        The amount of charge moved depends on how quickly we're rotating the ring, so I will assume that the angle is changing at a constant speed $\omega$.
       \begin{align*}
           \varepsilon &= \frac{d\Phi}{dt} = B_0 a^2 \frac{d}{dt} \cos\theta \\
           I &= -\frac{B_0 a^2}{R} \frac{d\cos\theta}{dt} && \left(\frac{\varepsilon}{R} = I = \frac{dQ}{dt}\right) \\
           I &= -\frac{B_0 a^2}{R} \frac{d \cos (\omega t)}{dt} && \left( \text{assume $\theta=\omega t$} \right) \\
           I &= \frac{B_0 a^2}{R}  \omega \sin (\omega t) \\
       \end{align*}
      Total charge is then: 
       $$Q = \int_0^{\pi/(2\omega)} I dt = \frac{B_0 a^2}{R} (1 - \cos(\pi/2)) = \frac{B_0 a^2}{R}$$

   \end{callout}
\end{homeworkProblem}

\newpage
\begin{homeworkProblem}
    Do Griffiths problem 7.17

    \textbf{(Problem 7.17)} A long solenoid of radius $a$, carrying $n$ turns per unit length, is looped by a wire with resistance $R$, as shown in Fig. 7.28.

    \begin{figure}[h]
      \centering
      \includegraphics[width=0.4\textwidth]{../assets/H1P3F1.png}
    \end{figure}

    \begin{enumerate}[(a)]
       \item If the current in the solenoid is increasing at a constant rate (dI/dt = k), what current flows in the loop, and which way (left or right) does it pass through the resistor?
          \begin{callout}{Solution:}
          
              The goal is to find the direction of induced current, which using Lenz's law we know would be the direction that opposes changes in magnetic flux.
              We just need to know how the magnetic field changes (decreasing or increasing) and in which direction. 
              We are told that it is increasing, and the right hand rule tells us that it is pointed towards the left. 
              The current must oppose this so it is moving to the right (counterclockwise).


          \end{callout}
       \item If the current I in the solenoid is constant but the solenoid is pulled out of the loop (toward the left, to a place far from the loop), what total charge passes through the resistor?
          \begin{callout}{Solution:}
          
              Same approach as before, use the current to get charge.
              For a solenoid, $B = \mu_0nI$, so $\phi = (\pi a^2) \mu_0nI$. 
              $$\varepsilon = - \frac{\partial \Phi_B}{dt} = -(\pi a^2) \mu_0 n \left( \frac{dI}{dt} \right)$$
              Current is then given by Ohm's law
              $$\frac{dQ}{dt} = I = -\frac{(\pi a^2) \mu_0 n \left( \frac{dI}{dt} \right)}{R}$$
              So we would just integrate this over time.

          \end{callout}
   \end{enumerate}

\end{homeworkProblem}

\newpage
\begin{homeworkProblem}
    A very long cylindrical sheet of metal with radius $r$ and length $L$ carries a current $K$ per unit length (azimuthal current) (units of A/m). What is the energy stored in the magnetic field in this cylinder in terms of $L$, $R$, and $K$?
   \begin{callout}{Solution:}
   
       We have here some current inducing a magnetic field, which we know to have energy:
       $$W_{mag}= \frac{1}{2} \int \left( \textbf{A} \cdot \textbf{J} \right)d\tau = \frac{\epsilon_0}{2}\int B^2 d\tau$$
       Where magnetic vector potential is:
       $$\textbf{A(r)} = \frac{\mu_0}{4\pi} \int \frac{\textbf{J}(\textbf{r'})}{\scriptr}d\tau'$$
       It would probably be easiest to apply ampere's law to determine $B$ to get to our solution.
       Inside, there is no field below the sheet, since there is no current enclosed. 
       Outside there is a magnetic field:
       \begin{align*}
           B(2 \pi s) = \mu_0 K \hat{z}
       \end{align*}
       And we can square it and integrate to get the final result (we can account for length in this integration)
      \begin{align*}
      W_{mag} &= \frac{1}{2}\int_{a}^{b}\left( \frac{\mu_0 K}{2\pi s}\hat{z} \right)^2s\,ds \int_{0}^L dL \int_0^{2\pi} d\phi \\ 
          &= \frac{\mu_0 K^2L}{4\pi} \ln \left( \frac{b}{a} \right) \hat{z}
      \end{align*}
       

   \end{callout}
\end{homeworkProblem}
