\begin{homeworkProblem}
    The time-dependent Schr\"odinger equation in 1D is:

    \begin{equation}
        i\hbar \frac{\partial \Psi}{\partial t} = - \frac{\hbar^2}{2m} \frac{\partial^2 \Psi}{\partial x^2} + V \Psi.
    \end{equation}

    Assume plane wave solutions in free space ($V=0$) of the form:
    \begin{equation}
        \Psi(x,t) = \Psi_0 \exp\{i(kx - \omega t)\}.
    \end{equation}
    where $\omega(k)$ is a function of $k$; that is, the dispersion relation.

    \begin{enumerate}
        \item[(a)] Find the dispersion relation $\omega(k)$.
            \begin{callout}{Solution:}

                The dispersion relation is just obtained by getting the full wave solution, and solving for $\omega$ algebraically. We're given a \textit{particular} solution, which we can take to the full solution by differentiating and substituting.
                \begin{align*}
                    \frac{\partial \Psi(x,t)}{\partial t} &= -i\omega \Psi_{0} e^{i(kx-\omega t)} = -i \omega \Psi \\
                    \frac{\partial^{2} \Psi(x,t)}{\partial x^2} &= -\left( k \right)^2\Psi_{0} e^{i(kx-\omega t)} = -k^{2} \Psi
                \end{align*}
                Together, these imply
                \begin{align*}
                    i \hbar \left( -i \omega \Psi \right) &= - \frac{\hbar^{2}}{2m} \left( -k^{2} \Psi \right) + V \Psi \\ 
                    \hbar \omega \cancel{\Psi} &= \frac{\hbar^{2}}{2m} \left( k^{2} \cancel{\Psi} \right) + V \cancel{\Psi} \\ 
                    \omega(k) &= \frac{k^{2} \hbar}{2m} + \frac{V}{\hbar} \\ 
                \end{align*}
                Of course, the $V$ term vanishes in free space.

            \end{callout}
        \newpage \item[(b)] From $\omega(k)$, find the phase and group velocities of the wave.
            \begin{callout}{Solution:}

                We have two definitions of velocity: phase and group velocity, which are respectively:
                \begin{align*}
                    v &= \frac{\omega}{k} = \frac{ \hbar k}{2m} + \frac{V}{\hbar k} \\
                    v_g &= \frac{d\omega}{dk} = \frac{ \hbar k}{m} \\
                \end{align*}
                Of course, the $V$ term vanishes in free space.

            \end{callout}
        \item[(c)] Assuming this wavefunction represents a free particle, what is its speed? What is its momentum?
            \begin{callout}{Solution:}

                Classically, $p=mv$, and the group velocity term corresponds to the classical velocity. I will suggest the de Broglie relationship holds, such that: 
                $$p = \hbar k = \frac{mv}{\hbar}$$

            \end{callout}
        \item[(d)] Now assume that the particle can be represented by a wavepacket. The Fourier transform of the wave function is given by 
            \begin{equation}
                A(k) = \Delta k^{-1} \operatorname{rect}(k_0 - \Delta k),
            \end{equation}
            where $\Delta k \ll k_0$. This is a square function. Find $\Psi(x,t)$ for this case. Describe it. What are the uncertainties in momentum and space (i.e., $x$)? That is, does Heisenberg’s uncertainty principle hold?
            \begin{callout}{Solution:}

            The inversion theorem for Fourier transformations states:
                $$\phi(z) = \int_{-\infty}^{\infty} \mathbf{\Phi}(k) e^{ikz} \,dk \leftrightarrow \mathbf{\Phi}(k) = \frac{1}{2\pi} \int_{-\infty}^{\infty} \phi(z) e^{-ikz} \,dz$$
                Where in this case, $A(k)=\mathbf{\Phi}(k)$, and $\Psi(x)$ will become:
                $$\Psi(x, t=0) = \int_{-\infty}^{\infty} \frac{1}{\Delta k} \text{rect}(k_{0}-\Delta k)e^{ikz} \,dk$$
                Wikipedia tells me the fourier transformation of the rectangular function is a sinc function, given by the following integral:
                $$\Psi(x, t=0) = \frac{1}{\Delta k} \int_{k_{0} + \Delta k / 2}^{k_{0} - \Delta k / 2} e^{ikx} \,dk = \frac{ 2 \sin(\Delta k x / 2)}{\Delta k} \Delta k e^{ik_{0} x} = \Delta k e^{i k_{0} x} \text{sinc}(\Delta k x / 2)$$
                Cracking open my quantum mechanics book, the propagator in the x-basis is 
                $$U=\left( \frac{m}{2\pi \hbar i t} \right)^{\frac{1}{2}} e^{im (x-x')^{2}/2\hbar t}$$
                and the time-dependent wave function is given as the integral
                $$\Psi(x,t) = \int_{-\infty}^{\infty} U'(x,t; x', t') \Psi(x, t=0) \,dx'$$
                Stopping for a moment, I think it would be good to look at the momentum and space distribution and uncertainty. 
                The Fourier transform of the wave function represents the momentum space, and it is possible to directly calculate $\Delta p$ as 
                $A(k)$ is not normalized, so the momentum space wave function is instead $\int_{-\infty}^{\infty} A^2(k) \,dk = 1 \implies A(k) \to \frac{1}{\sqrt{\Delta k}} \text{rect}(k_{0}-\Delta k)$.
                \begin{align*}
                    \langle p \rangle &= \hbar \braket{A(k)|k|A(k)} \\ 
                    &= \hbar \int_{-\infty}^{\infty} k A(k)A(k) \,dk \\
                    &= \frac{\hbar}{\Delta k}\int_{k_{0} - \Delta k / 2}^{k_{0} + \Delta k / 2} k \,dk \\
                    &= \frac{\hbar}{\Delta k} \left[ \frac{1}{2} (k_{0} + \Delta k / 2)^{2} - \frac{1}{2} (k_{0} - \Delta k / 2)^{2} \right] \\
                    &= \frac{\hbar}{2 \Delta k} \left[ \cancel{k_{0}^{2}} + k_{0}\Delta k + \cancel{\Delta k^{2}} - \cancel{k_{0}^{2}} + k_{0}\Delta k - \cancel{\Delta k^{2}} \right] \\
                    &= \frac{\hbar k_{0} \Delta k}{\Delta k} \\
                    &= \hbar k_{0}
                \end{align*}
                A similar integral gives $\langle p^{2} \rangle$
                \begin{align*}
                    \langle p^{2} \rangle &= \frac{\hbar^2}{\Delta k} \int_{k_{0}-\Delta k / 2}^{k_{0}+\Delta k / 2} k^{2} \,dk \\ 
                    &= \frac{\hbar^2}{3 \Delta k} \left( k_{0} + \Delta k / 2\right)^{3} - \left( k_{0} - \Delta k / 2 \right)^{3} \\
                    &= \frac{\hbar^2}{3 \Delta k} \frac{\Delta k\left(12k_{0}^2+ \Delta k^2\right)}{4} \\ 
                    &= \hbar^2 k_{0}^2+ \hbar^2 \frac{1}{12}\Delta k^2
                \end{align*}
                uncertainty is therefore
                \begin{align*}
                    \Delta p ^{2} = \left( \hbar^2 k_{0}^2 + \frac{\hbar^2}{12}\Delta k^2 \right) - \left( \hbar k_{0} \right)^{2} = \frac{\hbar^2}{12} \Delta k^{2}
                \end{align*}

                When doing these calculations, the complex conjugate thankfully eliminates the exponential. By symmetry, $\langle x \rangle=0$ for the sinc function. Before doing the other, it must be normalized such that 
                \begin{align*}
                    1 &= A \int_{-\infty}^{\infty} \Delta k \text{sinc}^{2}(x \Delta k / 2) \,dx \\ 
                    \implies A &= \frac{1}{\sqrt{2\pi \Delta k}}
                \end{align*}
                Now, $\langle x^{2} \rangle$ is given by:
                \begin{align*}
                    \frac{\left( \Delta k \right) ^{2}}{{2\pi \Delta k}} \int_{-\infty}^{\infty} x^{2} \text{sinc}^{2}(x \Delta k / 2) \,dx \\
                    \frac{\Delta k}{2\pi} \frac{\pi}{2(\Delta k / 2)^{3}} = \frac{4}{(\Delta k)^{2}}
                \end{align*}
                Finally this gives uncertainty:
                $$(\Delta x)^{2} = \frac{4}{\left( \Delta k \right)^{2}} - 0^{2} \qquad \implies \qquad \Delta x = \frac{2}{\Delta k}$$

                Now comparing with the uncertainty principle:
                \begin{align*}
                    \Delta p \Delta x &\stackrel{?}{\geq} \frac{\hbar}{2} \\ 
                    \left( \frac{\hbar \cdot \Delta k}{\sqrt{12}} \right) \left( \frac{2}{\Delta k} \right) = \frac{2\hbar}{\sqrt{12}} &\geq \frac{\hbar}{2}
                \end{align*}


            \end{callout}
    \end{enumerate}
\end{homeworkProblem}

\begin{homeworkProblem}
    \textbf{Griffiths Problem 10.15} A particle of charge $q$ moves in a circle of radius $a$ at constant angular velocity $\omega$. (Assume that the circle lies in the $xy$ plane, centered at the origin, and at time $t = 0$ the charge is at $(a,0)$, on the positive $x$ axis.) Find the Liénard-Wiechert potentials for points on the $z$ axis.
    \begin{callout}{Solution:}

        The goal is to apply equations 10.46 and 10.47:
        \begin{align*}
            V(\mathbf{r},t)&=\frac{1}{4\pi\epsilon_{0}} \frac{qc}{\rcurs c-\bcurs \cdot \mathbf{v}} \tag{10.46} \\
            \mathbf{A}(\mathbf{r},t)&= \frac{\mu_{0}}{4\pi} \frac{qc \mathbf{v}}{\rcurs c-\bcurs \cdot \mathbf{v}} = \frac{\mathbf{v}}{c^{2}}V(\mathbf{r},t) \tag{10.47}
        \end{align*}

        We know $\mathbf{w}(t) = \omega \mathbf{v}$, so step 1 of our algorithm is to define the path and velocity of the particle on the path so that I can take the difference between this and $z$:
        $$\bcurs = z \mathbf{\hat{z}} - a \left[ \cos \left( \omega t_r \right)\mathbf{\hat{x}} + \sin \left( \omega t_r \right)\mathbf{\hat{y}} \right], \qquad \rcurs^2 = a^2 + z^2$$
        $$\mathbf{v} = \omega \mathbf{w}(t) = a \omega \left[ -\sin \left( \omega t_r \right)\mathbf{\hat{x}} + \cos \left( \omega t_r \right)\mathbf{\hat{y}} \right]$$
        For all $\theta \in [0,2\pi)$. With the above information calculated, doing the dot products is very straightforward
        $$\bcurs \cdot \mathbf{v} = \omega a^2 \left[ \cos \left( \omega t_r \right)\mathbf{\hat{x}} + \sin \left( \omega t_r \right)\mathbf{\hat{y}} \right] \left[ -\sin(\omega t_r) \mathbf{\hat{x}} + \cos(\omega t_r) \mathbf{\hat{y}} \right] = 0$$
        So the potentials are straightforwardly
        \begin{align*}
            V(\mathbf{r}, t) &= \frac{1}{4\pi \epsilon_{0}} \frac{q}{\sqrt{a^2+z^2}} \\ 
            \mathbf{A}(\mathbf{r}, t) &= \frac{\mu_{0}}{4\pi} \frac{q a \omega}{\sqrt{a^2+z^2}} \left[ -\sin \left( \omega t_r \right)\mathbf{\hat{x}} + \cos \left( \omega t_r \right)\mathbf{\hat{y}} \right]
        \end{align*}

    \end{callout}
\end{homeworkProblem}

\begin{homeworkProblem}
    Consider an infinitely long straight wire carrying a steady current $I$ in the $z$-direction. The vector potential can be written in cylindrical coordinates as:
    \begin{equation}
        \mathbf{A}(s, \phi, z) = \left(A_0 + \frac{\mu_0 I}{2\pi} \ln(s/s_0)\right) \hat{z} + A_{00} \hat{s}
    \end{equation}
    where $s$ is the radial distance vector away from the wire, $A_0$ and $A_{00}$ are constants, as is $s_0$.

    Note that $\nabla \cdot \mathbf{A} \neq 0$.

    \begin{enumerate}
        \item[(a)] Demonstrate that $\mathbf{B} = \nabla \times \mathbf{A}$ gives the correct magnetic field for this case.
            \begin{callout}{Solution:}

                \begin{enumerate}[(a)]
                    \item \textbf{Magnetic field from Maxwell's Equations:}
                        \begin{align*}
                            B \cdot (2\pi s) &= \mu_{0} I &&\text{(outside the wire)} \\
                            B &= \frac{\mu_{0} I}{2\pi s} {\hat{\phi}} &&\text{(Right hand rule)}
                        \end{align*}
            \item \textbf{Magnetic field from curl of potential:}
                \[
                    \nabla \times \mathbf{A} =
                    \begin{vmatrix}
                        \hat{s} & \hat{\phi} & \hat{z} \\
                        \frac{\partial}{\partial s} & \frac{1}{s} \frac{\partial}{\partial \phi} & \frac{\partial}{\partial z} \\
                        A_s & A_{\phi} & A_z
                    \end{vmatrix}
                \]

                \begin{align*}
                \nabla \times \mathbf{A} &=
                \cancelto{0}{\left( \frac{1}{s} \frac{\partial A_z}{\partial \phi} - \frac{\partial A_{\phi}}{\partial z} \right) \hat{s}} 
                - \left( \frac{\partial A_s}{\partial z} - \frac{\partial A_z}{\partial s} \right) \hat{\phi} 
                + \frac{1}{s} \left( \frac{\partial (s A_{\phi})}{\partial s} - \frac{\partial A_s}{\partial \phi} \right) \hat{z} \\ 
                    &= \frac{\mu_{0} I}{2\pi s} \hat{\phi}
                \end{align*}
                \end{enumerate}

            \end{callout}
        \item[(b)] Find a different $\mathbf{A'}$ that satisfies the Coulomb Gauge ($\nabla \cdot \mathbf{A'} = 0$).
            \begin{callout}{Solution:}

            From equation 10.7, 
                \begin{align*}
                    \begin{cases}
                    \mathbf{A'} = \mathbf{A} + \nabla \lambda \\ 
                        V = V - \frac{\partial \lambda}{\partial t}
                    \end{cases} \tag{10.7}
                \end{align*}
                
                This makes the plan pretty simple, take $\nabla \cdot \mathbf{A}$ and then whatever is left lets me choose a term to differentiate, call it $\nabla \lambda$ that I can subtract off, such that $\nabla^2 \lambda = - \nabla \cdot \mathbf{A}$

                \begin{align*}
                    \nabla \cdot \mathbf{A} &= \frac{1}{s}\frac{\partial (sA_s)}{\partial s} + \frac{1}{s}\frac{\partial A_\phi}{\partial \phi} + \frac{\partial A_z}{\partial z} \\ 
                    &= \frac{1}{s} \left( \frac{\partial (s A_{00})}{\partial} \right) = \frac{A_{00}}{s}
                \end{align*}

                Now it's possible to choose a function that satisfies the equation I wrote above.
                
In cylindrical coordinates for a function that depends only on s:
\begin{align*}
 \frac{1}{s}\frac{d}{ds}\left(s\frac{d\lambda}{ds}\right) &= -\frac{A_{00}}{s} \\
 \frac{d}{ds}\left(s\frac{d\lambda}{ds}\right) &= -A_{00} \\
 s\frac{d\lambda}{ds} &= -A_{00}s + C_1 \\
 \frac{d\lambda}{ds} &= -A_{00} + \frac{C_1}{s} \\
 \lambda(s) &= -A_{00}s + C_1\ln(s) + C_2 \\
 \lambda(s) &= -A_{00}s
\end{align*}

Therefore:

 $$\mathbf{A'} = \mathbf{A} + \nabla \lambda = \left(A_0 + \frac{\mu_0 I}{2\pi}\ln(s/s_0)\right)\hat{z} + A_{00}\hat{s} + (-A_{00})\hat{s} = \left(A_0 + \frac{\mu_0 I}{2\pi}\ln(s/s_0)\right)\hat{z}$$

            \end{callout}
        \item[(c)] Find the gauge transformation function $\lambda(r,t)$ that accomplishes part (b).
            \begin{callout}{Solution:}

                The gauge transformation involves a function $\lambda$ such that 
                $$\nabla^{2}\lambda = - \nabla \cdot \mathbf{A}$$
                Which I solve above.

            \end{callout}
    \end{enumerate}
\end{homeworkProblem}

\begin{homeworkProblem}
    Demonstrate that the following spherically symmetric time-dependent function for the scalar potential is a solution of the homogeneous (i.e., source-free) wave equation if $\omega = kc$:
    \begin{equation}
        V(r, t) = \frac{A}{r} \exp\{i(kr - \omega t)\},
    \end{equation}
    where $A$ is a constant with the correct units (V·m).
    \begin{callout}{Solution:}

    The homogeneous wave equation, as discussed in chapter 10, takes the form:
        \begin{align*}
            \nabla^{2}V + \frac{1}{c^{2}} \frac{\partial^{2} V}{\partial t^{2}} &= 0
        \end{align*}

        \begin{align*}
            \nabla^{2}V \underset{\text{spherical}}{\to} \frac{1}{r}\nabla^2(rV)  &= -k^{2} \frac{A}{r} e^{i(kr - \omega t)} \\
            %\nabla \left( ik \frac{A}{r} e^{i(kr - \omega t)} - \frac{A}{r^{2}} e^{i(kr - \omega t)} \right) \\ 
            %&= -k \frac{A}{r} e^{i(kr - \omega t)} - \cancel{ik \frac{A}{r^{2}} e^{i(kr - \omega t)} + ik \frac{A}{r^{2}} e^{i(kr - \omega t)}} + \frac{2A}{r^{3}} e^{i(kr - \omega t)} \\ 
            \frac{1}{c^{2}}\frac{\partial^{2} V}{\partial t^{2}} &= \frac{1}{c^{2}} \frac{\partial V}{\partial t} \left( -i \omega \frac{A}{r} e^{i(kr - \omega t)} \right) \\ 
            &= \frac{\omega^{2}}{c^{2}} \frac{A}{r} e^{i(kr - \omega t)}
        \end{align*}
        Now the differential equation becomes:
        \begin{align*}
                    -k^{2} \left( \cancel{\frac{A}{r}e^{i(kr-\omega t)}} \right) + \frac{\omega^{2}}{c^{2}} \left( \cancel{\frac{A}{r}e^{i(kr-\omega t)}} \right) &= 0 \\
                    -k^{2} + \frac{\omega^{2}}{c^{2}} &= 0
        \end{align*}
        Which is clearly satistfied when $k^{2} = \frac{\omega^{2}}{c^{2}}$.

    \end{callout}
\end{homeworkProblem}

\end{document}
