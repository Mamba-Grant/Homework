\begin{homeworkProblem}
A spherical conductor, of radius $a$, carries a charge $Q$. It is surrounded by linear dielectric material of susceptibility $\chi_e$. Find the energy of this configuration.

\begin{figure}[h]
  \centering
  \includegraphics[width=0.35\textwidth]{../assets/H11P1F1.png}
\end{figure}

\begin{callout}{Solution:}
    
    To begin I need to find the electric displacement of the configuration. This was the configuration of example 4.5, which was derived quickly using Gauss's law and the free charge: 
    \begin{align*}
        \oint \textbf{D} \cdot d \textbf{a} &= Q_{f_{enc}} \\ 
        \textbf{D} \cdot 4 \pi r^2 &= Q \\ 
        \textbf{D} &= \frac{Q}{4\pi r^2} \hat{r}
    \end{align*}
    \textit{Recalling that $\textbf{D}$ is related to susceptibility $\chi_e$ by $\textbf{D}=\epsilon \textbf{E}$, $\epsilon=\epsilon_0(1+\chi_e)$.}

\end{callout}
\end{homeworkProblem}

\begin{homeworkProblem}
Two long coaxial cylindrical metal tubes (inner radius $a$, outer radius $b$) stand vertically in a tank of dielectric oil (susceptibility $\chi_e$, mass density $\rho$). The inner one is maintained at a potential $V$ and the outer one is grounded. To what height ($h$) does the oil rise, in the space between the tubes?

\begin{figure}[h]
  \centering
  \includegraphics[width=0.35\textwidth]{../assets/H11P2F1.png}
\end{figure}

\end{homeworkProblem}

\begin{homeworkProblem}
In 1897 J.J. Thompson "discovered" the electron by measuring the charge-to-mass ratio of "cathode rays" (actually, streams of electrons, with charge $q$ and mass $m$) as follows:
\begin{enumerate}
    \item First, he passed the beam through uniform crossed electric and magnetic fields $\mathbf{E}$ and $\mathbf{B}$ (mutually perpendicular and both of them are perpendicular to the beam), and adjusted the electric field until he got zero deflection. What, then, was the speed of the particles (in terms of $E$ and $B$)?
    \item Then he turned off the electric field, and measured the radius of curvature $R$, of the beam, as deflected by the magnetic field alone. In terms of $E$, $B$, and $R$, what is the charge-to-mass ratio ($q/m$) of the particles?
\end{enumerate}
\end{homeworkProblem}
