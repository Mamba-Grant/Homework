\documentclass{article}

\title{Determining the Index of Refraction of Solutions}
\author{Grant Saggars}
\date{\today}

\usepackage{amsmath}
\usepackage{import}
\usepackage{pdfpages}
\usepackage{transparent}
\usepackage{xcolor}
\usepackage{framed}
\usepackage{enumerate}
\usepackage{cancel}
\usepackage{silence}
\usepackage{multicol}
\usepackage{lipsum}  
\usepackage{caption}
\WarningsOff*
% \ErrorsOff*
\usepackage{tabularx}
\usepackage{geometry}
\usepackage[utf8]{inputenc}
\usepackage{pmboxdraw}
\usepackage{XCharter}
% \usepackage{fontspec}
% \setmonofont{SourceCodePro-Medium}

\definecolor{shadecolor}{RGB}{248,248,248}

\newenvironment{callout}[1] {\begin{shaded*} \textbf{#1}} {\end{shaded*}}

\begin{document}

\maketitle

\section{Objective}

The goal of this experiment was to determine and compare the index of refraction (IOR) for two solutions. This was done using a laser and trigonometric relationships which can relate the path vector of the laser through a substance with a known IOR to the altered sum of path vectors. All that is needed to derive the IOR of such substances is the angle at which the laser is pointed at the screen, and the points at which the laser is visible on the screen.

\subsection{Equipment}
\begin{enumerate}
	\item Laser (wavelength is irrelevant)
	\item Screen on which the light is projected
	\item Bi-plate petri-dish
	\item Substance A: water
	\item Substance B: a solution of water and coffee creamer
\end{enumerate}

\begin{figure}[h]
	\centering
	\includegraphics[width=0.5\textwidth]{fig.png}
	\caption{Setup Diagram}
	\label{Diagram}
\end{figure}

\section{Setup \& Theory}

\quad It is difficult to optically measure the path taken by the laser as it travels through our substances because they are transparent, so it would be more sensible to derive these path vectors trigonometrically. Measurements of position and angle of light were taken before and after inserting the pitri-dish containing the solution into the path of the laser, creating a setup which is represented in figure (1). Mathematically, this leaves us with a system with 5 degrees of freedom and 6 equations (2).

The measured values were $d_{tot}$, $y_{tot}$, $d_2$, and $d_3$. From this I expected to calculate $\theta_1$ (although this was not possible as discussed in section 2.2). These were measured in cm, with an uncertainty of $\pm 0.05$ cm. Three trials were conducted at different values of $\theta_1$ for each substance.

\newpage
\subsection{Relations}
\begin{multicols}{2}
	\begin{enumerate}[(1)]
		\item Using Snell's Law: ($n_r\sin\theta_r=n_i\sin\theta_i$), I relate the index of refraction between two mediums to the angles at which light travels. Air is taken to have an IOR of one. The angle $\theta_1$ is measured from the horizontal, so $\cos(\theta_1)$ is used instead of $\sin(\theta_1)$
		\item The vertical change in medium two.
		\item Again using Snell's Law to relate the IOR of medium two to medium three. Again, air is assumed to have an IOR of one. Also, angle is again measured from the horizontal.
		\item The vertical change in medium three.
		\item The sum of vertical changes (y-components) should equal the measured y-value on the screen.
		\item The sum of the angles of incidence and refraction for any medium interface is equal to 180 degrees, since they are on the same side of the normal line.

	\end{enumerate}
	\columnbreak
	\begin{callout}{}
		\begin{subequations}\label{eq:system}
			\begin{align*}
				\cos\left(\theta_1\right)     & =n_2\sin\left(\theta _2\right)                                         \\
				y_2                           & =d_2\left(\sin\left(\theta_1\right)-\sin\left(\theta _2\right)\right)  \\
				n_2\cos\left(\theta _2\right) & =\sin\left(\theta _3\right)                                            \\
				y_3                           & =d_3\left(\sin\left(\theta _2\right)-\sin\left(\theta _3\right)\right) \\
				y_{tot}                       & =y_1+y_2+y_3                                                           \\
				\pi                           & = \theta_1 + \theta_2 + \theta_3
			\end{align*}
			\captionof{figure}{System of Equations}
		\end{subequations}
	\end{callout}
\end{multicols}

\subsection{Calculations}

I should have set up this experiment differently, since I am not capable of solving this system analytically. A better approach to setting up this experiment would have been putting the edge of the petri-dish against the screen, so that the angle made by the laser within the water could be directly calculated using a similar approach. I was, however, capable of finding numerical solutions to this system. The julia code can be found at the end of this paper. While I believe the result to be realtively reasonable, I was not capable of propagating uncertainty this way.

\section{Data \& Results}

All samples were taken with the laser at $87 \pm 0.071$ cm from the screen. The petri-dish containing the water was placed at $27.125 \pm 0.071$ cm from the laser (measured to the far end of the dish).

My estimation of the index of refraction of mallott tap water is approximately 1.26192, and the index of refraction for the water with creamer is 1.26198. I believe these to be within the realm of plausibility, since distilled water is known to have an index of refraction of approximately 1.33. Additionally, the measurements were similar between solutions. I would have expected instrumental uncertainty to be the greatest source of error, however as mentioned previously it was not possible to calculate this.

\begin{table}[ht]
	\centering
	\caption{Measurement Data for Different Angles and Liquids}
	\begin{tabular}{|c|c|c|c|}
		\hline
		\textbf{$\theta$} & \textbf{Setup}  & \textbf{Fringe Distance $\pm$ 0.05 (cm)} & \textbf{Refracted Fringe Distance $\pm$ 0.05 (cm)} \\
		\hline
		1                 & Tap Water       & 3.25, 3.30, 3.40                         & -3.9, -4.0, 4.0                                    \\
		\hline
		2                 & Tap Water       & 0.91, 0.90, 0.89                         & -2.09, -1.95, -1.95                                \\
		\hline
		3                 & Tap Water       & 2.05, 1.9, 1.9                           & -2.9, -2.94, -2.9                                  \\
		\hline
		1                 & Creamer + Water & 2.85, 2.75, 2.8                          & -3.9, -3.95, -3.95                                 \\
		\hline
		2                 & Creamer + Water & 1.55, 1.5, 1.57                          & -2.95, -2.89, -2.9                                 \\
		\hline
		3                 & Creamer + Water & 1.21, 1.98, 1.99                         & -1.99, -1.98, -1.95                                \\
		\hline
	\end{tabular}
	\label{tab:data}
\end{table}

\begin{table}[h]
	\centering
	\begin{tabular}{|c|c|c|c|c|c|c|}
		\hline
		$n$     & $\theta_2$ (rad) & $\theta_3$ (rad) & $y_2$ (cm) & $y_3$ (cm) & $\delta f_95$  & $\sigma$ \\
		\hline
		1.26192 & 0.9139910        & 2.26194          & -3.409     & 1.291      & [1.256, 1.268] & 0.00318  \\
		1.26198 & 0.9139369        & 2.26180          & -3.402     & 1.290      & [1.256, 1.268] & 0.00316  \\
		\hline
	\end{tabular}
	\caption{Results}
\end{table}
%
% \newpage
% \begin{verbatim}
%     using Statistics
%     using NLsolve
%
%     const pitri_width = 8.75e-2
%     const Dtot = 87e-2
%     const D1 = 95.95e-2 - 68.25e-2 
%     const D2 = 68.25e-2 - 64e-2
%     const D3 = 64e-2 - 8.5e-2
%
% # Extract data into arrays
%     fresh_y0 = [[-3.9, -4, -4], [-2.09, -1.95, -1.95], [-2.9, -2.94, -2.9]] .* 10^-2 
%     fresh_y1 = [[3.25, 3.30, 3.40], [0.91, 0.90, 0.89], [2.05, 1.9, 1.9]] .* 10^-2
%
%     creamer_y0 = [[-3.9, -3.95, -3.95], [-2.95, -2.89, -2.9], [-1.99, -1.98, -1.95]] 
%     .* 10^-2
%     creamer_y1 = [[2.85, 2.75, 2.8], [1.55, 1.5, 1.57], [1.21, 1.98, 1.99]]
%     .* 10^-2
%
% # Calculate means
%     fresh_mean_y0 = [mean(y0) for y0 in fresh_y0] 
%     fresh_mean_y1 = [mean(y1) for y1 in fresh_y1]
%
%     creamer_mean_y0 = [mean(y0) for y0 in creamer_y0]
%     creamer_mean_y1 = [mean(y1) for y1 in creamer_y1]
%
% # Function to solve equations
%     function solve_equations(mean_y0, θ1)
%        initial_guess = [1.0, 0.0, 0.0, 0.1, 0.1, 1.0]
%        
%        function equations!(F, x)
%           n2, θ2, θ3, y2, y3 = x
%        
%           F[1] = cos(θ1) - n2 * sin(θ2)  
%           F[2] = y2 - D2 * (sin(θ1) - sin(θ2))
%           F[3] = n2 * cos(θ2) - sin(θ3)
%           F[4] = y3 - D3 * (sin(θ2) - sin(θ3)) 
%           F[5] = mean_y0 - (y2 + y3 + D1 * tan(θ1))
%           F[6] = π - (θ1 + θ2 + θ3)
%        end
%        
%        return nlsolve(equations!, initial_guess)
%     end
%
% # Calculate θ1 and solve for each data set
%     θ1_fresh = [atan(mean_y0 / Dtot) for mean_y0 in fresh_mean_y0] 
%     results_fresh = [solve_equations(mean_y0, θ1) for (mean_y0, θ1) in 
%     zip(fresh_mean_y0, θ1_fresh)]
%
%     θ1_creamer = [atan(mean_y0 / Dtot) for mean_y0 in creamer_mean_y0]
%     results_creamer = [solve_equations(mean_y0, θ1) for (mean_y0, θ1) in 
%     zip(creamer_mean_y0, θ1_creamer)]
%
% # Extract n2 values
%     r1 = mean([result.zero for result in results_fresh])  
%     r2 = mean([result.zero for result in results_creamer])
% \end{verbatim}
\end{document}
