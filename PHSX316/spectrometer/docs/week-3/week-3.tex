\documentclass{article}

\title{Week 3 Writeup}
\author{Grant Saggars}
\date{\today}

\usepackage{amsmath}
\usepackage{import}
\usepackage{pdfpages}
\usepackage{transparent}
\usepackage{xcolor}
\usepackage{framed}
\usepackage{enumerate}
\usepackage{geometry}
\usepackage{cancel}
\usepackage{multicol}
\usepackage{lipsum}  
\usepackage{caption}
\usepackage{float}
\usepackage{bbold}
% \usepackage{fontspec}

% \setmainfont{BespokeSerif-Regular}
\definecolor{shadecolor}{RGB}{235,235,235}

\geometry{top=1in, bottom=1in, left=1in, right=1in}
\newenvironment{callout}[1] {\begin{shaded*} \textbf{#1}} {\end{shaded*}}

%%%%%%%%%%%%%%%%%%%%%%%%
% DOCUMENT BEGINS HERE %
%%%%%%%%%%%%%%%%%%%%%%%%

\begin{document}
\maketitle

\section{Housing \& Mounts}
I finished and printed my first mounts over break. I spent this week revising my designs and I should be able to print everything monday of next week to finish assembly, since my tolerances were slightly off, and I had to rethink my camera mount.

\section{Software}
The rest of my weekend will be spent implementing calibration tools. This simply involves (1) determining a way to correlate pixel exposure to spectral energy intensity for my camera, (2) converting from pixel location to wavelength. These will all end up mathematically being a projection from the values in my buffers to the intensity/wavelength. Because of the nature of my data, this "projection" will be a very straightforward multiplication from pixel index to the desired unit value. 

\section{Timeline}
I feel behind schedule on where I would have liked to have been. Construction of the housing and case has taken two weeks longer than I would have liked, due to some delays getting full access to the 3D printer and design difficulties with the scale of my webcam lens. As a result, next week will have to be very busy to compensate. The weather fortunately seems quite good, so I should be able to get the measurements I need.


\end{document}
