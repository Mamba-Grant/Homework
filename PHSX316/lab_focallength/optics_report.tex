%%%%%%%%%%%%%%%%%%%%%%%%%%%%% Define Article %%%%%%%%%%%%%%%%%%%%%%%%%%%%%%%%%%
\documentclass{article}
%%%%%%%%%%%%%%%%%%%%%%%%%%%%%%%%%%%%%%%%%%%%%%%%%%%%%%%%%%%%%%%%%%%%%%%%%%%%%%%

%%%%%%%%%%%%%%%%%%%%%%%%%%%%% Using Packages %%%%%%%%%%%%%%%%%%%%%%%%%%%%%%%%%%
\usepackage{geometry}
\usepackage{graphicx}
\usepackage{amssymb}
\usepackage{amsmath}
\usepackage{amsthm}
\usepackage{empheq}
\usepackage{mdframed}
\usepackage{booktabs}
\usepackage{lipsum}
\usepackage{graphicx}
\usepackage{color}
\usepackage{psfrag}
\usepackage{pgfplots}
\usepackage{bm}
\usepackage{tcolorbox}
\usepackage{caption}
%%%%%%%%%%%%%%%%%%%%%%%%%%%%%%%%%%%%%%%%%%%%%%%%%%%%%%%%%%%%%%%%%%%%%%%%%%%%%%%

\newtcolorbox{equationbox}{
    colback=gray!10, % Background color (light gray)
    colframe=white, % Frame color (white)
    boxrule=0pt, % No frame around the box
    arc=0pt, % No rounded corners
    left=5pt, % Left margin
    right=5pt, % Right margin
    top=5pt, % Top margin
    bottom=5pt, % Bottom margin
    boxsep=0pt, % Space between the box and content
    before=\begin{equation*}, % Start equation* environment
    after=\end{equation*}, % End equation* environment
}

\NewEnviron{alignedequationbox}{%
    \begin{tcolorbox}[
        colback=gray!10, % Background color (light gray)
        colframe=white, % Frame color (white)
        boxrule=0pt, % No frame around the box
        arc=0pt, % No rounded corners
        left=5pt, % Left margin
        right=5pt, % Right margin
        top=5pt, % Top margin
        bottom=5pt, % Bottom margin
        boxsep=0pt, % Space between the box and content
    ]
    \begin{align*}
    \BODY
    \end{align*}
    \end{tcolorbox}
}

%%%%%%%%%%%%%%%%%%%%%%%%%% Page Setting %%%%%%%%%%%%%%%%%%%%%%%%%%%%%%%%%%%%%%%
\geometry{a4paper}

%%%%%%%%%%%%%%%%%%%%%%%%%% Define some useful colors %%%%%%%%%%%%%%%%%%%%%%%%%%
\definecolor{ocre}{RGB}{243,102,25}
\definecolor{mygray}{RGB}{243,243,244}
\definecolor{deepGreen}{RGB}{26,111,0}
\definecolor{shallowGreen}{RGB}{235,255,255}
\definecolor{deepBlue}{RGB}{61,124,222}
\definecolor{shallowBlue}{RGB}{235,249,255}
%%%%%%%%%%%%%%%%%%%%%%%%%%%%%%%%%%%%%%%%%%%%%%%%%%%%%%%%%%%%%%%%%%%%%%%%%%%%%%%

%%%%%%%%%%%%%%%%%%%%%%%%%% Define an orangebox command %%%%%%%%%%%%%%%%%%%%%%%%
\newcommand\orangebox[1]{\fcolorbox{ocre}{mygray}{\hspace{1em}#1\hspace{1em}}}
%%%%%%%%%%%%%%%%%%%%%%%%%%%%%%%%%%%%%%%%%%%%%%%%%%%%%%%%%%%%%%%%%%%%%%%%%%%%%%%

%%%%%%%%%%%%%%%%%%%%%%%%%%%% English Environments %%%%%%%%%%%%%%%%%%%%%%%%%%%%%
\newtheoremstyle{mytheoremstyle}{3pt}{3pt}{\normalfont}{0cm}{\rmfamily\bfseries}{}{1em}{{\color{black}\thmname{#1}~\thmnumber{#2}}\thmnote{\,--\,#3}}
\newtheoremstyle{myproblemstyle}{3pt}{3pt}{\normalfont}{0cm}{\rmfamily\bfseries}{}{1em}{{\color{black}\thmname{#1}~\thmnumber{#2}}\thmnote{\,--\,#3}}
\theoremstyle{mytheoremstyle}
\newmdtheoremenv[linewidth=1pt,backgroundcolor=shallowGreen,linecolor=deepGreen,leftmargin=0pt,innerleftmargin=20pt,innerrightmargin=20pt,]{theorem}{Theorem}[section]
\theoremstyle{mytheoremstyle}
\newmdtheoremenv[linewidth=1pt,backgroundcolor=shallowBlue,linecolor=deepBlue,leftmargin=0pt,innerleftmargin=20pt,innerrightmargin=20pt,]{definition}{Definition}[section]
\theoremstyle{myproblemstyle}
\newmdtheoremenv[linecolor=black,leftmargin=0pt,innerleftmargin=10pt,innerrightmargin=10pt,]{problem}{Problem}[section]
%%%%%%%%%%%%%%%%%%%%%%%%%%%%%%%%%%%%%%%%%%%%%%%%%%%%%%%%%%%%%%%%%%%%%%%%%%%%%%%

%%%%%%%%%%%%%%%%%%%%%%%%%%%%%%% Plotting Settings %%%%%%%%%%%%%%%%%%%%%%%%%%%%%
\usepgfplotslibrary{colorbrewer}
\pgfplotsset{width=8cm,compat=1.9}
%%%%%%%%%%%%%%%%%%%%%%%%%%%%%%%%%%%%%%%%%%%%%%%%%%%%%%%%%%%%%%%%%%%%%%%%%%%%%%%

%%%%%%%%%%%%%%%%%%%%%%%%%%%%%%% Title & Author %%%%%%%%%%%%%%%%%%%%%%%%%%%%%%%%
\title{Estimating the Focal Length of a Thin Lens}
\author{Grant Saggars}
%%%%%%%%%%%%%%%%%%%%%%%%%%%%%%%%%%%%%%%%%%%%%%%%%%%%%%%%%%%%%%%%%%%%%%%%%%%%%%%

\begin{document}
    \maketitle
    \section{Objective}
        I was given some lenses from a manufacturer, who stated that the focal length of the lens was exactly 15 cm with no given uncertainty. Put simply, I want to check this number. The process for measuring the focal length utilized simple, inexpensive lab equipment. 

        % HOW DO MEASURE, RELATIONS AND EQUATIONS

    \section{Setup \& Theory}

        \begin{alignedequationbox}
            \displaystyle\frac{1}{f} &= \frac{1}{d_o} + \frac{1}{d_i} \implies f = \left(\frac{1}{f}\right)^{-1} \tag{1}
            \intertext{where:}
            d_o & \text{\footnotesize\hspace{1em} is the distance from the lens to the object, } \\
            d_i & \text{\footnotesize\hspace{1em} is the distance from the lens to the image }
        \end{alignedequationbox}

I mounted a light which projected an image through my lens and onto a flat plate. I adjusted $v_o$ and $v_i$ until the image was in focus on the plate. 

% The relation: $\frac{h_{o}}{h_{i}} = \frac{d_{o}}{d_{i}}$ was used to determine the accuracy to which the image was in focus, since we did not have an ideal means to determine precisely measure the focus of the image. This is one source of error (CAN I EVEN MATHEMATICALLY MEASURE THIS??)


    \section{Results}

    \begin{minipage}{0.60\textwidth}
        \centering
        \begin{tabular}{cccc}
            \toprule
            $h_{o}$ (cm) & $h_{i}$ (cm) & $d_{o}$ (cm) & $d_{i}$ (cm) \\
            \midrule
            $1.55 \pm 0.05$ & $2.15 \pm 0.05$ & $30 \pm 0.05$ & $38.24 \pm 0.05$ \\
            $1.55 \pm 0.05$ & $1.5 \pm 0.05$ & $32.4 \pm 0.05$ & $35 \pm 0.05$ \\
            $1.55 \pm 0.05$ & $1.1 \pm 0.05$ & $28.7 \pm 0.05$ & $40 \pm 0.05$ \\
            $1.55 \pm 0.05$ & $1.9 \pm 0.05$ & $32 \pm 0.05$ & $35.7 \pm 0.05$ \\
            $1.55 \pm 0.05$ & $1.3 \pm 0.05$ & $37 \pm 0.05$ & $30.8 \pm 0.05$ \\
            $1.55 \pm 0.05$ & $1.9 \pm 0.05$ & $31 \pm 0.05$ & $36.75 \pm 0.05$ \\
            $1.55 \pm 0.05$ & $1.67 \pm 0.05$ & $33 \pm 0.05$ & $34.4 \pm 0.05$ \\
            \bottomrule
        \end{tabular}
        \captionof{table}{Measured Data}
        \label{tab:table1}
    \end{minipage}
    \hfill
    \begin{minipage}{0.35\textwidth}
        \centering
        \begin{tabular}{cc}
            \toprule
            $v$ (cm) & $u$ (cm) \\
            \midrule
            $16.90$ & \\
            $16.82$ & \\
            $16.70$ & \\
            $16.87$ & \\
            $16.81$ & \\
            $16.82$ & \\
            $16.84$ & \\
            \bottomrule
        \end{tabular}
        \captionof{table}{Results}
        \label{tab:table2}
    \end{minipage}   

        UNCERTAINTY SOURCES
    \section{Discussion}
\end{document}
